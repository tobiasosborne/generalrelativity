%!TEX root = ../GeneralRelativity.tex
% ──────────────────────────────────────────────────────────────
%  Lecture 2 — The Equivalence Principle and Mach's Principle
% ──────────────────────────────────────────────────────────────

\section{The equivalence principle and Mach's principle}%
\label{sec:equivalence-principle}

In the previous lecture we reviewed the fundamental assumptions of
Newtonian physics---in particular, the preferred reference frame and
the three conceptually distinct notions of mass.  We now turn to the
two big principles that motivated Einstein to formulate general
relativity: the \emph{equivalence principle} and \emph{Mach's
principle}.

\subsection{Einstein's equivalence principle (EEP)}

The first, and arguably most important, principle underlying the
formulation of general relativity is the Einstein equivalence
principle.  In its modern form, EEP consists of three parts:

\begin{enumerate}
  \item \textbf{Universality of free fall (UFF)}
    (also known as the \emph{weak equivalence principle}, WEP).\\
    Not rejected (a key distinction: in physics we cannot \emph{confirm}
    a hypothesis, only fail to reject it) e.g.\ in torsion balance
    experiments (cf.~\S\ref{sec:UFF}).

  \item \textbf{Local Lorentz invariance (LLI).}\\
    In a freely falling reference frame, the laws of non-gravitational
    physics are those of special relativity---there is no preferred
    reference frame.  Recall that the existence of a preferred frame
    was a fundamental assumption of Newtonian physics
    (\S\ref{sec:prerelativity}); this is one of the assumptions one
    must already abandon in the transition to special relativity.
    In particular, the laws of physics do not depend on the orientation
    or velocity of the observer.

    \smallskip
    \noindent Best bounds from Michelson--Morley--type experiments:
    \begin{equation}\label{eq:LLI-bound}
      \frac{\Delta c}{c} < 5 \times 10^{-17}\,.
    \end{equation}

  \item \textbf{Local position invariance (LPI).}\\
    Special relativity holds independent of spacetime position
    (no ``preferred'' location).
\end{enumerate}

\begin{intuition}[The three pillars of EEP]
  UFF says \emph{all bodies fall the same way}.
  LLI says \emph{physics looks the same regardless of your velocity}.
  LPI says \emph{physics looks the same regardless of where you are}.
  Together, they force gravity to be a property of spacetime
  geometry rather than an ordinary force.
\end{intuition}

\subsubsection{LPI and gravitational redshift}

LPI is equivalent (exercise) to the \emph{universality of gravitational
redshift} (UGR): clock rates are universally affected by the
gravitational field.

\begin{exercise}\label{ex:grav-redshift}
  Show that for small velocities and weak gravitational fields,
  the fractional change in clock rate between two positions
  $\vb{x}_1$ and $\vb{x}_2$ is
  \begin{equation}\label{eq:grav-redshift}
    \eqbox{\frac{\Delta\nu}{\nu}
      = \frac{\Delta\phi}{c^2}}\,,
  \end{equation}
  where $\nu(\vb{x}_1)$, $\nu(\vb{x}_2)$ are the clock rates at
  positions $1,2$ and
  $\Delta\phi = \phi(\vb{x}_1) - \phi(\vb{x}_2)$.
  \emph{Hint:} There are (at least) two approaches---a Doppler-shift
  argument, or an argument using the quantum expression $E = h\nu$ for
  the energy of a photon and tracking how it changes at different
  points in the gravitational field.
\end{exercise}

\noindent\textbf{Estimate:} On Earth's surface, for a height difference
$\norm{\vb{x}_1 - \vb{x}_2} \sim 30\;\mathrm{cm}$, this effect is
$\Delta\phi/c^2 \sim 10^{-17}$---and has been measured!  Modern atomic
clocks are sensitive enough that gravitational redshift is now
routinely measurable over laboratory-scale height differences.

\begin{intuition}[Gravity slows clocks]
  A clock at the bottom of a gravitational potential well runs slower
  than a clock at the top.  GPS satellites must correct for this:
  their clocks run faster (by $\sim 45\;\mu\mathrm{s}/\mathrm{day}$)
  than ground-based clocks.  Without this correction, GPS positions
  would drift by $\sim 10\;\mathrm{km}/\mathrm{day}$.
\end{intuition}

% ──────────────────────────────────────────────────────────────
\subsection{From EEP to geometry}

\textbf{Observation:} Using EEP, it is possible to ``derive'' (or at
least argue convincingly) that gravity should be described
\emph{geometrically}
(K.~Thorne, D.~Lee, \& A.~Lightman,
\emph{Phys.\ Rev.\ D}, \textbf{7}, 3563--3578, 1973).

One should be careful here: formulating a physical theory in
geometric language is, by itself, no particular magic.  Newton's
mechanics can be formulated geometrically (via symplectic geometry),
as can thermodynamics, statistical mechanics, and quantum mechanics
(via K\"ahler manifolds).  If you are willing to work hard enough,
just about \emph{any} physical theory admits a geometric formulation.

What makes general relativity special is not merely that it
\emph{has} a geometric formulation, but what that geometry
\emph{means} physically:

\begin{keyresult}[The geometric consequence of EEP]
  All matter---electrons, protons, gluons, the Higgs field---couples
  to the \emph{same geometry}, namely that of spacetime.  A single
  metric suffices to describe all of these couplings simultaneously.
  This universal coupling is what distinguishes general relativity
  from any other theory that happens to admit a geometric formulation.
\end{keyresult}

% ──────────────────────────────────────────────────────────────
\subsection{Calculations with EEP}

\textbf{Statement:} No external, static, homogeneous gravitational
field can be detected in a freely falling elevator.

\smallskip
This is one of Einstein's classic thought experiments---``Einstein's
elevator.''  Note the qualifiers \emph{static} and
\emph{homogeneous}: if the gravitational field varies in space (is
not translation-invariant), then tidal forces arise, and these
\emph{can} be detected even in a freely falling frame.  This
distinction will become important when we discuss curvature.

Suppose we have $N$ (non-relativistic) particles moving in a pair
force field $\vb{F}(\vb{x}_j - \vb{x}_k)$ and an external gravitational
field $\vb{g}(\vb{x}) = \vb{g}$ (constant), with $\vb{F}(\vb{0}) = \vb{0}$.

\noindent Equations of motion:
\begin{equation}\label{eq:eom-N-particles}
  m_j\,\frac{d^2\vb{x}_j}{dt^2}
    = m_j\,\vb{g}
      + \sum_{k=1}^{N} \vb{F}(\vb{x}_j - \vb{x}_k)\,.
\end{equation}

We now make a non-Galilean change to an \emph{accelerating
(freely falling) frame}:
\begin{equation}\label{eq:accelerating-frame}
  \vb{x}' = \vb{x} - \tfrac{1}{2}\,\vb{g}\,t^2\,,\qquad
  t' = t\,.
\end{equation}

Then the equations of motion become (exercise):
\begin{equation}\label{eq:eom-freefall}
  \eqbox{m_j\,\frac{d^2\vb{x}'_j}{dt'^2}
    = \sum_{k} \vb{F}(\vb{x}'_j - \vb{x}'_k)}\,.
\end{equation}
The gravitational field has been \emph{eliminated}---the freely falling
frame sees only the inter-particle forces.  This is a first concrete
demonstration of EEP within Newtonian physics.

% ──────────────────────────────────────────────────────────────
\subsection{Gravitational forces and freely falling coordinates}

We now tell essentially the same story, but within special
relativity---a more demanding calculation, and a good check on
one's special-relativistic mechanics.  These calculations are meant
to give facility with non-Galilean, non-Lorentzian coordinate
changes; along the way, many of the symbols of differential geometry
will already start to appear in this quite elementary setting.

According to EEP, for \emph{any} particle moving purely under the
influence of a gravitational field, there exists a freely falling
coordinate system $(\xi^0,\,\xi^1,\,\xi^2,\,\xi^3)$ such that
\begin{equation}\label{eq:force-free}
  \eqbox{\frac{d^2\xi^\alpha}{d\tau^2} = 0}
  \qquad (\equiv\;\text{force-free motion})\,,
\end{equation}
where $\tau$ is the \emph{proper time}, defined by
\begin{equation}\label{eq:proper-time-flat}
  d\tau^2 = -\eta_{\alpha\beta}\,d\xi^\alpha\,d\xi^\beta\,,
\end{equation}
or equivalently,
\begin{equation}\label{eq:4vel-normalisation}
  -1 = \eta_{\alpha\beta}\,
    \frac{d\xi^\alpha}{d\tau}\,\frac{d\xi^\beta}{d\tau}\,.
\end{equation}

Here $\eta_{\alpha\beta}$ is the \emph{Minkowski metric}:
\begin{equation}\label{eq:minkowski}
  \eta_{\alpha\beta}
  = \diag(-1,\,+1,\,+1,\,+1)\,.
\end{equation}

\noindent\textbf{Einstein's summation convention:}
\begin{equation}
  u_\alpha\,\xi^\alpha
  \;\equiv\; \sum_{\alpha=0}^{3} u_\alpha\,\xi^\alpha\,.
\end{equation}
Repeated indices (one ``upstairs,'' one ``downstairs'') are summed over.

% ──────────────────────────────────────────────────────────────
\subsection{The geodesic equation}

Suppose $x^\mu$ denotes some other---completely general---set of
coordinates: a lab frame, a rotating frame, a frame ``jumping up and
down on a trampoline''---anything at all:

\begin{center}
\begin{tikzpicture}[scale=0.9]
  % Freely-falling frame (xi)
  \begin{scope}
    \draw[axisstyle, spacecadet] (0,0) -- (0,2.5) node[left]{$\xi^0$};
    \draw[axisstyle, spacecadet] (0,0) -- (2.2,0) node[below]{$\xi^1$};
    \node[font=\sf\small, text=cgblue] at (1.1,2.6)
      {freely falling};
    % Grid lines
    \foreach \i in {0.5,1.0,...,2.0} {
      \draw[thin, spacecadet!20] (0,\i) -- (2.0,\i);
      \draw[thin, spacecadet!20] (\i,0) -- (\i,2.0);
    }
  \end{scope}
  % Arrow between
  \draw[thick, -{Stealth[length=6pt]}, cgblue] (2.8,1.0) -- (4.2,1.0)
    node[midway, above, font=\small]{$\xi^\alpha(x^\mu)$};
  \draw[thick, -{Stealth[length=6pt]}, cgblue] (4.2,0.6) -- (2.8,0.6)
    node[midway, below, font=\small]{$x^\mu(\xi^\alpha)$};
  % Lab frame (x)
  \begin{scope}[shift={(5,0)}]
    \draw[axisstyle, spacecadet] (0,0) -- (0,2.5) node[left]{$x^0$};
    \draw[axisstyle, spacecadet] (0,0) -- (2.2,0) node[below]{$x^1$};
    \node[font=\sf\small, text=cgblue] at (1.1,2.6) {lab frame};
    % Curved grid lines
    \foreach \i in {0.5,1.0,...,2.0} {
      \draw[thin, spacecadet!20] (0,\i) to[out=5,in=175] (2.0,{\i+0.15});
      \draw[thin, spacecadet!20] (\i,0) to[out=85,in=-85] ({\i-0.1},2.0);
    }
  \end{scope}
\end{tikzpicture}
\end{center}

The freely falling coordinates are functions of the lab coordinates:
\[
  \xi^\alpha = \xi^\alpha(x^0,\,x^1,\,x^2,\,x^3)\,.
\]

The force-free equation~\eqref{eq:force-free} becomes
\begin{equation}
  0 = \frac{d^2\xi^\alpha}{d\tau^2}
    = \frac{d}{d\tau}\!\left(
        \pd{\xi^\alpha}{x^\mu}\,\frac{dx^\mu}{d\tau}
      \right)
    = \pd{\xi^\alpha}{x^\mu}\,\frac{d^2 x^\mu}{d\tau^2}
      + \frac{\partial^2\xi^\alpha}{\partial x^\mu\,\partial x^\nu}\,
        \frac{dx^\mu}{d\tau}\,\frac{dx^\nu}{d\tau}\,.
\end{equation}

Multiplying by $\pd{x^\lambda}{\xi^\alpha}$ and using
$\pd{\xi^\alpha}{x^\mu}\pd{x^\lambda}{\xi^\alpha} = \delta^\lambda{}_\mu$,
we obtain the \textbf{geodesic equation}:
\begin{equation}\label{eq:geodesic}
  \eqbox{
    \frac{d^2 x^\lambda}{d\tau^2}
    + \chris{\lambda}{\mu\nu}\,
      \frac{dx^\mu}{d\tau}\,\frac{dx^\nu}{d\tau}
    = 0}\,,
\end{equation}
where the \emph{affine connection} (Christoffel symbol) is
\begin{equation}\label{eq:christoffel-def}
  \chris{\lambda}{\mu\nu}
    = \pd{x^\lambda}{\xi^\alpha}\,
      \frac{\partial^2\xi^\alpha}{\partial x^\mu\,\partial x^\nu}\,.
\end{equation}
The affine connection encodes how the freely falling coordinates
depend on our (possibly wild) lab coordinates, together with a
Jacobian factor.  When we come to formulate differential geometry
properly (starting in lecture~3), we will encounter the geodesic
equation and affine connections in much greater generality---but
they already appear here, in this entirely elementary calculation.

\begin{intuition}[Geodesics on a curved surface]
  To build intuition for the geodesic equation, consider a 2D
  surface embedded in~$\Rn{3}$.  The shortest paths on the
  surface---the geodesics---are \emph{not} straight lines in
  the ambient space; they bend to stay on the surface.
  The deviation from a straight line is governed by the
  Christoffel symbols, which encode the surface's curvature.
\end{intuition}

\begin{figure}[htbp]
  \centering
  % fig_lec02_geodesic_bump.tex — Parallel geodesics on a Gaussian-bump surface
% Data: generated by scripts/sim_lec02.jl
% 9 initially parallel geodesics (all moving in +x) deflected by the bump

% Panel (a): 3D surface with geodesics
\begin{center}
\begin{tikzpicture}
\begin{axis}[
    gr3d,
    xlabel={$x$}, ylabel={$y$}, zlabel={$z$},
    title={Initially parallel geodesics on a Gaussian bump},
    view={-35}{32},
    zmin=-0.05, zmax=1.15,
    xmin=-3, xmax=3,
    ymin=-3, ymax=3,
]
    % Surface mesh (faint)
    \addplot3[
        surf,
        opacity=0.35,
        shader=interp,
        colormap name=grsurface,
        forget plot,
        mesh/rows=50,
    ] table {data/lec02_surface.dat};

    % 9 geodesics: colour gradient from cgblue (bottom) through munsell to spacecadet (top)
    \addplot3[spacecadet!70!cgblue, line width=1.0pt, no markers, forget plot]
        table {data/lec02_geodesic_1.dat};
    \addplot3[spacecadet!50!cgblue, line width=1.0pt, no markers, forget plot]
        table {data/lec02_geodesic_2.dat};
    \addplot3[cgblue!80!munsell, line width=1.0pt, no markers, forget plot]
        table {data/lec02_geodesic_3.dat};
    \addplot3[cgblue, line width=1.0pt, no markers, forget plot]
        table {data/lec02_geodesic_4.dat};
    \addplot3[munsell, line width=1.3pt, no markers, forget plot]
        table {data/lec02_geodesic_5.dat};
    \addplot3[cgblue, line width=1.0pt, no markers, forget plot]
        table {data/lec02_geodesic_6.dat};
    \addplot3[cgblue!80!munsell, line width=1.0pt, no markers, forget plot]
        table {data/lec02_geodesic_7.dat};
    \addplot3[spacecadet!50!cgblue, line width=1.0pt, no markers, forget plot]
        table {data/lec02_geodesic_8.dat};
    \addplot3[spacecadet!70!cgblue, line width=1.0pt, no markers, forget plot]
        table {data/lec02_geodesic_9.dat};
\end{axis}
\end{tikzpicture}

\bigskip

% Panel (b): Top-down comparison — geodesics vs undeflected straight lines
\begin{tikzpicture}
\begin{axis}[
    grplot,
    width=8cm, height=8cm,
    xlabel={$x$}, ylabel={$y$},
    title={Geodesic (solid) vs.\ undeflected (dashed)},
    xmin=-3, xmax=3,
    ymin=-2.5, ymax=2.5,
    axis equal,
]
    % Bump contour (faint circle at 1σ)
    \draw[spacecadet!15, thick] (axis cs:0,0) circle[radius=100];
    \addplot[spacecadet!20, domain=0:360, samples=72, no markers, thick, forget plot]
        ({cos(x)}, {sin(x)});
    \addplot[spacecadet!10, domain=0:360, samples=72, no markers, forget plot]
        ({2*cos(x)}, {2*sin(x)});

    % Straight lines (dashed, faint) — undeflected parallel paths
    \foreach \i in {1,...,9} {
        \addplot[spacecadet!25, line width=0.5pt, dashed, no markers, forget plot]
            table[x index=0, y index=1] {data/lec02_straight_\i.dat};
    }

    % Geodesics (solid) — colour gradient matching 3D panel
    \addplot[spacecadet!70!cgblue, line width=0.9pt, no markers, forget plot]
        table[x index=0, y index=1] {data/lec02_geodesic_1.dat};
    \addplot[spacecadet!50!cgblue, line width=0.9pt, no markers, forget plot]
        table[x index=0, y index=1] {data/lec02_geodesic_2.dat};
    \addplot[cgblue!80!munsell, line width=0.9pt, no markers, forget plot]
        table[x index=0, y index=1] {data/lec02_geodesic_3.dat};
    \addplot[cgblue, line width=0.9pt, no markers, forget plot]
        table[x index=0, y index=1] {data/lec02_geodesic_4.dat};
    \addplot[munsell, line width=1.1pt, no markers, forget plot]
        table[x index=0, y index=1] {data/lec02_geodesic_5.dat};
    \addplot[cgblue, line width=0.9pt, no markers, forget plot]
        table[x index=0, y index=1] {data/lec02_geodesic_6.dat};
    \addplot[cgblue!80!munsell, line width=0.9pt, no markers, forget plot]
        table[x index=0, y index=1] {data/lec02_geodesic_7.dat};
    \addplot[spacecadet!50!cgblue, line width=0.9pt, no markers, forget plot]
        table[x index=0, y index=1] {data/lec02_geodesic_8.dat};
    \addplot[spacecadet!70!cgblue, line width=0.9pt, no markers, forget plot]
        table[x index=0, y index=1] {data/lec02_geodesic_9.dat};

    % Bump center marker
    \addplot[only marks, mark=+, mark size=4pt, spacecadet!40, thick]
        coordinates {(0,0)};
\end{axis}
\end{tikzpicture}
\end{center}

  \caption{Initially parallel geodesics on a Gaussian-bump
    surface $z = e^{-(x^2+y^2)/2}$.  Nine geodesics all start from
    $x = -2.5$ moving in the $+x$ direction at equally spaced $y$
    offsets.  Top: 3D view showing the geodesics bending over the
    bump.  Bottom: top-down view comparing geodesics (solid) with
    undeflected straight paths (dashed).  The bump acts as a
    ``gravitational lens'': curvature causes the initially parallel
    paths to converge, cross, and then
    diverge.}\label{fig:geodesic-bump}
\end{figure}

% ──────────────────────────────────────────────────────────────
\subsection{The metric}\label{sec:metric-from-EEP}

The proper time~\eqref{eq:proper-time-flat} expressed in the lab
coordinates becomes
\begin{align}
  d\tau^2
    &= -\eta_{\alpha\beta}\,d\xi^\alpha\,d\xi^\beta
     = -\eta_{\alpha\beta}\,
        \pd{\xi^\alpha}{x^\mu}\,\pd{\xi^\beta}{x^\nu}\,
        dx^\mu\,dx^\nu \notag\\
    &= -g_{\mu\nu}\,dx^\mu\,dx^\nu\,,
    \label{eq:proper-time-general}
\end{align}
where
\begin{equation}\label{eq:metric-def}
  \eqbox{g_{\mu\nu}
    = \pd{\xi^\alpha}{x^\mu}\,\pd{\xi^\beta}{x^\nu}\,
      \eta_{\alpha\beta}}
\end{equation}
is the \textbf{metric}.  The metric collects all the information that
appeared when we changed to the new coordinate system; it is the
object that allows us to compute proper time (and hence distances and
clock rates) without having to transform back to freely falling
coordinates every time.

\begin{intuition}[What the metric encodes]
  The metric $g_{\mu\nu}$ tells you how to measure distances and
  time intervals in arbitrary coordinates.  In freely falling
  coordinates it reduces to the flat Minkowski metric
  $\eta_{\alpha\beta}$.  The deviation of $g_{\mu\nu}$ from
  $\eta_{\alpha\beta}$ encodes the gravitational field.  Ultimately,
  $g_{\mu\nu}$ is all one needs to describe the geometry of
  spacetime.
\end{intuition}

So far we have seen only one half of the story: the geometry of
spacetime tells freely falling particles how to move (via the
geodesic equation).  The other half---that matter tells spacetime
how to curve---requires Mach's principle and, eventually, Einstein's
field equations.

% ──────────────────────────────────────────────────────────────
\subsection{Mach's principle}

Mach's principle is the second key set of ideas---really a
\emph{circle} of ideas, less precise than EEP and harder to make
mathematically sharp, but profoundly influential on Einstein's
thinking.

\begin{keyresult}[Mach's principle (summary)]
  All matter in the universe should contribute to the local definition
  of ``non-accelerating'' and ``non-rotating.''
  In a universe devoid of matter, these concepts should have no meaning.
\end{keyresult}

\noindent This is a deep statement: the very notions of acceleration
and rotation are not absolute but are determined by the matter content
of the universe.  Without matter, there is nothing with respect to
which one could be ``accelerating'' or ``rotating.''

% ──────────────────────────────────────────────────────────────
\subsection{General relativity: the synthesis}

General relativity is the following statement:

\begin{keyresult}[General relativity]
  The observer-independent properties of spacetime are described
  by a \emph{spacetime metric} $g_{\mu\nu}$---a symmetric rank-2
  tensor field assigning four numbers to each spacetime
  event---which need not have the flat form $\eta_{\mu\nu}$ of special
  relativity.  \emph{Curvature} is that property of the metric that
  accounts for the physical effects of gravitation.
  Furthermore, curvature is determined by the distribution of
  stress-energy and momentum, and vice versa.
\end{keyresult}

\noindent The ``vice versa'' is crucial: stress-energy tells spacetime
how to curve, and curvature tells matter how to move.  This
self-referential, nonlinear interplay between geometry and matter is
what makes general relativity so remarkable.  The extraordinary---one
might say almost miraculous---thing is that this all holds together:
the theory does not collapse into inconsistency, but instead forms a
beautifully self-contained whole.

\begin{center}
\begin{tikzpicture}[
  concept/.style={ellipse, draw=cgblue, fill=cgblue!10, thick,
    minimum width=3.2cm, minimum height=1.6cm,
    font=\sf\small, text=spacecadet, align=center, inner sep=4pt},
  arr/.style={-{Stealth[length=6pt]}, very thick, spacecadet!70},
]
  \node[concept] (SE) at (-3,0) {Stress-energy\\(matter)};
  \node[concept] (C)  at (3,0)  {Curvature\\(geometry)};
  \draw[arr] ([yshift=4pt]SE.east) -- ([yshift=4pt]C.west)
    node[midway, above, font=\small\itshape]{tells spacetime how to curve};
  \draw[arr] ([yshift=-4pt]C.west) -- ([yshift=-4pt]SE.east)
    node[midway, below, font=\small\itshape]{tells matter how to move};
  % Matter at bottom
  \node[concept, draw=munsell, fill=munsell!8, minimum width=2.4cm,
    minimum height=1.2cm] (M) at (0,-2.5) {Matter};
  \draw[arr, munsell!60] (M) -- (SE);
  \draw[arr, munsell!60] (M) -- (C);
\end{tikzpicture}
\end{center}

\noindent Actually implementing the physical content of this
formulation---computing curvature, writing down Einstein's field
equations, extracting predictions---requires the mathematics of
spacetime manifolds that are not flat.  That is the work we commence
in the next lecture, beginning with the theory of differentiable
manifolds.
