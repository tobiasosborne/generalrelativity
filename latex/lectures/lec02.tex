%!TEX root = ../GeneralRelativity.tex
% ──────────────────────────────────────────────────────────────
%  Lecture 2 — The Equivalence Principle and Mach's Principle
% ──────────────────────────────────────────────────────────────

\section{The equivalence principle and Mach's principle}%
\label{sec:equivalence-principle}

\subsection{Einstein's equivalence principle (EEP)}

The Einstein equivalence principle consists of three parts:

\begin{enumerate}
  \item \textbf{Universality of free fall (UFF)}
    (also known as the \emph{weak equivalence principle}, WEP).\\
    Verified (not rejected) e.g.\ in torsion balance experiments
    (cf.~\S\ref{sec:UFF}).

  \item \textbf{Local Lorentz invariance (LLI).}\\
    In a freely falling reference frame, the laws of physics are those
    of special relativity (no ``preferred'' reference frame);
    i.e.\ the laws of physics do not depend on the orientation or
    velocity.

    \smallskip
    \noindent Best bounds from Michelson--Morley--type experiments:
    \begin{equation}\label{eq:LLI-bound}
      \frac{\Delta c}{c} < 5 \times 10^{-17}\,.
    \end{equation}

  \item \textbf{Local position invariance (LPI).}\\
    Special relativity holds independent of spacetime position
    (no ``preferred'' location).
\end{enumerate}

\begin{intuition}[The three pillars of EEP]
  UFF says \emph{all bodies fall the same way}.
  LLI says \emph{physics looks the same regardless of your velocity}.
  LPI says \emph{physics looks the same regardless of where you are}.
  Together, they force gravity to be a property of spacetime
  geometry rather than an ordinary force.
\end{intuition}

\subsubsection{LPI and gravitational redshift}

LPI is equivalent (exercise) to the \emph{universality of gravitational
redshift} (UGR): clock rates are universally affected by the
gravitational field.

\begin{exercise}\label{ex:grav-redshift}
  Show that for small velocities and weak gravitational fields,
  the fractional change in clock rate between two positions
  $\vb{x}_1$ and $\vb{x}_2$ is
  \begin{equation}\label{eq:grav-redshift}
    \eqbox{\frac{\Delta\nu}{\nu}
      = \frac{\Delta\phi}{c^2}}\,,
  \end{equation}
  where $\nu(\vb{x}_1)$, $\nu(\vb{x}_2)$ are the clock rates at
  positions $1,2$ and
  $\Delta\phi = \phi(\vb{x}_1) - \phi(\vb{x}_2)$.
\end{exercise}

\noindent\textbf{Estimate:} On Earth's surface, for a height difference
$\norm{\vb{x}_1 - \vb{x}_2} \sim 30\;\mathrm{cm}$, this effect is
$\Delta\phi/c^2 \sim 10^{-17}$---and has been measured!

\begin{intuition}[Gravity slows clocks]
  A clock at the bottom of a gravitational potential well runs slower
  than a clock at the top.  GPS satellites must correct for this:
  their clocks run faster (by $\sim 45\;\mu\mathrm{s}/\mathrm{day}$)
  than ground-based clocks.  Without this correction, GPS positions
  would drift by $\sim 10\;\mathrm{km}/\mathrm{day}$.
\end{intuition}

% ──────────────────────────────────────────────────────────────
\subsection{From EEP to geometry}

\textbf{Observation:} Using EEP, it is possible to ``derive'' that
gravity should be described \emph{geometrically}
(K.~Thorne, D.~Lee, \& A.~Lightman,
\emph{Phys.\ Rev.\ D}, \textbf{7}, 3563--3578, 1973).

\begin{keyresult}[The geometric consequence of EEP]
  All matter couples to the \emph{same geometry}---namely, that of
  spacetime.  A single metric suffices to describe all of these
  couplings simultaneously.
\end{keyresult}

% ──────────────────────────────────────────────────────────────
\subsection{Calculations with EEP}

\textbf{Statement:} No external, static, homogeneous gravitational
field can be detected in a freely falling elevator.

\smallskip
Suppose we have $N$ (non-relativistic) particles moving in a pair
force field $\vb{F}(\vb{x}_j - \vb{x}_k)$ and an external gravitational
field $\vb{g}(\vb{x}) = \vb{g}$ (constant), with $\vb{F}(\vb{0}) = \vb{0}$.

\noindent Equations of motion:
\begin{equation}\label{eq:eom-N-particles}
  m_j\,\frac{d^2\vb{x}_j}{dt^2}
    = m_j\,\vb{g}
      + \sum_{k=1}^{N} \vb{F}(\vb{x}_j - \vb{x}_k)\,.
\end{equation}

Make a (non-Galilean) change to an \emph{accelerating frame}:
\begin{equation}\label{eq:accelerating-frame}
  \vb{x}' = \vb{x} - \tfrac{1}{2}\,\vb{g}\,t^2\,,\qquad
  t' = t\,.
\end{equation}

Then the equations of motion become (exercise):
\begin{equation}\label{eq:eom-freefall}
  \eqbox{m_j\,\frac{d^2\vb{x}'_j}{dt'^2}
    = \sum_{k} \vb{F}(\vb{x}'_j - \vb{x}'_k)}\,.
\end{equation}
The gravitational field has been \emph{eliminated}---the freely falling
frame sees only the inter-particle forces.

% ──────────────────────────────────────────────────────────────
\subsection{Gravitational forces and freely falling coordinates}

According to EEP, for \emph{any} particle moving purely under the
influence of a gravitational field, there exists a freely falling
coordinate system $(\xi^0,\,\xi^1,\,\xi^2,\,\xi^3)$ such that
\begin{equation}\label{eq:force-free}
  \eqbox{\frac{d^2\xi^\alpha}{d\tau^2} = 0}
  \qquad (\equiv\;\text{force-free motion})\,,
\end{equation}
where $\tau$ is the \emph{proper time}, defined by
\begin{equation}\label{eq:proper-time-flat}
  d\tau^2 = -\eta_{\alpha\beta}\,d\xi^\alpha\,d\xi^\beta\,,
\end{equation}
or equivalently,
\begin{equation}\label{eq:4vel-normalisation}
  -1 = \eta_{\alpha\beta}\,
    \frac{d\xi^\alpha}{d\tau}\,\frac{d\xi^\beta}{d\tau}\,.
\end{equation}

Here $\eta_{\alpha\beta}$ is the \emph{Minkowski metric}:
\begin{equation}\label{eq:minkowski}
  \eta_{\alpha\beta}
  = \diag(-1,\,+1,\,+1,\,+1)\,.
\end{equation}

\noindent\textbf{Einstein's summation convention:}
\begin{equation}
  u_\alpha\,\xi^\alpha
  \;\equiv\; \sum_{\alpha=0}^{3} u_\alpha\,\xi^\alpha\,.
\end{equation}
Repeated indices (one ``upstairs,'' one ``downstairs'') are summed over.

% ──────────────────────────────────────────────────────────────
\subsection{The geodesic equation}

Suppose $x^\mu$ denotes some other set of coordinates (e.g.\ lab,
rotating, \ldots):

\begin{center}
\begin{tikzpicture}[scale=0.9]
  % Freely-falling frame (xi)
  \begin{scope}
    \draw[axisstyle, spacecadet] (0,0) -- (0,2.5) node[left]{$\xi^0$};
    \draw[axisstyle, spacecadet] (0,0) -- (2.2,0) node[below]{$\xi^1$};
    \node[font=\sf\small, text=cgblue] at (1.1,2.6)
      {freely falling};
    % Grid lines
    \foreach \i in {0.5,1.0,...,2.0} {
      \draw[thin, spacecadet!20] (0,\i) -- (2.0,\i);
      \draw[thin, spacecadet!20] (\i,0) -- (\i,2.0);
    }
  \end{scope}
  % Arrow between
  \draw[thick, -{Stealth[length=6pt]}, cgblue] (2.8,1.0) -- (4.2,1.0)
    node[midway, above, font=\small]{$\xi^\alpha(x^\mu)$};
  \draw[thick, -{Stealth[length=6pt]}, cgblue] (4.2,0.6) -- (2.8,0.6)
    node[midway, below, font=\small]{$x^\mu(\xi^\alpha)$};
  % Lab frame (x)
  \begin{scope}[shift={(5,0)}]
    \draw[axisstyle, spacecadet] (0,0) -- (0,2.5) node[left]{$x^0$};
    \draw[axisstyle, spacecadet] (0,0) -- (2.2,0) node[below]{$x^1$};
    \node[font=\sf\small, text=cgblue] at (1.1,2.6) {lab frame};
    % Curved grid lines
    \foreach \i in {0.5,1.0,...,2.0} {
      \draw[thin, spacecadet!20] (0,\i) to[out=5,in=175] (2.0,{\i+0.15});
      \draw[thin, spacecadet!20] (\i,0) to[out=85,in=-85] ({\i-0.1},2.0);
    }
  \end{scope}
\end{tikzpicture}
\end{center}

The freely falling coordinates are functions of the lab coordinates:
\[
  \xi^\alpha = \xi^\alpha(x^0,\,x^1,\,x^2,\,x^3)\,.
\]

The force-free equation~\eqref{eq:force-free} becomes
\begin{equation}
  0 = \frac{d^2\xi^\alpha}{d\tau^2}
    = \frac{d}{d\tau}\!\left(
        \pd{\xi^\alpha}{x^\mu}\,\frac{dx^\mu}{d\tau}
      \right)
    = \pd{\xi^\alpha}{x^\mu}\,\frac{d^2 x^\mu}{d\tau^2}
      + \frac{\partial^2\xi^\alpha}{\partial x^\mu\,\partial x^\nu}\,
        \frac{dx^\mu}{d\tau}\,\frac{dx^\nu}{d\tau}\,.
\end{equation}

Multiplying by $\pd{x^\lambda}{\xi^\alpha}$ and using
$\pd{\xi^\alpha}{x^\mu}\pd{x^\lambda}{\xi^\alpha} = \delta^\lambda{}_\mu$,
we obtain the \textbf{geodesic equation}:
\begin{equation}\label{eq:geodesic}
  \eqbox{
    \frac{d^2 x^\lambda}{d\tau^2}
    + \chris{\lambda}{\mu\nu}\,
      \frac{dx^\mu}{d\tau}\,\frac{dx^\nu}{d\tau}
    = 0}\,,
\end{equation}
where the \emph{affine connection} (Christoffel symbol) is
\begin{equation}\label{eq:christoffel-def}
  \chris{\lambda}{\mu\nu}
    = \pd{x^\lambda}{\xi^\alpha}\,
      \frac{\partial^2\xi^\alpha}{\partial x^\mu\,\partial x^\nu}\,.
\end{equation}

% ──────────────────────────────────────────────────────────────
\subsection{The metric}\label{sec:metric-from-EEP}

The proper time~\eqref{eq:proper-time-flat} expressed in the lab
coordinates becomes
\begin{align}
  d\tau^2
    &= -\eta_{\alpha\beta}\,d\xi^\alpha\,d\xi^\beta
     = -\eta_{\alpha\beta}\,
        \pd{\xi^\alpha}{x^\mu}\,\pd{\xi^\beta}{x^\nu}\,
        dx^\mu\,dx^\nu \notag\\
    &= -g_{\mu\nu}\,dx^\mu\,dx^\nu\,,
    \label{eq:proper-time-general}
\end{align}
where
\begin{equation}\label{eq:metric-def}
  \eqbox{g_{\mu\nu}
    = \pd{\xi^\alpha}{x^\mu}\,\pd{\xi^\beta}{x^\nu}\,
      \eta_{\alpha\beta}}
\end{equation}
is the \textbf{metric}.

\begin{intuition}[What the metric encodes]
  The metric $g_{\mu\nu}$ tells you how to measure distances and
  time intervals in arbitrary coordinates.  In freely falling
  coordinates it reduces to the flat Minkowski metric
  $\eta_{\alpha\beta}$.  The deviation of $g_{\mu\nu}$ from
  $\eta_{\alpha\beta}$ encodes the gravitational field.
\end{intuition}

% ──────────────────────────────────────────────────────────────
\subsection{Mach's principle}

Mach's principle is the second key set of ideas underlying
general relativity---less precise than EEP, but deeply influential.

\begin{keyresult}[Mach's principle (summary)]
  All matter in the universe should contribute to the local definition
  of ``non-accelerating'' and ``non-rotating.''
  In a universe devoid of matter, these concepts should have no meaning.
\end{keyresult}

% ──────────────────────────────────────────────────────────────
\subsection{General relativity: the synthesis}

General relativity is the following statement:

\begin{keyresult}[General relativity]
  The observer-independent properties of spacetime are described
  by a \emph{spacetime metric} $g_{\mu\nu}$, which need not have the
  flat form $\eta_{\mu\nu}$ of special relativity.  \emph{Curvature}
  accounts for the physical effects of gravitation.
  Furthermore, curvature is determined by the distribution of
  stress-energy and momentum (and vice versa).
\end{keyresult}

\begin{center}
\begin{tikzpicture}[
  concept/.style={ellipse, draw=cgblue, fill=cgblue!10, thick,
    minimum width=3.2cm, minimum height=1.6cm,
    font=\sf\small, text=spacecadet, align=center, inner sep=4pt},
  arr/.style={-{Stealth[length=6pt]}, very thick, spacecadet!70},
]
  \node[concept] (SE) at (-3,0) {Stress-energy\\(matter)};
  \node[concept] (C)  at (3,0)  {Curvature\\(geometry)};
  \draw[arr] ([yshift=4pt]SE.east) -- ([yshift=4pt]C.west)
    node[midway, above, font=\small\itshape]{tells spacetime how to curve};
  \draw[arr] ([yshift=-4pt]C.west) -- ([yshift=-4pt]SE.east)
    node[midway, below, font=\small\itshape]{tells matter how to move};
  % Matter at bottom
  \node[concept, draw=munsell, fill=munsell!8, minimum width=2.4cm,
    minimum height=1.2cm] (M) at (0,-2.5) {Matter};
  \draw[arr, munsell!60] (M) -- (SE);
  \draw[arr, munsell!60] (M) -- (C);
\end{tikzpicture}
\end{center}
