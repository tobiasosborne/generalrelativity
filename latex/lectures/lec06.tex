%!TEX root = ../GeneralRelativity.tex
% ──────────────────────────────────────────────────────────────
%  Lecture 6 — Tensors continued
% ──────────────────────────────────────────────────────────────

\section{Tensors continued}\label{sec:tensors-continued}

Let $p\in\M$ be a point in a manifold~$\M$.  We write $V_p$ for
the tangent space at~$p$.  In the previous lecture we constructed
tensor products of~$V_p$ purely algebraically, ignoring the
manifold structure entirely.  But a manifold provides \emph{extra
data}: not just one vector space but a whole field of vector
spaces, one at each point, linked by coordinate charts.  Our goal
in this lecture is to study how vectors $v\in V_p$, covectors
$\omega\in V_p^*$, and more general tensors transform under a
change of coordinates on~$\M$.

% ──────────────────────────────────────────────────────────────
\subsection{The cotangent space (dual of $V_p$)}\label{sec:cotangent}

\begin{definition}[Cotangent space]\label{def:cotangent}
  The \textbf{cotangent space} at~$p$ is the dual vector space
  \[
    V_p^* \;\equiv\; \Tps\,.
  \]
  Elements of $V_p^*$ are called \textbf{covectors}
  (or \emph{one-forms}).
\end{definition}

Recall that $V_p$ has a coordinate basis
$\{e_\mu\} = \{\partial/\partial x^\mu|_p\}$.
We formally define the \textbf{dual basis}
$\{e^\mu\} = \{\dd x^\mu\}$ by the requirement
\begin{equation}\label{eq:dual-basis-def}
  e^\mu(e_\nu) = \delta^\mu{}_\nu\,.
\end{equation}
In coordinate notation this reads
\begin{equation}\label{eq:dual-basis-coord}
  \eqbox{\dd x^\mu\!\left(\pd{}{x^\nu}\bigg|_p\right)
    = \delta^\mu{}_\nu}\,.
\end{equation}
Thus $\dd x^\mu$ is a \emph{linear function of tangent vectors},
fully determined by~\eqref{eq:dual-basis-coord}.

\begin{remark}
  At this stage, $\dd x^\mu$ is \textbf{just a symbol} for a
  particular covector---a linear function on tangent vectors
  defined by the Kronecker-delta
  condition~\eqref{eq:dual-basis-coord}.  It has nothing (yet) to
  do with derivatives or integration measures.  Later, $\dd x^\mu$
  will acquire an interpretation as a differential form and
  integration measure, but for now resist the temptation to read
  more into the notation than is given by the definition.
\end{remark}

\begin{intuition}[Covectors as measuring devices]
  A tangent vector~$v$ specifies a direction and magnitude
  at~$p$.  A covector~$\omega$ is a linear machine that
  ``measures'' tangent vectors: it eats a vector and returns a
  real number.  The dual basis element $\dd x^\mu$ measures the
  $\mu$-th component of a vector.
\end{intuition}

% ──────────────────────────────────────────────────────────────
\subsection{Transformation law for covectors}\label{sec:covector-transform}

Recall the \emph{contravariant} vector transformation law
(Lecture~4):
\begin{equation}\label{eq:vector-transform-recall}
  v'^{\mu} = \sum_{\nu} v^\nu\,
    \pd{x'^\mu}{x^\nu}\,.
\end{equation}

Now let $\omega\in V_p^*$.  In the coordinate basis,
\[
  \omega = \sum_\mu \omega_\mu\,\dd x^\mu\,.
\]
We ask: how are the components $\omega'_{\mu'}$ in the new
coordinate system $x'^\mu$ related to the old
components~$\omega_\mu$?

\smallskip
Apply $\omega$ to an arbitrary vector $v\in V_p$:
\begin{align}
  \omega(v)
    &= \omega\!\left(\sum_\mu v^\mu\,\pd{}{x^\mu}\bigg|_p\right)
     = \sum_\mu v^\mu\,
       \omega\!\left(\pd{}{x^\mu}\bigg|_p\right)
     = \sum_\mu \omega_\mu\, v^\mu\,.
     \label{eq:omega-v-old}
\end{align}
In the primed coordinates the same contraction gives
\begin{align}
  \omega(v)
    &= \sum_{\mu'} \omega'_{\mu'}\, v'^{\mu'}
     = \sum_{\mu',\mu} \omega'_{\mu'}\,
       \pd{x'^{\mu'}}{x^\mu}\, v^\mu\,.
     \label{eq:omega-v-new}
\end{align}
Since~\eqref{eq:omega-v-old} and~\eqref{eq:omega-v-new} must
agree for \emph{all} $v^\mu$, we read off
\begin{equation}\label{eq:covariant-transform}
  \eqbox{\omega_\mu
    = \sum_{\mu'} \omega'_{\mu'}\,
      \pd{x'^{\mu'}}{x^\mu}}
  \qquad\Longleftrightarrow\qquad
  \eqbox{\omega'_{\mu'}
    = \sum_{\mu} \pd{x^\mu}{x'^{\mu'}}\,\omega_\mu}\,.
\end{equation}
This is the \textbf{covariant vector transformation law}.

\begin{remark}
  Compare the two transformation laws:
  \begin{itemize}
    \item \textbf{Contravariant} (vectors): $v'^{\mu'}
      = (\partial x'^{\mu'}/\partial x^\mu)\,v^\mu$.
    \item \textbf{Covariant} (covectors): $\omega'_{\mu'}
      = (\partial x^\mu/\partial x'^{\mu'})\,\omega_\mu$.
  \end{itemize}
  The Jacobian matrices that appear are \emph{inverse transposes}
  of each other.
\end{remark}

% ──────────────────────────────────────────────────────────────
\subsection{General tensor transformation law}\label{sec:tensor-transform}

Let $T\in\ttype{k}{l}$ be a tensor of type~$(k,l)$.
In a coordinate basis, $T$ is expanded as
\begin{equation}\label{eq:tensor-expansion}
  T = \sum T^{\mu_1\cdots\mu_k}{}_{\nu_1\cdots\nu_l}\;
    e_{\mu_1}\tp\cdots\tp e_{\mu_k}
    \tp e^{\nu_1}\tp\cdots\tp e^{\nu_l}\,,
\end{equation}
where $e_\mu = \partial/\partial x^\mu|_p$ and
$e^\nu = \dd x^\nu$.

Under a change of coordinates
$x^\mu\to x'^{\mu'}$, the components transform as
\begin{equation}\label{eq:tensor-transform}
  \eqbox{T'^{\mu'_1\cdots\mu'_k}{}_{\nu'_1\cdots\nu'_l}
    = \sum
      T^{\mu_1\cdots\mu_k}{}_{\nu_1\cdots\nu_l}\,
      \pd{x'^{\mu'_1}}{x^{\mu_1}}\cdots
      \pd{x'^{\mu'_k}}{x^{\mu_k}}\,
      \pd{x^{\nu_1}}{x'^{\nu'_1}}\cdots
      \pd{x^{\nu_l}}{x'^{\nu'_l}}}
\end{equation}
This is the \textbf{tensor transformation law}.

\begin{keyresult}[Classical definition of a tensor]
  Classically, a \emph{tensor of type~$(k,l)$} at a point
  $p\in\M$ is a collection of numbers
  $T^{\mu_1\cdots\mu_k}{}_{\nu_1\cdots\nu_l}(p)$ that
  transforms under coordinate changes
  via~\eqref{eq:tensor-transform}.
\end{keyresult}

\begin{remark}
  Not every collection of numbers indexed by coordinate labels is a
  tensor.  One can associate $n^{k+l}$ numbers to each point
  $p\in\M$ that do \emph{not} obey~\eqref{eq:tensor-transform}
  under coordinate changes (the Christoffel
  symbols, which we shall meet in Lecture~7, are a prominent
  example).  Tensor fields are therefore quite special: they are
  the coordinate-indexed quantities whose transformation law is
  purely algebraic and involves only Jacobian factors.
\end{remark}

% ──────────────────────────────────────────────────────────────
\subsection{Smooth tensor fields}\label{sec:smooth-tensor-field}

\begin{definition}[Smooth tensor field]\label{def:smooth-tensor-field}
  A \textbf{smooth tensor field} $T$ of type~$(k,l)$ on~$\M$ is
  an assignment of a tensor
  $T|_p\in\ttype{k}{l}(V_p)$ to each point $p\in\M$, such that
  for all smooth covector fields
  $\omega^1,\dots,\omega^k$ and all smooth vector fields
  $v_1,\dots,v_l$, the function
  \[
    p\;\longmapsto\;
      T|_p(\omega^1|_p,\dots,\omega^k|_p;\;
            v_1|_p,\dots,v_l|_p)
  \]
  is $C^\infty$ on~$\M$.
\end{definition}

\begin{remark}
  A covector field $\omega$ is \emph{smooth} if and only if
  $\omega(v)$ is a $C^\infty$ function on~$\M$ for every smooth
  vector field~$v$.
\end{remark}

% ──────────────────────────────────────────────────────────────
\subsection{Examples from special relativity}\label{sec:SR-examples}

The preceding lectures have built up an abstract formalism for
multilinear objects on manifolds.  We now put it to work by
re-examining several familiar physical quantities---the
four-current, the energy-momentum tensor, and the metric---through
the lens of tensor analysis, all set in Minkowski
spacetime~$\M = \Rn{1,3}$.

% ──────────────────────────────────────────────────────────────
\subsubsection{Currents and charge densities}\label{sec:four-current}

Consider $N$ particles with worldlines $x_j(t)$ and charges
$q_j$, $j = 1,\dots,N$.

\begin{definition}[Charge density and current]\label{def:charge-current}
  The \textbf{charge density} is
  \begin{equation}\label{eq:charge-density}
    \varepsilon(\vb{x},t)
      = \sum_{j=1}^{N} q_j\,
        \delta^{(3)}\!\bigl(\vb{x} - \vb{x}_j(t)\bigr)\,,
  \end{equation}
  and the \textbf{current density} is
  \begin{equation}\label{eq:current-density}
    \vb{J}(\vb{x},t)
      = \sum_{j=1}^{N} q_j\,
        \delta^{(3)}\!\bigl(\vb{x} - \vb{x}_j(t)\bigr)\,
        \frac{d\vb{x}_j(t)}{dt}\,.
  \end{equation}
\end{definition}

\noindent Define a \textbf{four-current} $J^\mu$ by setting
\begin{equation}\label{eq:four-current}
  J^\mu = (\varepsilon,\;\vb{J})\,.
\end{equation}

\begin{exercise}\label{ex:four-current-vector}
  Argue that $J^\mu$ is a vector (i.e.\ an element of~$V_p$)
  field under Lorentz transformations
  $x'^{\nu} = \Lambda^\nu{}_\mu\, x^\mu$.
\end{exercise}

% ──────────────────────────────────────────────────────────────
\subsubsection{The energy-momentum tensor}\label{sec:energy-momentum}

Consider $N$ particles with energy-momentum four-vectors
$p_j^\mu$, $j = 1,\dots,N$.

The density of the $\mu$-th component of four-momentum is
\begin{equation}\label{eq:Tmu0}
  T^{\mu 0}(\vb{x},t)
    = \sum_{j=1}^{N} p_j^\mu(t)\,
      \delta^{(3)}\!\bigl(\vb{x} - \vb{x}_j(t)\bigr)\,.
\end{equation}
The corresponding current (the flux of $p^\mu$ in the
$x^k$-direction) is
\begin{equation}\label{eq:Tmuk}
  T^{\mu k}(\vb{x},t)
    = \sum_{j=1}^{N} p_j^\mu(t)\,
      \frac{dx_j^k(t)}{dt}\,
      \delta^{(3)}\!\bigl(\vb{x} - \vb{x}_j(t)\bigr)\,.
\end{equation}
Combining into a single formula
(with $x^0(t) = t$):
\begin{equation}\label{eq:Tmunu-dt}
  T^{\mu\nu}(x)
    = \sum_{j=1}^{N} p_j^\mu\,
      \frac{dx_j^\nu(t)}{dt}\,
      \delta^{(3)}\!\bigl(\vb{x} - \vb{x}_j(t)\bigr)\,.
\end{equation}

Since $p_j^\nu = E_j\, dx_j^\nu/dt$, this may be rewritten as
\begin{equation}\label{eq:Tmunu-symmetric}
  \eqbox{T^{\mu\nu}(x)
    = \sum_{j=1}^{N}
      \frac{p_j^\mu\, p_j^\nu}{E_j}\,
      \delta^{(3)}\!\bigl(\vb{x} - \vb{x}_j(t)\bigr)}\,.
\end{equation}
Hence $T^{\mu\nu}$ is \textbf{symmetric}:
$T^{\mu\nu} = T^{\nu\mu}$.

\smallskip
In covariant (proper-time) form:
\begin{equation}\label{eq:Tmunu-covariant}
  T^{\mu\nu}(x)
    = \sum_{j=1}^{N} \int\!\dd\tau\;
      p_j^\mu\,\frac{dx_j^\nu}{d\tau}\,
      \delta^{(4)}\!\bigl(x - x_j(\tau)\bigr)\,.
\end{equation}

\begin{exercise}\label{ex:Tmunu-tensor}
  Under Lorentz transformations
  $x'^{\nu} = \Lambda^\nu{}_\mu\, x^\mu$
  (with $\Lambda^{\mu'}{}_\mu = \partial x'^{\mu'}/\partial x^\mu$),
  show that
  \begin{equation}\label{eq:Tmunu-transform}
    T'^{\mu'\nu'}
      = \Lambda^{\mu'}{}_\mu\,
        \Lambda^{\nu'}{}_\nu\,
        T^{\mu\nu}\,.
  \end{equation}
  Conclude that $T^{\mu\nu}$ is a tensor of type~$(2,0)$.
  \emph{Hint:} the covariant
  form~\eqref{eq:Tmunu-covariant} makes the argument more
  transparent, since the four-dimensional delta function and
  proper-time integral are manifestly Lorentz-invariant.
\end{exercise}

The energy-momentum tensor plays an extraordinarily important role
in general relativity: it is the object that encodes the complete
stress-energy content of matter and radiation, and will appear as
the source term on the right-hand side of Einstein's field
equations.

% ──────────────────────────────────────────────────────────────
\subsubsection{The metric tensor}\label{sec:metric-tensor}

\begin{definition}[Metric tensor]\label{def:metric-tensor}
  A \textbf{metric tensor} $\metric$ on a manifold~$\M$ is a
  smooth tensor field of type~$(0,2)$ that is
  \begin{enumerate}
    \item \textbf{symmetric:}
      $\metric(v_1,v_2) = \metric(v_2,v_1)$
      for all $v_1,v_2\in V_p$, and
    \item \textbf{nondegenerate:}
      $\metric(v,w) = 0$ for all $w\in V_p$ implies $v = 0$.
  \end{enumerate}
\end{definition}

\begin{intuition}[Why we need a metric]
  A manifold by itself has no notion of distance or angle.
  A metric is the extra datum that supplies an
  \emph{infinitesimal length}:
  an infinitesimal displacement is modelled by a tangent vector,
  and the ``infinitesimal squared distance'' is a quadratic
  function of that tangent vector---precisely what
  $\metric(v,v)$ provides.
\end{intuition}

In a coordinate basis
$\{\partial/\partial x^\mu|_p\}$, expand the metric as
\begin{equation}\label{eq:metric-components}
  \metric = \sum_{\mu,\nu} g_{\mu\nu}\,
    \dd x^\mu\tp\dd x^\nu\,.
\end{equation}
This is often written (omitting the tensor product sign) as
the \textbf{line element}
\begin{equation}\label{eq:line-element}
  \eqbox{\dd s^2 = \sum_{\mu,\nu} g_{\mu\nu}\,
    \dd x^\mu\,\dd x^\nu}\,.
\end{equation}

The metric defines an \textbf{inner product} on~$V_p$ for each
$p\in\M$:
\begin{align}
  (v,w)_p
    &= \sum_{\mu,\nu} g_{\mu\nu}\,
       (\dd x^\mu\tp\dd x^\nu)(v,w)
     = \sum_{\mu,\nu} g_{\mu\nu}\,
       \dd x^\mu(v)\,\dd x^\nu(w) \notag\\
    &= \sum_{\mu,\nu} g_{\mu\nu}\, v^\mu\, w^\nu\,.
     \label{eq:inner-product}
\end{align}

\subsubsection{Signature}\label{sec:signature}

By the Gram--Schmidt procedure, there exists an orthonormal
basis $\{v_{\hat\mu}\}$ for~$V_p$ such that
\begin{equation}\label{eq:orthonormal-basis}
  \metric(v_{\hat\mu},\, v_{\hat\nu})
    = s_\mu\,\delta_{\mu\nu}\,,
  \qquad s_\mu \in \{+1,\,-1\}\,.
\end{equation}

\begin{exercise}\label{ex:gram-schmidt}
  Prove that an orthonormal basis satisfying
  \eqref{eq:orthonormal-basis} exists.
\end{exercise}

\noindent The number of $+1$'s and $-1$'s is independent of
the choice of orthonormal basis.  This invariant is the
\textbf{signature} of~$\metric$.

\begin{definition}[Riemannian and Lorentzian metrics]%
  \label{def:riemannian-lorentzian}
  \leavevmode
  \begin{itemize}
    \item A metric $\metric$ with $s_\mu = +1$ for all~$\mu$ is
      called \textbf{Riemannian} (positive definite).
    \item The metric of spacetime has signature
      $(-1,+1,+1,+1)$---this is called a \textbf{Lorentzian}
      metric.
  \end{itemize}
\end{definition}

% ──────────────────────────────────────────────────────────────
\subsection{The metric as a map $V_p\to V_p^*$}\label{sec:metric-isomorphism}

A metric tensor $\metric$ can be viewed simultaneously as:
\begin{enumerate}
  \item a $(0,2)$ tensor,
  \item a bilinear map $V_p\times V_p\to\R$, and
  \item a linear map $V_p\to V_p^*$.
\end{enumerate}

The third interpretation is induced by the map
\begin{equation}\label{eq:metric-map}
  v \;\longmapsto\; \metric(\cdot\,,v) \;=:\; \omega\,.
\end{equation}
Here $\metric(\cdot\,,v)\colon V_p\to\R$, so indeed
$\metric(\cdot\,,v)\in V_p^*$.

\begin{theorem}\label{thm:metric-iso}
  If $\metric$ is nondegenerate, then the
  map~\eqref{eq:metric-map} is a linear isomorphism
  $V_p\xrightarrow{\;\sim\;} V_p^*$.
\end{theorem}

\begin{proof}
  The map $v\mapsto\metric(\cdot\,,v)$ is linear by bilinearity
  of~$\metric$.  Suppose $\metric(\cdot\,,v) = 0$, i.e.\
  $\metric(w,v) = 0$ for all $w\in V_p$.  By nondegeneracy,
  $v = 0$.  Hence the kernel is trivial and the map is injective.
  Since $\dim V_p = \dim V_p^*$, it is also surjective.
\end{proof}

\begin{keyresult}[Raising and lowering indices]
  The metric provides a canonical, basis-independent
  correspondence between vectors and covectors.  In components:
  \[
    v_\mu = \sum_\nu g_{\mu\nu}\, v^\nu
    \qquad(\text{lowering})\,,
    \qquad
    \omega^\mu = \sum_\nu g^{\mu\nu}\,\omega_\nu
    \qquad(\text{raising})\,,
  \]
  where $g^{\mu\nu}$ is the matrix inverse of $g_{\mu\nu}$.
\end{keyresult}

\begin{remark}
  This canonical isomorphism is so convenient that many authors
  identify vectors with covectors from the outset, never
  introducing the cotangent space as a separate object.  We prefer
  to keep the distinction explicit: the metric is extra data that
  may not always be present (e.g.\ in symplectic geometry or
  contact geometry), and understanding the difference between $V_p$
  and~$V_p^*$ clarifies which structures depend on that extra data
  and which do not.
\end{remark}

% ──────────────────────────────────────────────────────────────
\subsection{Abstract index notation}\label{sec:abstract-index}

Suppose $T\in\ttype{k}{l}$.  Think of $T$ as a multilinear map
\[
  T\colon \underbrace{V_p^*\tp\cdots\tp V_p^*}_{k}
    \;\tp\;
    \underbrace{V_p\tp\cdots\tp V_p}_{l}
    \;\longrightarrow\; \R\,.
\]
One can specify $T$ by its components
$T^{\mu_1\cdots\mu_k}{}_{\nu_1\cdots\nu_l}$ in some basis.
However, it is often sufficient to know merely \emph{which
arguments of~$T$ take vectors and which take covectors},
without committing to a particular basis.

\begin{definition}[Abstract index notation]\label{def:abstract-index}
  In the \textbf{abstract index notation}, each argument of a
  tensor is labelled by a lower-case Latin letter.
  \begin{itemize}
    \item A \textbf{superscript} (upper) index labels a
      \emph{contravariant} slot (takes a covector).
    \item A \textbf{subscript} (lower) index labels a
      \emph{covariant} slot (takes a vector).
  \end{itemize}
\end{definition}

\begin{example}\label{ex:abstract-index}
  \leavevmode
  \begin{itemize}
    \item A vector is written $v^a$---one contravariant slot.
    \item A covector is written $\omega_a$---one covariant slot.
    \item $T^{ab}{}_{cd}$ denotes a tensor of type~$(2,2)$.
  \end{itemize}
  Here the lower-case Latin letters $a,b,c,d,\dots$ label
  \emph{arguments and their type}, not components with respect
  to a basis (for which we use Greek
  indices~$\mu,\nu,\rho,\dots$).
\end{example}

\begin{remark}
  The abstract index notation (due to R.~Penrose) is a
  compromise between the coordinate-free and
  component-based approaches.  It carries the structural
  information of index placement (contravariant vs.\ covariant)
  without fixing a basis, yet retains the computational
  convenience of the index calculus.  In practice, any physical
  calculation eventually requires choosing a basis---there is no
  such thing as doing physics entirely without coordinates.  But
  one can, and should, carry out as many manipulations as possible
  in a basis-free manner; the abstract index notation is designed
  to facilitate exactly this.
\end{remark}
