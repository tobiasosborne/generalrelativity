%!TEX root = ../GeneralRelativity.tex
% ──────────────────────────────────────────────────────────────
%  Lecture 18 — FLRW solutions and the Schwarzschild ansatz
% ──────────────────────────────────────────────────────────────

\subsection{Exact FLRW solutions}%
\label{sec:flrw-solutions}

We now solve the Friedmann equations analytically for various matter
contents and spatial curvatures.

% ──────────────────────────────────────────────────────────────
\subsubsection{Conservation laws from the fluid equation}%
\label{sec:flrw-conservation}

The fluid equation~\eqref{eq:fluid-equation},
$\dot\rho = -3(\dot a/a)(\rho + p)$, admits immediate first
integrals for the two most important equations of state.

\medskip\noindent
\textbf{Dust} ($p = 0$).\;
The fluid equation becomes $\dot\rho + 3\rho\,\dot a/a = 0$, which
is a total derivative in disguise:
\begin{equation}\label{eq:dust-conservation}
  \frac{d}{d\tau}\bigl(\rho\, a^3\bigr) = 0
  \qquad\Longrightarrow\qquad
  \eqbox{\rho\, a^3 = \text{const.}}
\end{equation}
This is simply the conservation of rest mass: the energy density
scales inversely with the volume $\propto a^3$, which is exactly
what one expects for non-relativistic matter that is neither created
nor destroyed.

\medskip\noindent
\textbf{Radiation} ($p = \rho/3$).\;
The fluid equation gives $\dot\rho + 4\rho\,\dot a/a = 0$, hence:
\begin{equation}\label{eq:radiation-conservation}
  \eqbox{\rho\, a^4 = \text{const.}}
\end{equation}

\begin{intuition}[Why $a^4$ for radiation?]
  The photon number density scales as $a^{-3}$ (dilution by volume
  expansion), just as for dust.  But each photon also loses energy
  as the universe expands: its wavelength is stretched by the scale
  factor, contributing an additional factor of $a^{-1}$
  (\textbf{cosmological redshift}).  Together:
  $\rho_{\text{rad}} \propto a^{-3} \cdot a^{-1} = a^{-4}$.
  Consequently, \emph{radiation dominated the energy budget of the
  early universe} (when $a$ was small), even though dust dominates
  today.
\end{intuition}

% ──────────────────────────────────────────────────────────────
\subsubsection{Late-time behaviour}%
\label{sec:flrw-late-time}

The qualitative late-time behaviour depends on the spatial
curvature~$k$.  Assuming $\rho > 0$ and $p \geq 0$ throughout:

\begin{itemize}
  \item \textbf{$k = 0$ or $k = -1$} (flat or hyperbolic):
    Equation~\eqref{eq:friedmann-I} with $\Lambda = 0$ gives
    $\dot a^2 = 8\pi\rho\, a^2/3 - k$.  Since $\rho$ decreases
    with $a$ and the right-hand side never reaches zero (for $k = 0$
    the right-hand side is positive whenever $\rho > 0$; for
    $k = -1$ the $-k = +1$ term keeps it positive), the expansion
    continues forever: \textbf{eternal expansion}.  For $k = 0$,
    $\dot a \to 0$ as $\tau \to \infty$; for $k = -1$,
    $\dot a \to 1$.
  \item \textbf{$k = +1$} (spherical): the curvature term
    $-k/a^2 = -1/a^2$ eventually dominates, driving $\dot a$ to
    zero.  The universe reaches a maximum size and then
    \emph{recollapses}, ending in a \textbf{Big Crunch}---the
    time-reverse of the Big Bang.
\end{itemize}

% ──────────────────────────────────────────────────────────────
\subsubsection{Exact solutions}%
\label{sec:flrw-exact}

Using the conservation laws, Friedmann equation~\eqref{eq:friedmann-I}
(with $\Lambda = 0$) reduces to a single ODE for~$a(\tau)$.
For \textbf{dust} ($p = 0$), define
$C = 8\pi\rho\, a^3/3 = \text{const}$:
\begin{equation}\label{eq:friedmann-dust-ode}
  \dot a^2 = \frac{C}{a} - k\,.
\end{equation}
For \textbf{radiation} ($p = \rho/3$), define
$C' = 8\pi\rho\, a^4/3 = \text{const}$:
\begin{equation}\label{eq:friedmann-rad-ode}
  \dot a^2 = \frac{C'}{a^2} - k\,.
\end{equation}
These are solved by separation of variables (the integrals are
mildly irritating but standard).  The results are collected in the
following table.

\begin{center}
\renewcommand{\arraystretch}{1.8}
\begin{tabular}{c|cc}
  \toprule
  & \textbf{Dust} ($p = 0$) & \textbf{Radiation} ($p = \rho/3$)\\
  \midrule
  $k = +1$
    & $\begin{aligned}
        a &= \tfrac{1}{2}C(1 - \cos\eta)\\
        \tau &= \tfrac{1}{2}C(\eta - \sin\eta)
      \end{aligned}$
    & $a = \sqrt{C'}\bigl(1 - (1 - \tau/\sqrt{C'})^2\bigr)^{1/2}$
    \\[6pt]
  $k = 0$
    & $a = \bigl(\tfrac{9C}{4}\bigr)^{1/3}\, \tau^{2/3}$
    & $a = (4C')^{1/4}\, \tau^{1/2}$
    \\[6pt]
  $k = -1$
    & $\begin{aligned}
        a &= \tfrac{1}{2}C(\cosh\eta - 1)\\
        \tau &= \tfrac{1}{2}C(\sinh\eta - \eta)
      \end{aligned}$
    & $a = \sqrt{C'}\bigl((1 + \tau/\sqrt{C'})^2 - 1\bigr)^{1/2}$
    \\
  \bottomrule
\end{tabular}
\end{center}

\noindent
Here $\eta$ is a parameter (conformal time for the dust solutions).
The $k = +1$ dust solution is given in parametric form---the
cycloid---and describes a universe that expands from a Big Bang,
reaches maximum size at $\eta = \pi$, and recollapses to a Big
Crunch at $\eta = 2\pi$.

\begin{exercise}\label{ex:k0-dust}
  Derive the $k = 0$ dust solution $a \propto \tau^{2/3}$ directly
  from~\eqref{eq:friedmann-dust-ode} with $k = 0$.
\end{exercise}

\begin{intuition}[The Friedmann--Lema\^itre--Robertson--Walker cosmologies]
  All solutions in the table above share the same qualitative
  feature: they begin with $a = 0$ (the Big Bang).  For $k = 0$ and
  $k = -1$, they expand forever; for $k = +1$, the expansion halts
  and reverses.  Despite the extreme simplicity of the model---a
  homogeneous isotropic universe filled with a single type of
  matter---these solutions capture the essential large-scale dynamics
  of our universe remarkably well.  The FLRW cosmologies serve as
  the starting point for all of modern observational cosmology;
  perturbations around them describe the formation of galaxies and
  the temperature fluctuations of the CMB.  This concludes our
  discussion of cosmology; we now turn to the other celebrated exact
  solution of Einstein's equations.
\end{intuition}

% ══════════════════════════════════════════════════════════════
%  Part 2: Schwarzschild — setting up the ansatz
% ══════════════════════════════════════════════════════════════

\section{The Schwarzschild solution}%
\label{sec:schwarzschild}

Roughly speaking, there are two major families of well-known exact
solutions to Einstein's field equations: the FLRW cosmologies
(describing the large-scale evolution of the universe) and the
\textbf{Schwarzschild solution} (describing the gravitational
field outside a spherically symmetric, non-rotating body).  We now
turn to the latter.

% ──────────────────────────────────────────────────────────────
\subsection{Static and stationary spacetimes}%
\label{sec:static-stationary}

We seek solutions that ``don't change in time.''  There are two
levels of precision for this intuition.

\begin{definition}[Stationary spacetime]\label{def:stationary}
  A spacetime $(\M, g_{ab})$ is \textbf{stationary} if there exists
  a one-parameter group $\phi_t$ of isometries whose orbits are
  timelike curves (``time translations'').

  Equivalently: there exists a timelike Killing vector field~$\xi^a$
  (i.e.\ $\covd_{(a}\xi_{b)} = 0$ with $\xi^a \xi_a < 0$).
\end{definition}

\begin{definition}[Static spacetime]\label{def:static}
  A spacetime is \textbf{static} if it is stationary \emph{and}
  there exists a spacelike hypersurface~$\Sigma$ that is orthogonal
  to the orbits of the isometry group.

  By Frobenius' theorem, this is equivalent to requiring that the
  timelike Killing vector field additionally satisfies
  \begin{equation}\label{eq:frobenius}
    \xi_{[a}\, \covd_b\, \xi_{c]} = 0\,.
  \end{equation}
\end{definition}

\begin{remark}
  A rotating body (e.g.\ a Kerr black hole) generates a
  \emph{stationary but not static} spacetime: it has a timelike
  Killing vector (time-translation symmetry), but the rotation
  introduces off-diagonal $dt\,d\phi$ terms in the metric that
  prevent the existence of a hypersurface orthogonal to the time
  direction everywhere.
\end{remark}

In adapted coordinates, a static metric takes the form
\begin{equation}\label{eq:static-metric}
  ds^2 = -V^2(x^1, x^2, x^3)\, dt^2
    + h_{\mu\nu}(x^1, x^2, x^3)\, dx^\mu\, dx^\nu\,,
\end{equation}
where $V^2 = -\xi^a \xi_a$ is the squared norm of the Killing
vector.  Crucially, there are \emph{no} cross-terms $dt\, dx^\mu$
(this is the consequence of staticity, not merely stationarity), and
the components depend only on the spatial coordinates, not on~$t$.

% ──────────────────────────────────────────────────────────────
\subsection{Spherical symmetry}%
\label{sec:spherical-symmetry}

\begin{definition}[Spherically symmetric spacetime]%
  \label{def:spherical-symmetry}
  A spacetime is \textbf{spherically symmetric} if its isometry
  group contains a subgroup isomorphic to $SO(3)$ whose orbits are
  two-spheres~$S^2$.
\end{definition}

The full spacetime metric~$g_{ab}$ induces a metric $\hat h_{ab}$
on each orbit sphere.  By $SO(3)$ invariance, the induced metric
must be a multiple of the standard round metric on~$S^2$---there is
simply no other $SO(3)$-invariant metric on~$S^2$.
The proportionality constant is characterised by the area~$A$ of
the orbit sphere; we define the \textbf{areal radius}
\begin{equation}\label{eq:areal-radius}
  r = \sqrt{A/4\pi}\,,
\end{equation}
so that the induced metric on each orbit sphere is
$r^2(d\theta^2 + \sin^2\!\theta\, d\phi^2)$.

For a spacetime that is both \textbf{static} and
\textbf{spherically symmetric}, the timelike Killing
field~$\xi^a$ is unique (up to normalisation).  One can show that
$\xi^a$ must be orthogonal to the orbit spheres (project $\xi^a$
onto $S^2$; the projection must be $SO(3)$-invariant, but the only
invariant vector field on $S^2$ is the zero field).  The orbit
spheres therefore lie within the spatial hypersurfaces~$\Sigma_t$.

% ──────────────────────────────────────────────────────────────
\subsection{The Schwarzschild ansatz}%
\label{sec:schwarzschild-ansatz}

Combining staticity and spherical symmetry, we choose coordinates
as follows: pick a fiducial orbit sphere in~$\Sigma$, introduce
spherical coordinates $(\theta, \phi)$ on it, and propagate them to
neighbouring spheres along geodesics orthogonal to the fiducial
sphere (parameterised by the areal radius~$r$).  Using $t$ for the
Killing parameter, the full coordinate system is $(t, r, \theta,
\phi)$.

In these coordinates, the most general static spherically symmetric
metric takes the form:
\begin{equation}\label{eq:schwarzschild-ansatz}
  \eqbox{ds^2 = -f(r)\, dt^2 + h(r)\, dr^2
    + r^2\bigl(d\theta^2 + \sin^2\!\theta\, d\phi^2\bigr)}
\end{equation}
where $f(r)$ and $h(r)$ are two unknown functions of the radial
coordinate alone.  All the symmetry content---staticity and
spherical symmetry---has been used to arrive at this ansatz.

\begin{intuition}[From ten unknowns to two]
  The general metric in four dimensions has ten independent
  components.  Staticity eliminated the four space--time cross-terms
  and the dependence on~$t$, leaving seven spatial functions.
  Spherical symmetry then fixed the angular part to be the standard
  round metric on~$S^2$ (multiplied by~$r^2$), reducing the
  unknowns to just \emph{two functions of a single variable}:
  $f(r)$ and $h(r)$.  In the next lecture, we will substitute this
  ansatz into the vacuum Einstein equations and solve for~$f$
  and~$h$, obtaining the celebrated \textbf{Schwarzschild metric}.
\end{intuition}
