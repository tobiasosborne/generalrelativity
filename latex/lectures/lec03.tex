%!TEX root = ../GeneralRelativity.tex
% ──────────────────────────────────────────────────────────────
%  Lecture 3 — Manifolds
% ──────────────────────────────────────────────────────────────

\section{Manifolds}\label{sec:manifolds}

We want to describe curved continua \emph{intrinsically}, without
reference to an ambient space $\Rn{m}$ in which they might be
embedded.  The appropriate mathematical notion is that of a
\emph{manifold}.

\begin{intuition}[Why manifolds?]
  In general relativity, spacetime is a manifold~$\M$.
  Although we may picture $\M$ as sitting inside some higher-dimensional
  $\Rn{m}$, all physically meaningful statements should be intrinsic:
  they must not depend on the embedding.
\end{intuition}

% ──────────────────────────────────────────────────────────────
\subsection{Notation and prerequisites}

We work in $n$ dimensions throughout.

\begin{itemize}
  \item $\Rn{n} = \{(x^1,\dots,x^n) \mid x^j \in \R\}$.
  \item For $\vb{x} = (x^1,\dots,x^n)$, the Euclidean norm is
    \[
      \norm{\vb{x}-\vb{y}}
        = \Bigl(\sum_{j=1}^{n}(x^j - y^j)^2\Bigr)^{1/2}.
    \]
  \item The \emph{open ball} of radius~$r$ about $\vb{y}$ is
    \[
      B_r(\vb{y})
        = \{\vb{x}\in\Rn{n} \mid \norm{\vb{x}-\vb{y}} < r\}.
    \]
  \item A subset $U\subseteq\Rn{n}$ is \emph{open} if
    for every $\vb{x}\in U$ there exists $\varepsilon > 0$ such that
    $B_\varepsilon(\vb{x})\subset U$.
  \item $C^n$ denotes the set of $n$-times differentiable functions
    (from $\Rn{n}$ to $\R$, or between open subsets thereof).
  \item $C^0 = {}$continuous functions,\quad
    $C^\infty = {}$smooth (infinitely differentiable) functions.
\end{itemize}

% ──────────────────────────────────────────────────────────────
\subsection{Definition of a manifold}\label{sec:manifold-def}

\begin{definition}[Manifold]\label{def:manifold}
  An $n$-dimensional, $C^\infty$, real \textbf{manifold}~$\M$ is a set
  together with a collection $\{O_\alpha\}$ of subsets satisfying:
  \begin{enumerate}
    \item[\textup{(0)}] $\M$ is a topological space (Hausdorff and
      paracompact).\footnote{The Hausdorff and paracompact conditions
      exclude pathological examples; they ensure, e.g., the existence
      of partitions of unity.}
    \item[\textup{(1)}] The subsets $\{O_\alpha\}$ \textbf{cover} $\M$,
      i.e.\ $\bigcup_\alpha O_\alpha = \M$.
      Equivalently: for every $p\in\M$ there exists at least one
      $\alpha$ with $p\in O_\alpha$.
    \item[\textup{(2)}] For each $\alpha$ there is a \textbf{one-to-one
      and onto} map
      \[
        \psi_\alpha \colon O_\alpha \to U_\alpha\,,
        \qquad U_\alpha \subset \Rn{n}\;\text{(open)}\,,
      \]

      \begin{center}
      \begin{tikzpicture}[scale=0.85]
        % Manifold blob
        \draw[thick, spacecadet, rounded corners=12pt]
          (0,0.3) to[out=30,in=150] (3,0.5)
          to[out=-30,in=30] (3.2,-0.8)
          to[out=210,in=-10] (0,-0.6) -- cycle;
        \node[font=\sf\small, text=cgblue] at (1.6,0.8) {$\M$};
        \draw[thick, cgblue, fill=cgblue!10, rounded corners=5pt]
          (0.5,-0.2) to[out=40,in=150] (2.2,0.1)
          to[out=-30,in=50] (2.0,-0.5)
          to[out=210,in=-20] (0.4,-0.5) -- cycle;
        \node[font=\small] at (1.3,-0.15) {$O_\alpha$};
        % Arrow
        \draw[thick, -{Stealth[length=6pt]}, cgblue]
          (3.6,-0.1) -- (5.0,-0.1)
          node[midway, above, font=\small]{$\psi_\alpha$};
        % R^n region
        \begin{scope}[shift={(5.4,0)}]
          \draw[axisstyle, spacecadet] (0,-1) -- (0,1) node[left]{};
          \draw[axisstyle, spacecadet] (-0.3,0) -- (2.5,0) node[below]{};
          \node[font=\sf\small, text=spacecadet] at (2.4,1) {$\Rn{n}$};
          \draw[thick, munsell, fill=munsell!10, rounded corners=4pt]
            (0.3,-0.6) to[out=30,in=200] (1.8,0.5)
            to[out=-30,in=80] (1.9,-0.4)
            to[out=210,in=-10] (0.3,-0.6);
          \node[font=\small] at (1.1,-0.05) {$U_\alpha$};
        \end{scope}
      \end{tikzpicture}
      \end{center}

    \item[\textup{(3)}] If any two sets $O_\alpha$ and $O_\beta$
      \textbf{overlap}, i.e.\ $O_\alpha \cap O_\beta \neq \varnothing$,
      then the \emph{transition map}
      \[
        \psi_\beta \circ \psi_\alpha^{-1}
          \colon \psi_\alpha[O_\alpha\cap O_\beta]
          \;\to\; \psi_\beta[O_\alpha\cap O_\beta]
      \]
      (which is a map between open subsets of $\Rn{n}$)
      is $C^\infty$.
  \end{enumerate}
\end{definition}

\begin{center}
\begin{tikzpicture}[scale=0.85]
  % Manifold
  \draw[thick, spacecadet, rounded corners=14pt]
    (2,-0.3) to[out=60,in=170] (5.5,0.8)
    to[out=-10,in=80] (6.5,-0.5)
    to[out=260,in=-20] (2.5,-1.5)
    to[out=160,in=280] cycle;
  \node[font=\sf\small, text=cgblue] at (6.2,0.8) {$\M$};

  % O_alpha
  \draw[thick, cgblue, fill=cgblue!8, rounded corners=6pt]
    (2.5,0.1) to[out=40,in=160] (4.5,0.3)
    to[out=-20,in=60] (4.3,-0.7)
    to[out=200,in=-30] (2.3,-0.5) -- cycle;
  \node[font=\small, text=cgblue] at (2.7,0.3) {$O_\alpha$};

  % O_beta
  \draw[thick, munsell, fill=munsell!8, rounded corners=6pt]
    (3.8,0.5) to[out=20,in=140] (5.8,0.2)
    to[out=-40,in=30] (5.6,-0.8)
    to[out=210,in=-10] (3.6,-0.4) -- cycle;
  \node[font=\small, text=munsell] at (5.6,0.4) {$O_\beta$};

  % Point p in overlap
  \node[circle, fill=banana, inner sep=1.5pt, label={above:\small $p$}]
    at (4.2,-0.1) {};

  % psi_alpha arrow
  \draw[thick, -{Stealth[length=6pt]}, cgblue]
    (2.2,-1.0) -- (0.8,-2.4)
    node[midway, left, font=\small]{$\psi_\alpha$};

  % psi_beta arrow
  \draw[thick, -{Stealth[length=6pt]}, munsell]
    (5.5,-1.0) -- (6.8,-2.4)
    node[midway, right, font=\small]{$\psi_\beta$};

  % U_alpha in R^n (left)
  \begin{scope}[shift={(0,-3.2)}]
    \draw[axisstyle, spacecadet] (0,-0.8) -- (0,0.8);
    \draw[axisstyle, spacecadet] (-0.3,0) -- (2.6,0);
    \node[font=\sf\small, text=spacecadet] at (-0.1,1.0) {$\Rn{n}$};
    \draw[thick, cgblue, fill=cgblue!10, rounded corners=4pt]
      (0.3,-0.5) to[out=30,in=200] (1.8,0.4)
      to[out=-20,in=80] (2.0,-0.3)
      to[out=210,in=-10] (0.3,-0.5);
    \node[font=\small, text=cgblue] at (1.1,0.0) {$U_\alpha$};
    \node[circle, fill=banana, inner sep=1.2pt] (pa) at (1.5,0.1) {};
    \node[font=\scriptsize, below left] at (pa) {$\psi_\alpha(p)$};
  \end{scope}

  % U_beta in R^n (right)
  \begin{scope}[shift={(5.5,-3.2)}]
    \draw[axisstyle, spacecadet] (0,-0.8) -- (0,0.8);
    \draw[axisstyle, spacecadet] (-0.3,0) -- (2.6,0);
    \node[font=\sf\small, text=spacecadet] at (2.5,1.0) {$\Rn{n}$};
    \draw[thick, munsell, fill=munsell!10, rounded corners=4pt]
      (0.4,-0.5) to[out=30,in=200] (1.9,0.4)
      to[out=-20,in=80] (2.1,-0.3)
      to[out=210,in=-10] (0.4,-0.5);
    \node[font=\small, text=munsell] at (1.2,0.0) {$U_\beta$};
    \node[circle, fill=banana, inner sep=1.2pt] (pb) at (0.8,0.1) {};
    \node[font=\scriptsize, below right] at (pb) {$\psi_\beta(p)$};
  \end{scope}

  % Transition map arrow
  \draw[thick, -{Stealth[length=6pt]}, spacecadet!70]
    (2.2,-3.2) to[out=-10,in=190] (5.7,-3.2)
    node[midway, below, font=\small]{$\psi_\beta\circ\psi_\alpha^{-1}$};
\end{tikzpicture}
\end{center}

% ──────────────────────────────────────────────────────────────
\subsection{Charts and atlases}

The maps $\psi_\alpha$ are called \textbf{charts} (in mathematics)
or \textbf{coordinate systems} (in physics).

The definition as stated depends on the choice of cover
$\{O_\alpha\}$ and charts $\{\psi_\alpha\}$.
Adding a new chart $\psi'$ compatible with the existing ones
would formally give a ``new'' manifold, even though it carries
no new information.  To eliminate this ambiguity we require the
atlas $\{\psi_\alpha\}$ to be \textbf{maximal}: \emph{all}
coordinate systems compatible with conditions~(2) and~(3) are
included.

\begin{intuition}[Maximal atlas]
  Requiring maximality is a bookkeeping device.
  In practice one specifies a convenient atlas and declares
  that every compatible chart is implicitly included.
\end{intuition}

% ──────────────────────────────────────────────────────────────
\subsection{Examples}\label{sec:manifold-examples}

\subsubsection{$\Rn{n}$ as a manifold}

$\Rn{n}$ is the trivial example: it can be covered by a
\emph{single} chart $O_1 = \Rn{n}$, $\psi_1 = \id$.
As a manifold, $\Rn{n}$ has uncountably many compatible covers
(any collection of open sets that covers $\Rn{n}$, together with
the identity or other smooth maps, will do).

\subsubsection{Spacetime}

Minkowski spacetime is $\R^1\times\R^3 \cong \R^{1,3} \cong \Rn{4}$
as a manifold.  (The Lorentzian structure---the metric
signature---comes later.)

\subsubsection{The sphere $S^n$}
\label{sec:sphere}

The $n$-sphere is
\begin{equation}\label{eq:Sn-def}
  S^n = \bigl\{(x^1,\dots,x^{n+1})\in\Rn{n+1}
    \;\big|\; \textstyle\sum_{j=1}^{n+1}(x^j)^2 = 1\bigr\}.
\end{equation}
$S^n$ cannot be covered by a single chart (it is compact, while
any chart image is an open subset of $\Rn{n}$).
Define the $2(n{+}1)$ open sets
\begin{equation}\label{eq:Sn-open-sets}
  O_k^+ = \{(x^1,\dots,x^{n+1})\in S^n \mid x^k > 0\}\,,
  \qquad
  O_k^- = \{(x^1,\dots,x^{n+1})\in S^n \mid x^k < 0\}\,,
\end{equation}
for $k = 1,\dots,n{+}1$.
These cover $S^n$.  The corresponding charts are
\begin{equation}\label{eq:Sn-charts}
  \psi_k^\pm \colon O_k^\pm \to \Rn{n}\,,
  \qquad
  \psi_k^\pm(x^1,\dots,x^{n+1})
    = (x^1,\dots,x^{k-1},x^{k+1},\dots,x^{n+1})\,,
\end{equation}
i.e.\ one simply drops the $k$-th coordinate (whose sign is
determined by the choice of $O_k^\pm$).

\medskip
\noindent\textbf{Special case: $S^2$.}\quad
Consider the charts $\psi_x^+$ (drop $x$, restricted to $x>0$)
and $\psi_y^-$ (drop $y$, restricted to $y<0$).
On the overlap, the transition function is
\[
  (\psi_y^-\circ(\psi_x^+)^{-1})(y,z)
    = \bigl(\sqrt{1-y^2-z^2},\; z\bigr).
\]

\begin{center}
\begin{tikzpicture}[scale=0.9]
  % Sphere
  \shade[ball color=cgblue!15, opacity=0.6] (0,0) circle (1.4);
  \draw[thick, spacecadet] (0,0) circle (1.4);
  % Equator
  \draw[thick, spacecadet, dashed]
    (-1.4,0) arc (180:360:1.4 and 0.45);
  \draw[thick, spacecadet]
    (-1.4,0) arc (180:0:1.4 and 0.45);
  % Axes
  \draw[axisstyle, spacecadet!60] (0,-1.8) -- (0,1.8) node[right]{$y$};
  \draw[axisstyle, spacecadet!60] (-1.8,0) -- (1.8,0) node[below]{$z$};
  \draw[axisstyle, spacecadet!60] (0.7,-0.7) -- (-0.9,0.9) node[left]{$x$};
  % Labels
  \node[font=\sf\small, text=cgblue] at (1.9,1.2) {$S^2$};
  \node[font=\scriptsize, text=munsell] at (1.6,-0.6) {$O_x^+$};
\end{tikzpicture}
\end{center}

\begin{exercise}\label{ex:Sn-transitions}
  Find the remaining transition functions for the atlas
  $\{\psi_k^\pm\}$ of $S^n$ and show that they are all $C^\infty$.
\end{exercise}

\begin{intuition}[Do we need manifolds?]
  Yes!  The non-trivial topology of manifolds is essential for
  describing cosmological spacetimes (which may be compact) and
  black hole spacetimes (which have non-trivial causal structure).
\end{intuition}

% ──────────────────────────────────────────────────────────────
\subsection{New manifolds from old: product manifolds}

Suppose $\M$ and $\M'$ are manifolds of dimension $n$ and $n'$,
respectively.  We can form the \textbf{product manifold}
$\M\times\M'$.

Let $\psi_\alpha\colon O_\alpha\to U_\alpha$ and
$\psi'_\beta\colon O'_\beta\to U'_\beta$ be charts for $\M$ and
$\M'$.  Define charts for $\M\times\M'$ via
\[
  \psi_{\alpha\beta}\colon O_\alpha\times O'_\beta
    \;\to\; U_\alpha\times U'_\beta
    \;\subset\; \Rn{n+n'}\,,
\]
with
\begin{equation}\label{eq:product-chart}
  \psi_{\alpha\beta}(p,p')
    = \bigl(\psi_\alpha(p),\;\psi'_\beta(p')\bigr)
    \;\in\; \Rn{n+n'}\,.
\end{equation}

\begin{exercise}\label{ex:product-manifold}
  Verify that $\M\times\M'$ with the charts
  $\{\psi_{\alpha\beta}\}$ satisfies the manifold axioms.
\end{exercise}

\begin{example}[Products build familiar spaces]
\leavevmode
\begin{itemize}
  \item $\Rn{n} = \underbrace{\R\times\R\times\cdots\times\R}_{n}$.
  \item The \emph{torus}: $T^2 = S^1\times S^1$.
  \item With just $\R$ and $S^n$ and their products, one can
    construct many of the manifolds relevant to general relativity.
\end{itemize}
\end{example}

% ──────────────────────────────────────────────────────────────
\subsection{Smooth maps and diffeomorphisms}

Let $\M$ and $\M'$ be manifolds with charts $\{\psi_\alpha\}$ and
$\{\psi'_\beta\}$.

\begin{definition}[Smooth map]\label{def:smooth-map}
  A map $f\colon\M\to\M'$ is said to be $C^\infty$ (\emph{smooth})
  if for all $\alpha,\beta$ the composite
  \[
    \psi'_\beta\circ f\circ\psi_\alpha^{-1}
      \colon U_\alpha \to U'_\beta
  \]
  is $C^\infty$ (as a map between open subsets of Euclidean spaces).
\end{definition}

\begin{center}
\begin{tikzpicture}[scale=0.85]
  % M
  \draw[thick, spacecadet, rounded corners=10pt]
    (0,0.3) to[out=30,in=150] (2.5,0.4)
    to[out=-30,in=30] (2.6,-0.6)
    to[out=210,in=-10] (0,-0.5) -- cycle;
  \node[font=\sf\small, text=cgblue] at (1.3,0.7) {$\M$};
  \draw[thick, cgblue, fill=cgblue!8, rounded corners=4pt]
    (0.4,0.0) to[out=30,in=170] (1.8,0.1)
    to[out=-10,in=40] (1.7,-0.3)
    to[out=200,in=-10] (0.3,-0.3) -- cycle;
  \node[font=\scriptsize] at (1.0,-0.1) {$O_\alpha$};

  % f arrow
  \draw[thick, -{Stealth[length=6pt]}, spacecadet]
    (2.9,-0.1) -- (4.6,-0.1)
    node[midway, above, font=\small]{$f$};

  % M'
  \begin{scope}[shift={(5,0)}]
    \draw[thick, spacecadet, rounded corners=10pt]
      (0,0.3) to[out=30,in=150] (2.5,0.4)
      to[out=-30,in=30] (2.6,-0.6)
      to[out=210,in=-10] (0,-0.5) -- cycle;
    \node[font=\sf\small, text=cgblue] at (1.3,0.7) {$\M'$};
    \draw[thick, munsell, fill=munsell!8, rounded corners=4pt]
      (0.5,0.0) to[out=30,in=170] (1.9,0.1)
      to[out=-10,in=40] (1.8,-0.3)
      to[out=200,in=-10] (0.4,-0.3) -- cycle;
    \node[font=\scriptsize] at (1.1,-0.1) {$O'_\beta$};
  \end{scope}

  % psi_alpha down-left
  \draw[thick, -{Stealth[length=5pt]}, cgblue]
    (0.5,-0.7) -- (0.0,-1.8)
    node[midway, left, font=\scriptsize]{$\psi_\alpha$};

  % psi'_beta down-right
  \draw[thick, -{Stealth[length=5pt]}, munsell]
    (6.8,-0.7) -- (7.3,-1.8)
    node[midway, right, font=\scriptsize]{$\psi'_\beta$};

  % U_alpha
  \begin{scope}[shift={(-0.8,-2.6)}]
    \draw[thick, cgblue, rounded corners=3pt]
      (0.2,-0.3) to[out=20,in=210] (1.5,0.3)
      to[out=-20,in=70] (1.6,-0.2)
      to[out=220,in=-10] (0.2,-0.3);
    \node[font=\scriptsize, text=cgblue] at (0.9,0.0) {$U_\alpha$};
    \node[font=\sf\scriptsize, text=spacecadet] at (-0.2,0.2) {$\Rn{n}$};
  \end{scope}

  % U'_beta
  \begin{scope}[shift={(6.5,-2.6)}]
    \draw[thick, munsell, rounded corners=3pt]
      (0.2,-0.3) to[out=20,in=210] (1.5,0.3)
      to[out=-20,in=70] (1.6,-0.2)
      to[out=220,in=-10] (0.2,-0.3);
    \node[font=\scriptsize, text=munsell] at (0.9,0.0) {$U'_\beta$};
    \node[font=\sf\scriptsize, text=spacecadet] at (1.9,0.2) {$\Rn{n'}$};
  \end{scope}

  % Composite arrow
  \draw[thick, -{Stealth[length=5pt]}, spacecadet!70]
    (1.0,-2.8) to[out=-10,in=190] (6.8,-2.8)
    node[midway, below, font=\scriptsize]
      {$\psi'_\beta\circ f\circ\psi_\alpha^{-1}$};
\end{tikzpicture}
\end{center}

\begin{definition}[Diffeomorphism]\label{def:diffeomorphism}
  A map $f\colon\M\to\M'$ is a \textbf{diffeomorphism} if it is
  $C^\infty$, one-to-one, onto, and its inverse $f^{-1}$ is also
  $C^\infty$.  If such a map exists, $\M$ and $\M'$ are said to be
  \textbf{diffeomorphic}: they have identical manifold structure.
\end{definition}

% ──────────────────────────────────────────────────────────────
\subsection{Vectors on manifolds}\label{sec:vectors-manifolds}

Euclidean space $\Rn{n}$ (and Minkowski space) carries a natural
\emph{global} vector space structure: $V = \Rn{n}$ is simultaneously
a manifold and a vector space.

For a general manifold this is \textbf{no longer true}---there is
no natural additive structure.

\begin{example}[$S^2$ is not a vector space]
  The 2-sphere $S^2$ is a manifold, but it is \emph{not} a vector
  space: there is no meaningful way to ``add'' two points on $S^2$.
\end{example}

It is, however, still possible to associate a vector space to each
point of a manifold.  For a manifold $\M$ embedded in $\Rn{m}$
(such as $S^2\subset\Rn{3}$), one can visualise the
\textbf{tangent plane} at a point~$p$:

\begin{center}
\begin{tikzpicture}[scale=0.9]
  % Sphere
  \shade[ball color=cgblue!12, opacity=0.5] (0,0) circle (1.4);
  \draw[thick, spacecadet] (0,0) circle (1.4);
  % Equator
  \draw[thick, spacecadet, dashed]
    (-1.4,0) arc (180:360:1.4 and 0.45);
  \draw[thick, spacecadet]
    (-1.4,0) arc (180:0:1.4 and 0.45);
  % Point p
  \node[circle, fill=banana, inner sep=1.8pt] (P) at (0.4,1.2) {};
  \node[font=\small, above right] at (P) {$p$};
  % Tangent plane
  \draw[thick, munsell, fill=munsell!8, opacity=0.7]
    (-1.0,1.8) -- (1.8,1.8) -- (2.2,0.8) -- (-0.6,0.8) -- cycle;
  \node[font=\sf\small, text=munsell] at (2.5,1.4) {$\Tp$};
  % Label
  \node[font=\sf\small, text=cgblue] at (-1.8,0.8) {$S^2\subset\Rn{3}$};
\end{tikzpicture}
\end{center}

But we want an \emph{intrinsic} definition, independent of any
embedding.

\begin{keyresult}[Intrinsic tangent vectors]
  The key idea: identify the vector space of tangent vectors at a
  point with the vector space of \emph{directional derivatives}
  at that point.  Directional derivatives can be defined using
  only the manifold structure---no embedding required.
  This is developed in the next lecture.
\end{keyresult}

\begin{figure}[htbp]
  \centering
  % fig_lec03_ellipsoid.tex — Geodesics + tangent/parallel transport on ellipsoid
% Data: generated by scripts/sim_lec03.jl

% Panel (a): Ellipsoid with geodesics
\begin{center}
\begin{tikzpicture}
\begin{axis}[
    gr3d,
    xlabel={$x$}, ylabel={$y$}, zlabel={$z$},
    title={Geodesics on a triaxial ellipsoid},
    view={-25}{30},
    xmin=-1.7, xmax=1.7,
    ymin=-1.2, ymax=1.2,
    zmin=-0.85, zmax=0.85,
]
    % Ellipsoid surface (faint)
    \addplot3[
        surf,
        opacity=0.25,
        shader=interp,
        colormap name=grsurface,
        forget plot,
        mesh/rows=60,
    ] table {data/lec03_ellipsoid.dat};

    % Geodesics
    \addplot3[cgblue, line width=1.2pt, no markers]
        table {data/lec03_geodesic_1.dat};
    \addplot3[munsell, line width=1.2pt, no markers]
        table {data/lec03_geodesic_2.dat};
    \addplot3[banana!80!black, line width=1.2pt, no markers]
        table {data/lec03_geodesic_3.dat};
    \addplot3[spacecadet, line width=1.2pt, no markers]
        table {data/lec03_geodesic_4.dat};
\end{axis}
\end{tikzpicture}

\bigskip

% Panel (b): Tangent vectors + parallel transport along geodesic 1
\begin{tikzpicture}
\begin{axis}[
    gr3d,
    xlabel={$x$}, ylabel={$y$}, zlabel={$z$},
    title={Tangent vectors and parallel transport},
    view={-25}{30},
    xmin=-1.7, xmax=1.7,
    ymin=-1.2, ymax=1.2,
    zmin=-0.85, zmax=0.85,
]
    % Ellipsoid surface (very faint)
    \addplot3[
        surf,
        opacity=0.15,
        shader=interp,
        colormap name=grsurface,
        forget plot,
        mesh/rows=60,
    ] table {data/lec03_ellipsoid.dat};

    % Geodesic curve
    \addplot3[cgblue, line width=1.0pt, no markers]
        table {data/lec03_geodesic_1.dat};

    % Tangent vectors (banana arrows)
    \addplot3[
        quiver={u=\thisrowno{3}, v=\thisrowno{4}, w=\thisrowno{5}},
        -{Stealth[length=4pt, width=3pt]},
        banana!80!black, line width=0.8pt,
        no markers,
    ] table[x index=0, y index=1, z index=2] {data/lec03_tangent_1.dat};

    % Parallel-transported vectors (munsell arrows)
    \addplot3[
        quiver={u=\thisrowno{3}, v=\thisrowno{4}, w=\thisrowno{5}},
        -{Stealth[length=4pt, width=3pt]},
        munsell, line width=0.8pt,
        no markers,
    ] table[x index=0, y index=1, z index=2] {data/lec03_parallel_1.dat};

    % Legend
    \addlegendimage{banana!80!black, line width=0.8pt, -{Stealth[length=4pt]}}
    \addlegendentry{tangent $\dot\gamma$}
    \addlegendimage{munsell, line width=0.8pt, -{Stealth[length=4pt]}}
    \addlegendentry{parallel transport}
\end{axis}
\end{tikzpicture}
\end{center}

  \caption{Top: geodesics on a triaxial ellipsoid
    ($a=1.5$, $b=1.0$, $c=0.7$).  The curves are shortest paths
    intrinsic to the surface.
    Bottom: tangent vectors (gold) and a parallel-transported
    vector (teal) along a geodesic, illustrating how
    parallel transport preserves inner products but the
    transported vector rotates relative to the tangent
    direction.}\label{fig:ellipsoid-geodesics}
\end{figure}

% ──────────────────────────────────────────────────────────────
\subsection{Formal definitions (summary)}

We collect the precise versions of the definitions introduced
above for reference.

\begin{definition}[Locally Euclidean space]\label{def:loc-eucl}
  A \textbf{locally Euclidean space} of dimension~$d$ is a
  Hausdorff topological space~$\M$ such that every point has a
  neighbourhood homeomorphic to an open subset of~$\Rn{d}$.
\end{definition}

\begin{definition}[Differentiable structure]\label{def:diff-struct}
  A \textbf{differentiable structure}~$\mathcal{F}$ of class~$C^k$
  ($1\le k\le\infty$) on a locally Euclidean space~$\M$ is a
  collection of coordinate systems
  $\{(O_\alpha,\psi_\alpha)\mid\alpha\in A\}$ such that:
  \begin{enumerate}
    \item[\textup{(a)}] $\displaystyle\bigcup_{\alpha\in A}
      O_\alpha = \M$.
    \item[\textup{(b)}] $\psi_\alpha\circ\psi_\beta^{-1}$ is
      $C^k$ for all $\alpha,\beta\in A$.
    \item[\textup{(c)}] The collection $\mathcal{F}$ is maximal
      with respect to~(b).
  \end{enumerate}
\end{definition}

A $d$-dimensional \textbf{differentiable manifold of class $C^k$}
is a pair $(\M,\mathcal{F})$ consisting of a second-countable,
locally Euclidean space~$\M$ together with a differentiable
structure~$\mathcal{F}$ of class~$C^k$.
(The $C^k$ condition can be generalised to $C^\infty$, complex
analytic, etc.)
