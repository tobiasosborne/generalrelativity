%!TEX root = ../GeneralRelativity.tex
% ──────────────────────────────────────────────────────────────
%  Lecture 22 — Null geodesics in Schwarzschild: light deflection
% ──────────────────────────────────────────────────────────────

\subsection{Null geodesics and the deflection of light}%
\label{sec:null-geodesics}

For the final lecture of this course we study the trajectories of
light rays---null geodesics ($\kappa = 0$)---in the Schwarzschild
spacetime.  For me, this is one of the most spectacular predictions
of general relativity: that light itself is curved by the
gravitational field, and that this curvature can be detected with
terrestrial observations.

% ──────────────────────────────────────────────────────────────
\subsubsection{The null effective potential}%
\label{sec:null-potential}

Setting $\kappa = 0$ in the effective
potential~\eqref{eq:effective-potential}:
\begin{equation}\label{eq:null-potential}
  V_{\text{null}}(r) = \frac{L^2}{2r^2}\Bigl(1
    - \frac{2M}{r}\Bigr)
  = \frac{L^2}{2r^3}\,(r - 2M)\,.
\end{equation}
The energy equation~\eqref{eq:radial-energy} becomes
\begin{equation}\label{eq:null-energy}
  \tfrac{1}{2}\,\dot r^2 + V_{\text{null}}(r)
    = \tfrac{1}{2}\, E^2\,.
\end{equation}
The potential has a single maximum at $r = 3M$ (exercise: set
$dV_{\text{null}}/dr = 0$) with height
$V_{\text{max}} = L^2 M / (2(3M)^3) = L^2/(54M^2)$.

\begin{intuition}[Scattering vs.\ capture]
  A photon approaching from infinity either has enough energy to
  crest the potential barrier and plunge to $r = 0$
  (\textbf{capture}), or it reflects off the barrier and escapes
  back to infinity (\textbf{scattering}).  The critical case is
  the unstable circular photon orbit at $r = 3M$---the
  \textbf{photon sphere}.
\end{intuition}

% ──────────────────────────────────────────────────────────────
\subsubsection{The impact parameter and capture cross section}%
\label{sec:impact-parameter}

The ratio $b = L/E$ has a natural interpretation.  In flat
spacetime ($M = 0$), a null geodesic travelling along a straight
line at distance~$r_0$ from the origin satisfies $b = r_0$:
the impact parameter equals the distance of closest approach.  For
the Schwarzschild geometry, which is \textbf{asymptotically flat},
$b = L/E$ represents the \textbf{apparent impact parameter} as
seen by a distant observer.

A photon is captured if its energy exceeds the potential maximum:
$\frac{1}{2}E^2 > V_{\text{max}}$, i.e.\
$L^2/E^2 < 27M^2$.  The \textbf{capture radius} is therefore
\begin{equation}\label{eq:capture-radius}
  b_c = 3\sqrt{3}\; M\,,
\end{equation}
and the \textbf{capture cross section} is
\begin{equation}\label{eq:capture-cross-section}
  \sigma = \pi\, b_c^2 = 27\pi\, M^2\,.
\end{equation}
Any photon with impact parameter $b < b_c$ is captured; those with
$b > b_c$ are deflected and escape.

% ──────────────────────────────────────────────────────────────
\subsubsection{Total angular deflection}%
\label{sec:angular-deflection}

For a scattering trajectory ($b > b_c$), the photon approaches from
$r = \infty$, reaches a turning point at $r = R_0$ (where
$\dot r = 0$, i.e.\ $E^2 = L^2(R_0 - 2M)/R_0^3$), and recedes
to $r = \infty$.  The total change in the angular coordinate is
\begin{equation}\label{eq:delta-phi-integral}
  \Delta\phi = 2 \int_{R_0}^{\infty}
    \frac{L\, dr}{r^2\sqrt{E^2 - \frac{L^2}{r^3}(r - 2M)}}
\end{equation}
(the factor of~$2$ accounts for the symmetry of the trajectory
about the turning point).

Substituting $u = 1/r$ and eliminating $b$ via the turning-point
condition $R_0^3 = b^2(R_0 - 2M)$:
\begin{equation}\label{eq:delta-phi-u}
  \Delta\phi = 2 \int_0^{1/R_0}
    \frac{du}{\sqrt{R_0^{-2} - 2M R_0^{-3} - u^2 + 2M u^3}}\,.
\end{equation}

\medskip\noindent
\textbf{Sanity check.}\;
For $M = 0$ (flat spacetime), $\Delta\phi = \pi$ (exercise)---the
photon travels in a straight line and the total angular change from
$\phi(-\infty)$ to $\phi(+\infty)$ is exactly~$\pi$, as expected.

% ──────────────────────────────────────────────────────────────
\subsubsection{Perturbative expansion for small $M$}%
\label{sec:light-deflection-expansion}

For realistic celestial bodies ($M/R_0 \ll 1$), we expand
$\Delta\phi$ to first order in~$M$.  The expansion is subtle
because the turning point $R_0$ itself depends on~$M$; one
promotes $M$ and~$R_0$ to independent variables and Taylor-expands.
Using the turning-point condition to relate $R_0$ and $b$
to zeroth order ($R_0 = b + O(M)$), one finds after a careful
calculation:
\begin{equation}\label{eq:light-deflection}
  \eqbox{\Delta\phi \approx \pi + \frac{4M}{b}}
\end{equation}
The \textbf{deflection angle} (the deviation from the flat-spacetime
value of~$\pi$) is therefore
\begin{equation}\label{eq:deflection-angle}
  \delta\phi = \Delta\phi - \pi = \frac{4M}{b}\,.
\end{equation}
Restoring factors of $G$ and $c$:
$\delta\phi = 4GM/(c^2 b)$.

\begin{keyresult}[Deflection of light by a massive body]
  A light ray passing a spherically symmetric body of mass~$M$ at
  impact parameter~$b$ is deflected by an angle
  \[
    \delta\phi = \frac{4M}{b}
  \]
  (in geometrised units).  For a ray grazing the surface of the
  Sun ($M_\odot \approx 1.5\,\text{km}$,
  $R_\odot \approx 7 \times 10^5\,\text{km}$):
  \[
    \delta\phi \approx \frac{4 \times 1.5}{7 \times 10^5}
      \approx 8.5 \times 10^{-6}\;\text{rad}
      \approx 1.75''\,.
  \]
\end{keyresult}

\begin{historical}[The 1919 eclipse expedition]
  The deflection of light by the Sun was confirmed by the
  Dyson--Eddington--Davidson expedition during the total solar
  eclipse of 29~May 1919.  Photographs of stars near the eclipsed
  Sun showed apparent displacements consistent with Einstein's
  prediction of~$1.75''$, rather than the $0.87''$ predicted by a
  naive Newtonian calculation (which treats light as a particle
  subject to Newtonian gravity).  The result was announced on
  6~November 1919 and made Einstein an international celebrity
  overnight.  Modern measurements using radio waves from quasars
  bent by the Sun achieve far better precision and have confirmed
  the prediction to within $0.01\%$.
\end{historical}

% ══════════════════════════════════════════════════════════════
%  Conclusion
% ══════════════════════════════════════════════════════════════

\bigskip
\begin{center}
  \rule{0.5\textwidth}{0.4pt}
\end{center}
\bigskip

This concludes our introduction to general relativity.  We have
travelled from the equivalence principle and the mathematics of
manifolds, through the differential geometry of connections,
curvature, and geodesics, to Einstein's field equations and their
consequences: gravitational waves, the expanding universe, and the
geometry of spacetime around stars and black holes.  We have only
scratched the surface of an extraordinary theory---one that has
been the subject of active research for over a century and whose
deepest mysteries, from the nature of dark energy to the quantum
structure of spacetime, remain open.

\begin{intuition}[An extraordinary theory]
  General relativity is a theory of unparalleled beauty in physics.
  From a handful of physical principles---the equivalence of
  gravitational and inertial mass, the requirement that the laws of
  physics be independent of the coordinate system---Einstein
  constructed a framework that describes the large-scale structure
  of the universe, the dynamics of black holes, and the propagation
  of gravitational waves, all as consequences of the curvature of
  spacetime.  That the same theory which explains the perihelion of
  Mercury also predicts the Big Bang and gravitational waves from
  merging black holes is, by any measure, breathtaking.
\end{intuition}
