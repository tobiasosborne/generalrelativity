%!TEX root = ../GeneralRelativity.tex
% ──────────────────────────────────────────────────────────────
%  Lecture 1 — Prerelativity Gravitation
% ──────────────────────────────────────────────────────────────

\section{Prerelativity gravitation}\label{sec:prerelativity}

\subsection{Course overview}

General relativity is, by wide consensus among physicists, one of the
most beautiful and elegant physical theories we possess.  The connection
it draws between the curvature of spacetime and the motion of freely
falling bodies has an almost miraculous quality---there is something
deeply compelling, perhaps even inevitable, about the way the theory
holds together.

That said, general relativity is also \emph{wrong}, or at least
incomplete: as far as we know, it is not compatible with quantum
mechanics in any fully satisfactory way.  A deeper, more fundamental
theory must underlie both.  Nonetheless, the predictions of general
relativity have been confirmed to extraordinary accuracy across an
enormous range of cosmological and astrophysical phenomena, and it
remains the indispensable framework for understanding gravity.

\medskip
\noindent\textbf{Text.} Wald, \emph{General Relativity}~\cite{Wald84}.
We follow Wald closely, though we deviate where other sources offer
clearer treatments.\\
\textbf{Other sources:}
\begin{itemize}
  \item Misner, Thorne \& Wheeler~\cite{MTW73} --- the ``telephone
    book,'' a comprehensive and physically rich discussion of general
    relativity.
  \item Weinberg~\cite{Weinberg72} --- an excellent resource for
    explicit calculations.
\end{itemize}

\noindent\textbf{Structure.}
The course proceeds from mathematical methods to the physics of
gravitation.  The emphasis falls deliberately on the mathematical
side---in particular, on differential geometry.  While it is perfectly
possible to do general relativity concretely, working in a single
coordinate patch and never invoking coordinate-free notation, a
command of the underlying differential geometry is indispensable for
further studies in theoretical physics.  Anytime one wishes to do
serious work in quantum gravity or string theory, fluency in
differential geometry is a prerequisite; the investment made here will
pay dividends well beyond this course.

\begin{center}
\begin{tikzpicture}[
  block/.style={rectangle, rounded corners=3pt, draw=cgblue,
    fill=cgblue!10, minimum height=1.8em, text=spacecadet,
    font=\sf\small, inner sep=5pt},
  arr/.style={-{Stealth[length=5pt]}, thick, cgblue}
]
  \node[block] (math) {Differential geometry};
  \node[block, right=1.8cm of math] (phys) {Physics of gravitation};
  \node[block, below=0.8cm of math, font=\sf\small\itshape,
    draw=spacecadet!30, fill=isabelline]
    (qg) {Quantum gravity \& string theory};
  \draw[arr] (math) -- (phys);
  \draw[arr, spacecadet!50] (math) -- (qg);
  \draw[arr, spacecadet!50] (phys.south) to[out=-90,in=0] (qg.east);
\end{tikzpicture}
\end{center}

\noindent\textbf{Videos.} Lenny Susskind's lecture series on general
relativity offers an absolutely different---and completely
complementary---perspective to the one presented here.

\medskip\noindent\textbf{Prerequisites:}
\begin{itemize}
  \item Special relativity
  \item Calculus
  \item Linear algebra
  \item Combinatorics
\end{itemize}

% ──────────────────────────────────────────────────────────────
\subsection{Newtonian mechanics and inertial frames}

\textbf{Assumption:} Newtonian physics presupposes the existence of a
preferred reference system and clocks.  This is the core assumption
that will be broken in the transition from pre-relativity to
relativity physics: a common thread in the development of physics is
that one makes (often implicit) assumptions in a physical theory, and
as one learns more about the universe one is forced to re-evaluate and
potentially discard them.

\begin{center}
\begin{tikzpicture}
  \draw[axisstyle] (0,0,0) -- (2.5,0,0) node[right]{$x$};
  \draw[axisstyle] (0,0,0) -- (0,2,0)   node[above]{$z$};
  \draw[axisstyle] (0,0,0) -- (0,0,2)   node[below left]{$y$};
\end{tikzpicture}
\end{center}

A free test mass (i.e.\ no force acts) moves along a straight trajectory
at a constant rate, uniformly in time---constant displacement per unit
time.  This defines an \emph{inertial reference frame}.  There is one
preferred reference frame, but one can obtain others by performing
Galilean boosts: an observer moving at constant velocity with respect
to the preferred frame will also see free test masses moving along
straight trajectories at constant rates.

Newton's second law is written with respect to an inertial reference system:
\begin{equation}\label{eq:newton2}
  \eqbox{\vb{F}(\vb{x}(t),\, t) \;=\; m_I\, \ddt{\vb{x}}(t)}\,,
\end{equation}
where $\vb{F}$ is the force, $m_I$ is the \emph{inertial mass}, and
$\ddt{\vb{x}}$ is the acceleration.

\subsubsection{Non-inertial reference frames}
With respect to a non-inertial reference frame, define three
(time-dependent) unit vectors $\uv{x}'(t)$, $\uv{y}'(t)$, $\uv{z}'(t)$.
A trajectory can be expressed in either basis:
\begin{align}
  \vb{w}(t) &= w_x(t)\,\uv{x} + w_y(t)\,\uv{y} + w_z(t)\,\uv{z}
  \label{eq:traj-inertial}\\
  &= w_{x'}(t)\,\uv{x}'(t) + w_{y'}(t)\,\uv{y}'(t) + w_{z'}(t)\,\uv{z}'(t).
  \label{eq:traj-noninertial}
\end{align}

\noindent\textbf{Index notation.}
For a vector $\vb{a}$ and a matrix $M$:
\[
  [\vb{a}]_x = a_x\,,\qquad
  [M]_{xy} = M_{xy}\,.
\]

Physics in non-inertial frames (as captured by Newton's second law) gives
rise to additional \emph{inertial force} terms, e.g.\ centrifugal,
Coriolis, Euler forces, etc.

\begin{exercise}
  Distinguish ``true forces'' from apparent (inertial) forces.
\end{exercise}

\begin{intuition}[Why inertial forces matter]
  In a rotating reference frame, a ball thrown in a straight line
  \emph{appears} to curve (Coriolis effect).  The ``force'' causing this
  curve is not due to any physical interaction---it is an artefact of the
  non-inertial frame.  In Newtonian physics, gravity is classified as a
  \emph{true} force.  A central insight of general relativity is that
  gravity, too, can be understood as an artefact of geometry---effectively
  reclassifying it alongside the inertial forces.
\end{intuition}

% ──────────────────────────────────────────────────────────────
\subsection{Mass: three conceptually distinct quantities}

\begin{enumerate}[(i)]
  \item \textbf{Inertial mass} $m_I$: the constant of proportionality in
    Newton's second law,
    \begin{equation}\label{eq:newton2-abstract}
      F = m_I\, a\,.
    \end{equation}

  \item \textbf{Active gravitational mass} $m_g^{(a)}$: the mass that
    \emph{produces} the gravitational field that other masses respond to.
    The gravitational field of a point source at position $\vb{x}'$ is
    \begin{equation}\label{eq:grav-field-point}
      \vb{g}(\vb{x})
        = -\grav\, m_g^{(a)}\,
          \frac{\vb{x} - \vb{x}'}{\norm{\vb{x} - \vb{x}'}^3}\,,
    \end{equation}
    where $\grav$ is the gravitational constant,
    \begin{equation}
      \grav = 6.674\,30(15) \times 10^{-11}\;\mathrm{m}^3\,
              \mathrm{kg}^{-1}\,\mathrm{s}^{-2}\,.
    \end{equation}

  \item \textbf{Passive gravitational mass} $m_g^{(p)}$: the mass that
    \emph{responds} to an external gravitational field by accelerating,
    \begin{equation}\label{eq:grav-force}
      \vb{F}_g = m_g^{(p)}\,\vb{g}\,.
    \end{equation}
\end{enumerate}

It is assumed that $m_I,\, m_g^{(a)},\, m_g^{(p)} \geq 0$.  This
non-negativity is an assumption, not a theorem---one \emph{cannot}
deduce it from the other laws of Newtonian physics.

\begin{remark}
  It is possible to build a consistent Newtonian physics with negative
  masses.  The consequences are exotic---negative masses behave
  somewhat like negative charges, and all manner of counterintuitive
  phenomena arise---but the theory does not immediately break down.
  The non-negativity of mass is supported by extensive experimental
  evidence, not by logical necessity.
\end{remark}

\begin{exercise}
  Using Newton's third law, argue that $m_g^{(a)}/m_g^{(p)}$ is a
  universal constant.
\end{exercise}

Choosing units: $m_g^{(a)} = m_g^{(p)} \equiv m_g$.

\begin{intuition}[Three masses, one body]
  Every massive body carries \emph{three} logically independent
  numbers: how much it resists acceleration ($m_I$), how strongly
  it sources a gravitational field ($m_g^{(a)}$), and how strongly
  it responds to one ($m_g^{(p)}$).  That nature chooses to make all
  three equal is not obvious---it is a deep empirical fact that general
  relativity elevates to a geometric principle.
\end{intuition}

% ──────────────────────────────────────────────────────────────
\subsection{The universality of free fall}\label{sec:UFF}

Extensive experimental evidence suggests $m_g = m_I$.

\begin{definition}[Universality of Free Fall]\label{def:UFF}
  In a gravitational field $\vb{g}$, the orbit $\vb{x}(t)$ of a test mass
  depends \emph{only} on the initial conditions.  That is, identical
  initial conditions produce identical trajectories regardless of the
  composition or internal structure of the test mass.
\end{definition}

Consider a test mass in a gravitational field:

\begin{center}
\begin{tikzpicture}[scale=1.1]
  % Background field region
  \begin{scope}[on background layer]
    \fill[isabelline, rounded corners=5pt] (-0.5,-0.8) rectangle (6.2,3.5);
  \end{scope}
  % Source mass
  \node[pointmass, minimum size=8pt,
    label={[font=\small]below:{source mass $M$}}] (M) at (5.2,1.2) {};
  % Field arrows (radial, pointing inward)
  \foreach \a in {120,140,...,240} {
    \draw[fieldline] (M) ++(\a:2.0) -- ++(\a:-0.7);
  }
  \foreach \a in {130,150,...,230} {
    \draw[fieldline, thin, opacity=0.5] (M) ++(\a:2.8) -- ++(\a:-0.6);
  }
  % Label field
  \node[font=\small] at (2.8,0.0) {$\vb{g}(\vb{x})$};
  % Test mass trajectory
  \node[testmass, minimum size=5pt,
    label={[font=\small]above left:{$\vb{x}(t)$}}] (m) at (2.0,2.2) {};
  \draw[trajectory] (0.3,3.2) to[out=-15,in=155] (m);
  \draw[trajectory] (m) to[out=-25,in=160] (4.0,0.8);
  % Velocity arrow
  \draw[vecstyle] (m) -- ++(0.8,-0.35) node[right, font=\small]{$\dt{\vb{x}}$};
\end{tikzpicture}
\end{center}

The equation of motion reads
\begin{equation}\label{eq:eom-grav}
  m_g\,\vb{g}(\vb{x}(t)) = m_I\,\ddt{\vb{x}}(t)\,,
\end{equation}
or equivalently,
\begin{equation}\label{eq:eom-UFF}
  \eqbox{\ddt{\vb{x}}(t) = \frac{m_g}{m_I}\,\vb{g}(\vb{x}(t))}\,.
\end{equation}
Since $m_g/m_I$ is a universal constant, we may choose units such that
$m_g = m_I$.

\begin{remark}
\begin{enumerate}[(1)]
  \item The universality of free fall is \emph{not} a consequence of
    Newtonian theory---it is a falsifiable hypothesis.  That $m_g =
    m_I$ cannot be proved from Newton's laws; it is an empirical
    statement that could, in principle, be refuted by experiment.
    Experiments testing this relationship continue to this day.
  \item UFF applies specifically to \emph{test masses}.  A test mass
    is an idealised notion: one should visualise it as a ``point''
    particle.  The concept is already subtle in Newtonian physics---a
    finite mass concentrated into an arbitrarily small volume would, in
    general relativity, form a black hole---and it becomes more so when
    test masses are numerous enough to generate their own gravitational
    fields.  These subtleties will be resolved once we have the full
    apparatus of general relativity.
\end{enumerate}
\end{remark}

\begin{historical}[Experimental tests of the equivalence principle]
  The equality $m_g = m_I$ has been tested with extraordinary precision:
  \begin{center}
  \begin{tikzpicture}[
    exp/.style={rectangle, rounded corners=2pt, draw=spacecadet!40,
      fill=isabelline, minimum width=3cm, minimum height=1.4em,
      font=\small, inner sep=4pt, text=spacecadet},
    prec/.style={font=\small\sf, text=munsell!80!black},
    timeline/.style={thick, spacecadet!30}
  ]
    % Timeline
    \draw[timeline] (0,0) -- (11.5,0);
    % Entries
    \node[exp, anchor=south] (N) at (0.6,0.3)  {Newton (1687)};
    \node[prec, anchor=north] at (0.6,-0.2) {$\sim10^{-3}$};

    \node[exp, anchor=south] (E) at (3.7,0.3)
      {E\"otv\"os (1922)};
    \node[prec, anchor=north] at (3.7,-0.2) {$\sim10^{-9}$};

    \node[exp, anchor=south] (LLR) at (6.8,0.3)
      {Lunar laser (1969)};
    \node[prec, anchor=north] at (6.8,-0.2) {$\sim10^{-13}$};

    \node[exp, anchor=south] (MS) at (10.2,0.3)
      {\textsc{microscope} (2022)};
    \node[prec, anchor=north] at (10.2,-0.2) {$\sim10^{-15}$};

    % Dots on timeline
    \foreach \x in {0.6,3.7,6.8,10.2}
      \fill[spacecadet] (\x,0) circle(2pt);
  \end{tikzpicture}
  \end{center}
  \smallskip
  The E\"otv\"os experiment uses a \emph{torsion balance}: two masses of
  different composition are suspended at opposite ends of a bar.  If
  $m_g/m_I$ differed between materials, the bar would rotate.  No such
  rotation has ever been observed.
  The \textsc{microscope} satellite (2017--2022)
  achieved the most stringent test to date, confirming
  $|m_g/m_I - 1| < 10^{-15}$.
\end{historical}

% ──────────────────────────────────────────────────────────────
\subsection{From Newtonian gravity to general relativity}

We now reformulate Newton's gravity as a \emph{field theory}, drawing on
the analogy between Newtonian gravity and electrostatics.  The
motivation is direct: electrodynamics is compatible with special
relativity---Lorentz transformations leave Maxwell's equations
invariant and information propagates at the speed of light.  If
Newtonian gravity looks structurally like electrostatics (and it
does), then perhaps we can add dynamics in the same way and arrive at
a relativistic theory of gravity.  As we shall see, this hope is
partially fulfilled: the approach does lead somewhere, but the
destination is more radical than a simple gravitational analog of
Maxwell's equations.

\subsubsection{Coulomb vs.\ Newton: a detailed comparison}

\begin{center}
\begin{tikzpicture}[scale=0.85]
  % Coulomb side
  \begin{scope}[shift={(-4,0)}]
    \node[font=\sf\bfseries, text=cgblue] at (0,2.4) {Electrostatics};
    % Charges
    \node[circle, fill=red!60, inner sep=3pt, font=\footnotesize\sf,
      text=white] (q1) at (-1.2,0) {$+$};
    \node[circle, fill=blue!60, inner sep=3pt, font=\footnotesize\sf,
      text=white] (q2) at (1.2,0) {$-$};
    % Force arrows
    \draw[vecstyle, red!70!black] (q1) -- ++(0.6,0)
      node[above,font=\footnotesize]{$\vb{F}$};
    \draw[vecstyle, blue!70!black] (q2) -- ++(-0.6,0)
      node[above,font=\footnotesize]{$\vb{F}$};
    % Label
    \node[font=\small, text=spacecadet, text width=3.5cm, align=center]
      at (0,-1.2) {Attractive \emph{and} repulsive};
  \end{scope}
  % Newton side
  \begin{scope}[shift={(4,0)}]
    \node[font=\sf\bfseries, text=cgblue] at (0,2.4) {Newtonian gravity};
    % Masses
    \node[circle, fill=spacecadet, inner sep=3pt,
      font=\footnotesize\sf, text=white] (m1) at (-1.2,0) {$m$};
    \node[circle, fill=spacecadet!70, inner sep=3pt,
      font=\footnotesize\sf, text=white] (m2) at (1.2,0) {$m$};
    % Force arrows (attractive only)
    \draw[vecstyle, spacecadet] (m1) -- ++(0.6,0)
      node[above,font=\footnotesize]{$\vb{F}$};
    \draw[vecstyle, spacecadet] (m2) -- ++(-0.6,0)
      node[above,font=\footnotesize]{$\vb{F}$};
    % Label
    \node[font=\small, text=spacecadet, text width=3.5cm, align=center]
      at (0,-1.2) {Attractive \emph{only}};
  \end{scope}
\end{tikzpicture}
\end{center}

\begin{center}
\small
\begin{tabular}{@{}lcc@{}}
  \toprule
  & \textbf{Electrostatics} & \textbf{Newtonian gravity} \\
  \midrule
  Force law
    & $\displaystyle\vb{F} = k_e\frac{q_1 q_2}{r^2}\,\uv{r}$
    & $\displaystyle\vb{F} = -\grav\frac{m_1 m_2}{r^2}\,\uv{r}$ \\[10pt]
  Field
    & $\vb{E}$
    & $\vb{g}$  \\[4pt]
  Source
    & charge density $\rho_e$
    & mass density $\rho$ \\[4pt]
  Source sign
    & both signs
    & $\rho \geq 0$ only \\[4pt]
  Potential
    & $V$
    & $\phi$  \\[4pt]
  Field from potential
    & $\vb{E} = -\nabla V$
    & $\vb{g} = -\nabla\phi$ \\[4pt]
  Gauss's law
    & $\nabla\cdot\vb{E} = \rho_e/\varepsilon_0$
    & $\nabla\cdot\vb{g} = -4\pi\grav\rho$ \\[4pt]
  Poisson equation
    & $\Delta V = -\rho_e/\varepsilon_0$
    & $\Delta\phi = 4\pi\grav\rho$ \\[4pt]
  \bottomrule
\end{tabular}
\end{center}

\begin{intuition}[The gravity--electrostatics analogy]
  The structural parallel is almost perfect: both are
  $1/r^2$ force laws sourced by a density and governed by a Poisson
  equation.  The crucial difference is that electric charge comes in
  two signs (allowing shielding and cancellation), while mass is always
  positive.  This means gravity cannot be screened---every mass in the
  universe contributes---which is ultimately why a geometric description
  (curved spacetime) becomes necessary.
\end{intuition}

\subsubsection{Gravitational field}
The gravitational field is a map
\[
  \vb{g}\colon \R\times\Rn{3} \to \Rn{3}\,,\qquad
  (t,\,\vb{x}) \mapsto \vb{g}(t,\,\vb{x})\,.
\]

\subsubsection{Sources}
The (gravitational) mass density is
\[
  \rho\colon \R\times\Rn{3} \to \R^+\,,
\]
and the field equation reads
\begin{equation}\label{eq:gauss-grav}
  \eqbox{\nabla\cdot\vb{g} = -4\pi\grav\,\rho}\,.
\end{equation}

\subsubsection{Gravitational potential}
Since $\nabla\times\vb{g} = 0$, the field $\vb{g}$ is conservative.
By Poincar\'e's lemma there exists a scalar potential
$\phi\colon \R\times\Rn{3}\to\R$ such that
\begin{equation}\label{eq:g-from-phi}
  \eqbox{\vb{g} = -\nabla\phi}
  \qquad\text{(sign convention).}
\end{equation}
Substituting into~\eqref{eq:gauss-grav}:
\begin{equation}\label{eq:poisson}
  \eqbox{\lap\phi = 4\pi\grav\,\rho}
  \qquad\text{(Poisson's equation).}
\end{equation}

\subsubsection{Equation of motion for test masses}
\begin{equation}\label{eq:eom-potential}
  \eqbox{\ddt{\vb{x}}(t) = -\nabla\phi\big(t,\,\vb{x}(t)\big)}\,.
\end{equation}

\subsubsection{Boundary conditions and uniqueness}
Assume $\supp(\rho)$ is compact (closed and bounded by the Heine--Borel
theorem), with
\[
  \phi(t,\,\vb{x}) \to 0 \quad\text{as}\quad \norm{\vb{x}}\to\infty\,.
\]
Then the unique solution to Poisson's equation is
\begin{equation}\label{eq:poisson-solution}
  \eqbox{\phi(t,\,\vb{x})
    = -\grav\!\int_{\Rn{3}}\!\dd^3 x'\;
      \frac{\rho(t,\,\vb{x}')}{\norm{\vb{x} - \vb{x}'}}}\,.
\end{equation}
This gives the ``gravitostatic'' formulation of Newtonian gravity:
symbol for symbol, it mirrors electrostatics.  Just as in
electrostatics, the field equation, potential, and integral solution
form a self-contained package.  The key difference, as always, is
that $\rho \geq 0$---gravity has only one sign of ``charge.''

% ──────────────────────────────────────────────────────────────
\subsection{Adding dynamics: routes to general relativity}

How do we add dynamics to the gravitational field?  At present, the
gravitational field in our formulation is static: the Poisson equation
determines $\phi$ from the instantaneous mass distribution.  But what
happens when the sources move?  Several approaches suggest
themselves---with very different outcomes:

\begin{center}
\begin{tikzpicture}[
  start/.style={rectangle, rounded corners=4pt, draw=cgblue, fill=cgblue!12,
    minimum width=4cm, minimum height=2em,
    font=\sf\small, text=spacecadet, inner sep=6pt},
  route/.style={rectangle, rounded corners=3pt, draw=spacecadet!40,
    fill=isabelline, minimum width=3.5cm, minimum height=1.8em,
    font=\small, text=spacecadet, inner sep=5pt, text width=3.3cm,
    align=center},
  outcome/.style={font=\small\sf},
  fail/.style={outcome, text=red!70!black},
  win/.style={outcome, text=munsell!80!black},
  arr/.style={-{Stealth[length=5pt]}, thick, spacecadet!60},
]
  % Starting point
  \node[start] (N) at (0,0) {Newtonian gravity: scalar $\phi$};
  \node[font=\sf\small\itshape, text=spacecadet!70, below=2pt of N]
    {How to make it dynamical?};

  % Four routes
  \node[route] (r0) at (-5.0,-2.8) {(0)~Instantaneous propagation};
  \node[route] (r1) at (-1.7,-2.8) {(1)~Scalar wave equation for $\phi$};
  \node[route] (r2) at (1.7,-2.8) {(2)~Promote $\phi$ to a 4-vector};
  \node[route] (r3) at (5.0,-2.8)
    {(3)~Promote $\phi$ to a symmetric tensor};

  % Arrows from start
  \draw[arr] (N.south) -- (r0.north);
  \draw[arr] (N.south) -- (r1.north);
  \draw[arr] (N.south) -- (r2.north);
  \draw[arr] (N.south) -- (r3.north);

  % Outcomes
  \node[fail, below=6pt of r0, text width=2.8cm, align=center]
    {Violates causality\\$\times$};
  \node[fail, below=6pt of r1, text width=2.8cm, align=center]
    {No light bending\\$\times$};
  \node[fail, below=6pt of r2, text width=3cm, align=center]
    {Negative-energy waves; perihelion\\$\times$};
  \node[win, below=6pt of r3, text width=3cm, align=center]
    (gr) {Linearised Einstein eqs.\\$\Downarrow$\\
    \textbf{General relativity}};

  % Highlight the winning route
  \draw[thick, munsell, rounded corners=3pt]
    ([shift={(-3pt,-3pt)}]r3.south west)
    rectangle ([shift={(3pt,3pt)}]r3.north east);
\end{tikzpicture}
\end{center}

See Misner, Thorne \& Wheeler~\cite{MTW73}, Chapter~7, for a
detailed and beautiful discussion of all four routes.

Route~(0) has obvious causality problems: by rearranging masses one
could signal faster than light, violating special relativity.
Route~(1)---promoting the Poisson equation to a scalar wave
equation---is mathematically consistent but predicts no bending of
light by gravity, which has been observed.  Route~(2)---completing
$\phi$ to a 4-vector, as one does for the electromagnetic
potential---suffers from negative-energy wave solutions and fails to
reproduce the observed perihelion precession of Mercury.  Only
route~(3), completing $\phi$ to a symmetric rank-2 tensor, survives
all experimental tests.  Remarkably, the dynamics one is forced to
write down turn out to be precisely the linearised form of Einstein's
field equations, from which the full nonlinear theory of general
relativity can be recovered.

We will \emph{not} follow this route to discover general relativity;
it is not the path Einstein took (at least not as far as the
historical record reveals).  Instead, we will arrive at the theory
from a completely different direction, starting from the equivalence
principle in the next lecture.

\begin{intuition}[Why a tensor?]
  A scalar field $\phi$ has one component---not enough to encode the
  geometry of a 4-dimensional spacetime.  A 4-vector has four components,
  but leads to unphysical negative-energy solutions.  A symmetric
  rank-2 tensor $g_{\mu\nu}$ has ten independent components in four
  dimensions---exactly the right number to describe a metric on spacetime.
  This is the gravitational field of general relativity.
\end{intuition}

% ──────────────────────────────────────────────────────────────
\subsection{Example: Lagrange points in a two-body system}

The Newtonian gravitational potential for a system of two bodies
in the \emph{co-rotating frame} illustrates the interplay between
gravity and inertial (centrifugal/Coriolis) forces.
The \textbf{effective potential} is
\begin{equation}\label{eq:Phi-eff-cr3bp}
  \eqbox{\Phi_{\mathrm{eff}}(x,y)
    = -\frac{1-\mu}{r_1} - \frac{\mu}{r_2}
      - \tfrac{1}{2}(x^2 + y^2)}\,,
\end{equation}
where $\mu = M_2/(M_1+M_2)$ is the mass ratio, and $r_{1,2}$
are the distances to the two masses.
This potential has five critical points---the \textbf{Lagrange
points} $L_1$--$L_5$.  The collinear points $L_1$, $L_2$, $L_3$
are saddle points (unstable), while $L_4$ and $L_5$ form
equilateral triangles with the two masses and are linearly stable
for $\mu < \mu_{\mathrm{crit}} \approx 0.0385$.

\begin{figure}[htbp]
  \centering
  % fig_lec01_lagrange.tex — Effective potential + trajectory near L4
% Data: generated by scripts/sim_lec01.jl (μ = 0.01)

\begin{center}
\begin{tikzpicture}
\begin{axis}[
    grplotwide,
    xlabel={$x$},
    ylabel={$y$},
    title={Effective potential $\Phi_{\mathrm{eff}}$ in the co-rotating frame},
    xmin=-1.5, xmax=2.0,
    ymin=-1.5, ymax=1.5,
    view={0}{90},
    colormap name=grpotential,
    colorbar,
    colorbar style={tick label style={font=\tiny}, width=0.2cm},
    point meta min=-3.5,
    point meta max=-1.2,
    unbounded coords=jump,
]
    % Potential as pseudo-colour
    \addplot3[
        surf,
        shader=interp,
        mesh/rows=60,
        opacity=0.9,
    ] table {data/lec01_potential.dat};

    % Lagrange points
    \addplot[only marks, mark=diamond*, mark size=3.5pt, banana,
             draw=spacecadet, thick] table[x index=0, y index=1]
             {data/lec01_lagrange_points.dat};

    % Mass positions
    \addplot[only marks, mark=*, mark size=4pt, spacecadet,
             fill=spacecadet] table[x index=0, y index=1] {data/lec01_masses.dat};

    % L-point labels
    \node[font=\sf\scriptsize, text=spacecadet, fill=white, fill opacity=0.7,
          text opacity=1, inner sep=1pt, rounded corners=1pt]
        at (axis cs:0.85,0.12) {$L_1$};
    \node[font=\sf\scriptsize, text=spacecadet, fill=white, fill opacity=0.7,
          text opacity=1, inner sep=1pt, rounded corners=1pt]
        at (axis cs:1.15,0.12) {$L_2$};
    \node[font=\sf\scriptsize, text=spacecadet, fill=white, fill opacity=0.7,
          text opacity=1, inner sep=1pt, rounded corners=1pt]
        at (axis cs:-1.00,0.12) {$L_3$};
    \node[font=\sf\scriptsize, text=spacecadet, fill=white, fill opacity=0.7,
          text opacity=1, inner sep=1pt, rounded corners=1pt]
        at (axis cs:0.49,0.97) {$L_4$};
    \node[font=\sf\scriptsize, text=spacecadet, fill=white, fill opacity=0.7,
          text opacity=1, inner sep=1pt, rounded corners=1pt]
        at (axis cs:0.49,-0.97) {$L_5$};

    % Mass labels
    \node[font=\sf\scriptsize, text=white]
        at (axis cs:-0.01,-0.15) {$M_1$};
    \node[font=\sf\scriptsize, text=white]
        at (axis cs:0.99,-0.15) {$M_2$};
\end{axis}
\end{tikzpicture}
\end{center}

\begin{center}
\begin{tikzpicture}
\begin{axis}[
    grplot,
    width=7cm, height=7cm,
    xlabel={$x$},
    ylabel={$y$},
    title={Tadpole orbit near $L_4$},
    axis equal,
]
    % Trajectory
    \addplot[cgblue, line width=0.8pt, no markers]
        table[x index=0, y index=1] {data/lec01_trajectory.dat};

    % L4 point
    \addplot[only marks, mark=diamond*, mark size=4pt, banana,
             draw=spacecadet, thick]
        coordinates {(0.49, 0.866025)};
    \node[font=\sf\footnotesize, text=spacecadet, anchor=south west]
        at (axis cs:0.51, 0.88) {$L_4$};

    % Mass positions (faint)
    \addplot[only marks, mark=*, mark size=3pt, spacecadet!40]
        coordinates {(-0.01, 0) (0.99, 0)};
\end{axis}
\end{tikzpicture}
\end{center}

  \caption{Top: pseudo-colour plot of the effective potential
    $\Phi_{\mathrm{eff}}$ in the co-rotating frame ($\mu = 0.01$),
    with the five Lagrange points marked.
    Bottom: tadpole orbit of a test particle released near~$L_4$
    with a small perturbation, exhibiting the characteristic
    epicyclic motion.}\label{fig:lagrange-points}
\end{figure}
