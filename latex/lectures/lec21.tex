%!TEX root = ../GeneralRelativity.tex
% ──────────────────────────────────────────────────────────────
%  Lecture 21 — Timelike geodesics in Schwarzschild: effective
%              potential and perihelion precession
% ──────────────────────────────────────────────────────────────

\subsection{Constants of motion and the effective potential}%
\label{sec:schw-effective-potential}

We study timelike geodesics ($\kappa = 1$) in the exterior
Schwarzschild spacetime ($r > 2M$), parameterised by proper
time~$\tau$.  Setting $\theta = \pi/2$ (no loss of generality by
\S\ref{sec:equatorial-plane}), the metric
norm~$g_{ab}\, u^a u^b = -1$ gives
\begin{equation}\label{eq:norm-constraint}
  -\Bigl(1 - \frac{2M}{r}\Bigr)\dot t^2
  + \Bigl(1 - \frac{2M}{r}\Bigr)^{-1}\dot r^2
  + r^2\, \dot\phi^2 = -\kappa\,,
\end{equation}
where $\kappa = 1$ for timelike and $\kappa = 0$ for null
geodesics, and dots denote $d/d\tau$.

The two Killing vectors $\xi^a = (\partial/\partial t)^a$ and
$\psi^a = (\partial/\partial\phi)^a$ yield, via
Proposition~\ref{prop:killing-constant}, the conserved quantities:
\begin{align}
  E &= \Bigl(1 - \frac{2M}{r}\Bigr)\,\dot t\,,
    \label{eq:energy-constant}
    \tag{i}\\[4pt]
  L &= r^2\, \dot\phi\,.
    \label{eq:angular-momentum-constant}
    \tag{ii}
\end{align}
For a timelike particle far from the source, $E$ reduces to the
special-relativistic energy per unit rest mass as measured by a
static observer; $L$ is the angular momentum per unit rest mass.
The constancy of~$L$ is the relativistic analogue of Kepler's
second law: equal areas are swept out in equal proper times.

Substituting~\eqref{eq:energy-constant}
and~\eqref{eq:angular-momentum-constant}
into~\eqref{eq:norm-constraint}:
\begin{equation}\label{eq:radial-energy}
  \eqbox{\tfrac{1}{2}\,\dot r^2 + V(r) = \tfrac{1}{2}\, E^2}
\end{equation}
where the \textbf{effective potential} is
\begin{equation}\label{eq:effective-potential}
  \eqbox{V(r) = \tfrac{1}{2}\,\kappa
    - \frac{\kappa\, M}{r}
    + \frac{L^2}{2r^2}
    - \frac{M L^2}{r^3}}
\end{equation}

\begin{intuition}[One-dimensional mechanics]
  Equation~\eqref{eq:radial-energy} has the structure of a
  unit-mass particle with ``kinetic energy'' $\frac{1}{2}\dot r^2$
  moving in the potential~$V(r)$ with ``total energy''
  $\frac{1}{2}E^2$.  We can therefore import the full machinery of
  one-dimensional classical mechanics---turning points, potential
  wells, stability analysis---to understand the radial motion.
  The angular motion is then recovered
  from~\eqref{eq:angular-momentum-constant}.
\end{intuition}

The first three terms of~\eqref{eq:effective-potential} are familiar
from Newtonian orbital mechanics ($\frac{1}{2}\kappa$ is a constant,
$-\kappa M/r$ is the gravitational potential, $L^2/2r^2$ is the
centrifugal barrier).  The \emph{new} term is
\begin{equation}\label{eq:gr-correction}
  -\frac{ML^2}{r^3}\,,
\end{equation}
which is the relativistic correction.  It is negligible for large
$r$, but \emph{dominates over the centrifugal barrier for small~$r$},
dragging the potential to $-\infty$ and qualitatively changing the
orbital dynamics.

% ──────────────────────────────────────────────────────────────
\subsection{Circular orbits and their stability}%
\label{sec:circular-orbits}

Circular orbits correspond to extrema of~$V(r)$ (where
$\dot r = 0$).  Setting $dV/dr = 0$ for $\kappa = 1$:
\begin{equation}\label{eq:dVdr}
  0 = \frac{dV}{dr}
    = \frac{1}{r^4}\bigl(Mr^2 - L^2 r + 3ML^2\bigr)\,.
\end{equation}
The roots are
\begin{equation}\label{eq:circular-radii}
  r_\pm = \frac{L^2 \pm \sqrt{L^4 - 12\, L^2 M^2}}{2M}\,.
\end{equation}

\subsubsection{Small angular momentum: $L^2 < 12M^2$}

The discriminant is negative: \textbf{no extrema exist}.  The
potential decreases monotonically for small~$r$ (due to the
$-ML^2/r^3$ term), so any particle spirals inexorably inward.
There are no stable orbits.

\subsubsection{Large angular momentum: $L^2 > 12M^2$}

Two extrema exist: $r_+$ is a \textbf{minimum}
(stable circular orbit) and $r_-$ is a \textbf{maximum}
(unstable circular orbit).

\begin{keyresult}[Innermost stable circular orbit]
  Since $r_+ > 6M$ always, there are \textbf{no stable circular
  orbits} at radii $r \leq 6M$.  The \textbf{innermost stable
  circular orbit} (ISCO) is at $r = 6M$ (achieved in the limit
  $L^2 \to 12M^2$).  The unstable circular orbits lie in the range
  $3M < r_- < 6M$.
\end{keyresult}

In the Newtonian limit $L \gg M$, the stable radius $r_+ \approx
L^2/M$ reproduces the Newtonian formula for the circular orbital
radius.

\begin{exercise}\label{ex:circular-energy}
  Show that the energy of a particle in a circular orbit at
  radius~$R$ is
  \[
    E(R) = \frac{R - 2M}{\sqrt{R}\,\sqrt{R - 3M}}\,.
  \]
  Verify that $E > 1$ for $R < 4M$ (particles in unstable circular
  orbits between $3M$ and $4M$ can escape to infinity if
  perturbed outward).
\end{exercise}

% ──────────────────────────────────────────────────────────────
\subsection{Perihelion precession}%
\label{sec:perihelion-precession}

Consider a particle in a slightly perturbed stable circular orbit
at radius~$r_+$.  For sufficiently small displacements, the
potential is effectively quadratic and the particle executes simple
harmonic motion in~$r$ with radial frequency (exercise):
\begin{equation}\label{eq:omega-r}
  \omega_r^2 = \frac{d^2V}{dr^2}\bigg|_{r_+}
    = \frac{M(r_+ - 6M)}{r_+^3\,(r_+ - 3M)}\,.
\end{equation}
(Note: the ``time'' here is the proper time~$\tau$ of the
orbiting particle.)

The angular frequency of the circular orbit is found from
$\omega_\phi = \dot\phi = L/r_+^2$; eliminating~$L$ using the
circular orbit condition:
\begin{equation}\label{eq:omega-phi}
  \omega_\phi^2 = \frac{M}{r_+^2\,(r_+ - 3M)}\,.
\end{equation}

In the Newtonian limit ($r_+ \gg M$), both frequencies approach
$\omega_r^2 \approx \omega_\phi^2 \approx M/r_+^3$: the
oscillations are \textbf{commensurate} and the orbit \emph{closes}
(as it must for a $1/r$ potential, by Bertrand's theorem).

In general relativity, however, $\omega_r \neq \omega_\phi$.  The
orbit does \emph{not} close: the perihelion (point of closest
approach) shifts from one orbit to the next.

\begin{keyresult}[Precession of the perihelion]
  The precession rate per orbit is
  \begin{equation}\label{eq:precession-rate}
    \omega_p = \omega_\phi - \omega_r
      = -\Bigl(\sqrt{1 - \frac{6M}{r_+}} - 1\Bigr)\,\omega_\phi\,.
  \end{equation}
  For $r_+ \gg M$:
  \begin{equation}\label{eq:precession-approx}
    \eqbox{\omega_p \approx \frac{3\, M^{3/2}}{r_+^{5/2}}}
  \end{equation}
  (radians per orbit in proper time).
\end{keyresult}

\begin{historical}[Mercury's perihelion: a triumph of general
  relativity]
  The precession of Mercury's perihelion had been a long-standing
  puzzle in celestial mechanics.  After accounting for the
  gravitational perturbations of the other planets, an anomalous
  precession of approximately $43''$ (seconds of arc) per century
  remained unexplained by Newtonian gravity.  When Einstein
  completed his general theory of relativity in November~1915, one
  of his first calculations was the perihelion precession of
  Mercury.  The result---$43''$ per century---matched the observed
  anomaly precisely, without any adjustable parameters.  Einstein
  later wrote that this calculation gave him ``palpitations of the
  heart.''  It remains one of the most celebrated confirmations of
  general relativity.
\end{historical}

\medskip
In the final lecture, we will study \textbf{null geodesics} in
the Schwarzschild spacetime---the trajectories of light rays---and
derive the deflection of light by a massive body, another
celebrated prediction of Einstein's theory.
