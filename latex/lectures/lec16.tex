%!TEX root = ../GeneralRelativity.tex
% ──────────────────────────────────────────────────────────────
%  Lecture 16 — Dynamics of a homogeneous and isotropic universe
% ──────────────────────────────────────────────────────────────

\subsection{Dynamics of a homogeneous and isotropic universe}%
\label{sec:flrw-dynamics}

In the previous section we described the \emph{kinematics} of a
homogeneous and isotropic universe: we derived the FLRW
metric~\eqref{eq:flrw} and reduced the geometric degrees of freedom
to a single unknown function, the scale factor~$a(\tau)$.  Now we
turn to the \emph{dynamics}: how does such a universe evolve from
its inception to its final moments---if there are any?

To answer this question we need two ingredients: the left-hand side
of Einstein's field equations (the Einstein tensor, which we can
compute from the FLRW metric) and the right-hand side (the
stress-energy tensor, which models the matter content of the
universe).  We begin with the latter.

% ──────────────────────────────────────────────────────────────
\subsubsection{Stress-energy tensor for a homogeneous isotropic
  universe}%
\label{sec:flrw-stress-energy}

In our highly idealised model, we treat galaxies as grains of dust:
point particles with negligible random velocities, moving on the
worldlines of the isotropic observers.  Since the fluctuations
in galaxy velocities are small compared to the speed of light,
pressure is neglected, and the stress-energy tensor of the matter
content is simply that of pressureless dust:
\begin{equation}\label{eq:dust-tab}
  T_{ab}^{(\text{dust})} = \rho\, u_a\, u_b\,,
\end{equation}
where $\rho$ is the average mass-energy density (one ``smears out''
the galaxies into a homogeneous soup) and $u^a$ is the
four-velocity of the isotropic observers.

\begin{intuition}[Modelling the universe as dust]
  This model treats the entire visible matter content of the
  universe---hundreds of billions of galaxies, each containing
  hundreds of billions of stars---as a uniform pressureless gas.
  The approximation is justified on cosmological scales: the
  peculiar velocities of galaxies (a few hundred km/s) are tiny
  compared to the speed of light, and the gravitational interaction
  between galaxies is negligible compared to the overall expansion
  dynamics.
\end{intuition}

However, the visible matter is not the only contribution to the
energy content of the universe.  There is an experimentally
determined background of thermal radiation pervading the cosmos at a
temperature of approximately $3\,\text{K}$: the \textbf{cosmic
microwave background} (CMB).  We model this radiation as a perfect
fluid with the equation of state
\begin{equation}\label{eq:radiation-eos}
  p_{\text{rad}} = \tfrac{1}{3}\,\rho_{\text{rad}}\,.
\end{equation}
(Deriving this from statistical mechanics is a somewhat advanced
exercise; the factor of $\frac{1}{3}$ reflects the fact that
radiation exerts equal pressure in all three spatial directions.)

The total stress-energy tensor, incorporating both the galaxy
(dust) content and the radiation, takes the form of a
\textbf{perfect fluid}:
\begin{equation}\label{eq:perfect-fluid-cosmo}
  \eqbox{T_{ab} = \rho\, u_a\, u_b
    + p\,(g_{ab} + u_a\, u_b)}
\end{equation}
where $\rho$ and $p$ are adjusted to include all contributions
(visible matter, radiation, and any other fields).

\begin{exercise}\label{ex:most-general-tab}
  Show that~\eqref{eq:perfect-fluid-cosmo} is the most general
  stress-energy tensor consistent with homogeneity and isotropy.
  \emph{Hint}: in the rest frame of the isotropic observers,
  $T_{ab}$ must be invariant under spatial rotations.  What form
  can a symmetric $(0,2)$ tensor take if all spatial directions
  are equivalent?
\end{exercise}

% ──────────────────────────────────────────────────────────────
\subsubsection{The cosmological constant}%
\label{sec:cosmological-constant}

We add one further term to Einstein's field equations: the
\textbf{cosmological constant}~$\Lambda$.  The modified field
equations read
\begin{equation}\label{eq:einstein-lambda}
  \eqbox{\Ein_{ab} + \Lambda\, g_{ab} = 8\pi\, T_{ab}}
\end{equation}
This term was introduced by Einstein himself (for very different
reasons---he sought a static universe), and for decades it was hoped
that $\Lambda = 0$.

\begin{historical}[The cosmological constant]
  Einstein introduced $\Lambda$ in 1917 to obtain a static
  cosmological solution, which he believed was necessary on
  philosophical grounds.  After Hubble's discovery of the expansion
  of the universe in 1929, Einstein reportedly called the
  cosmological constant his ``greatest blunder.''  However,
  observations since the late 1990s---first from Type~Ia supernovae,
  then from the CMB and large-scale structure surveys---have
  established that $\Lambda > 0$.  A non-zero cosmological constant
  (or, more generally, ``dark energy'') is now an essential
  ingredient of the standard cosmological model.  The cosmological
  constant acts effectively as a \emph{negative pressure} term,
  driving the accelerated expansion of the universe observed today.
\end{historical}

% ──────────────────────────────────────────────────────────────
\subsubsection{Reduction to two independent equations}%
\label{sec:two-equations}

In principle, Einstein's field equations~\eqref{eq:einstein-lambda}
constitute ten simultaneous coupled nonlinear partial differential
equations for the ten independent components of the metric---a
system of monstrous complexity, far exceeding anything one typically
encounters in other branches of physics.  There is no Green's
function, no superposition principle; nothing is going to save us in
general.

However, one can either use perturbation theory (as we did for
linearised gravity) or exploit symmetry to reduce the number of
equations.  In the present case, homogeneity and isotropy achieve a
dramatic simplification.

\begin{keyresult}[Only two independent equations]
  Of the ten components of Einstein's field
  equations~\eqref{eq:einstein-lambda}, only \textbf{two} are
  independent for a homogeneous and isotropic spacetime.
\end{keyresult}

The argument proceeds as follows.  First, note that the vectors
$\Ein^{ab}\, u_b$ and $T^{ab}\, u_b$ can have no spatial
components (by isotropy---a non-vanishing spatial component would
define a preferred direction).  Similarly, when restricted to a
spatial slice~$\Sigma_t$, the same eigenvalue argument used
for~$\Riem^{(3)}_{ab}{}^{cd}$ in \S\ref{sec:eigenvalue-argument}
implies that the spatial part of the Einstein tensor~$\Ein^{(3)}_{ab}$
is proportional to the identity (otherwise one could construct a
preferred spatial vector).  Consequently:
\begin{itemize}
  \item The off-diagonal (space--time) components
    of~\eqref{eq:einstein-lambda} vanish identically.
  \item The diagonal spatial components are all \emph{equal}
    (by the eigenvalue/isotropy argument).
  \item By homogeneity, the space--time components also vanish.
\end{itemize}

We are left with exactly two independent equations, which we write
as:
\begin{align}
  \Ein_{\tau\tau} + \Lambda\, g_{\tau\tau} &= 8\pi\, \rho\,,
    \label{eq:friedmann-tt}\\[4pt]
  \Ein_{**} + \Lambda\, g_{**} &= 8\pi\, p\,,
    \label{eq:friedmann-ss}
\end{align}
where $\Ein_{\tau\tau} = \Ein_{ab}\, u^a u^b$ is the time--time
component and $\Ein_{**} = \Ein_{ab}\, s^a s^b$ denotes \emph{any}
spatial diagonal component (contracted with a unit spacelike
vector~$s^a$ orthogonal to~$u^a$, with $s^a s_a = 1$); all spatial
components give the same equation.

% ──────────────────────────────────────────────────────────────
\subsubsection{The FLRW metric in unified form}%
\label{sec:flrw-unified}

To carry out the explicit calculation of the Christoffel symbols,
Riemann tensor, and Einstein tensor from the FLRW metric, we need a
coordinate representation.  Basis-free reasoning is wonderful for
deducing logical consequences of geometric statements, but it is
hopeless for doing any real calculation---for that, we must choose
coordinates and compute.

It is convenient to write the three spatial
metrics~\eqref{eq:metric-S3}/\eqref{eq:metric-flat}/\eqref{eq:metric-H3}
in a single unified form using spherical coordinates $(r, \theta,
\phi)$.  The FLRW metric becomes:
\begin{equation}\label{eq:flrw-unified}
  \eqbox{ds^2 = -d\tau^2
    + a^2(\tau)\biggl[
      \frac{dr^2}{1 - k\,r^2}
      + r^2\bigl(d\theta^2
      + \sin^2\!\theta\, d\phi^2\bigr)
    \biggr]}
\end{equation}
where the \textbf{curvature parameter}
\begin{equation}\label{eq:k-values}
  k =
  \begin{cases}
    +1 & \text{(three-sphere, closed universe)}\,,\\
    \phantom{+}0 & \text{(flat space, open universe)}\,,\\
    -1 & \text{(hyperboloid, open universe)}\,.
  \end{cases}
\end{equation}

\begin{exercise}\label{ex:flrw-unified}
  Verify that~\eqref{eq:flrw-unified} with $k = +1$ reduces to the
  $S^3$ metric~\eqref{eq:metric-S3} under the substitution
  $r = \sin\psi$, and that $k = -1$ gives the $H^3$
  metric~\eqref{eq:metric-H3} under $r = \sinh\psi$.
\end{exercise}

\begin{intuition}[The road ahead]
  We now have the metric~\eqref{eq:flrw-unified} and the
  stress-energy tensor~\eqref{eq:perfect-fluid-cosmo}, and we know
  that Einstein's equations reduce to just two independent
  equations~\eqref{eq:friedmann-tt}--\eqref{eq:friedmann-ss}.  What
  remains is an explicit but mechanical calculation: compute the
  Christoffel symbols from~\eqref{eq:flrw-unified}, build the
  Riemann and Ricci tensors, form the Einstein tensor, substitute
  into~\eqref{eq:friedmann-tt}--\eqref{eq:friedmann-ss}, and arrive
  at ordinary differential equations for~$a(\tau)$.  These are the
  \textbf{Friedmann equations}, and---remarkably---they can be
  solved.  That calculation is the subject of the next lecture.
\end{intuition}
