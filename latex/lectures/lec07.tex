%!TEX root = ../GeneralRelativity.tex
% ──────────────────────────────────────────────────────────────
%  Lecture 7 — Derivative operators and parallel transport (Part I)
% ──────────────────────────────────────────────────────────────

\section{Derivative operators and parallel transport~(I)}%
\label{sec:derivative-operators}

In the previous lectures we developed the algebra of tensors on a
manifold~$\M$.  We can add tensors, take tensor products, and
contract indices---but we cannot yet \emph{differentiate} tensor
fields.  In $\Rn{n}$ one differentiates by comparing values of a
tensor at nearby points; on a general manifold this comparison
requires extra structure.  That structure is a \textbf{derivative
operator} (or \textbf{affine connection}).

\begin{notation}
  We write $\mathscr{X}(\M)$ for the space of smooth vector
  fields on~$\M$, and $\mathscr{F}(\M)$ for the space of smooth
  functions $f\colon\M\to\R$.
\end{notation}

% ──────────────────────────────────────────────────────────────
\subsection{Affine connections: definition}\label{sec:affine-connection}

\begin{definition}[Affine connection / covariant derivative]%
\label{def:affine-connection}
  An \textbf{affine connection} (or \textbf{covariant derivative
  operator}) on a manifold~$\M$ is a map
  \[
    \covd\colon \mathscr{X}(\M)\times\mathscr{X}(\M)
      \to \mathscr{X}(\M)\,,
    \qquad (X,Y)\mapsto \covd_X Y\,,
  \]
  which we immediately extend to a map
  $\covd\colon \mathscr{X}(\M)\times\ttype{k}{l}
  \to\ttype{k}{l}$, satisfying axioms~(0)--(5) below.
\end{definition}

% ──────────────────────────────────────────────────────────────
\subsection{The axioms}\label{sec:connection-axioms}

\subsubsection*{Axiom~(0): Additivity in the first argument}

For all $X,Y\in\mathscr{X}(\M)$ and $Z\in\ttype{k}{l}$:
\begin{equation}\label{eq:axiom0}
  \covd_{X+Y} Z = \covd_X Z + \covd_Y Z\,.
\end{equation}

\subsubsection*{Axiom~(0$'$): $\mathscr{F}(\M)$-linearity in
the first argument}

For all $f\in\mathscr{F}(\M)$, $X\in\mathscr{X}(\M)$, and
$Y\in\ttype{k}{l}$:
\begin{equation}\label{eq:axiom0prime}
  \covd_{fX} Y = f\,\covd_X Y\,.
\end{equation}

\begin{remark}
  Axioms~(0) and~(0$'$) together say that $\covd_X$ depends
  $\mathscr{F}(\M)$-linearly on the ``direction''~$X$.  In
  abstract index notation we write
  $\covd_X \equiv X^a\covd_a$, so that the derivative in
  direction~$X$ is determined by the components of~$X$ and the
  operator~$\covd_a$.
\end{remark}

\subsubsection*{Axiom~(1): Linearity}

For all $A,B\in\ttype{k}{l}$:
\begin{equation}\label{eq:axiom1}
  \covd_X(A + B) = \covd_X A + \covd_X B\,.
\end{equation}
In abstract index notation:
\[
  \covd_c\bigl(
    A^{a_1\cdots a_k}{}_{b_1\cdots b_l}
    + B^{a_1\cdots a_k}{}_{b_1\cdots b_l}
  \bigr)
  = \covd_c A^{a_1\cdots a_k}{}_{b_1\cdots b_l}
  + \covd_c B^{a_1\cdots a_k}{}_{b_1\cdots b_l}\,.
\]

\subsubsection*{Axiom~(2): Leibniz rule}

For all $A\in\ttype{k}{l}$ and $B\in\ttype{k'}{l'}$:
\begin{equation}\label{eq:axiom2}
  \covd_X(A\tp B) = (\covd_X A)\tp B + A\tp(\covd_X B)\,.
\end{equation}
In abstract index notation:
\begin{multline}\label{eq:axiom2-ain}
  \covd_e\bigl(
    A^{a_1\cdots a_k}{}_{b_1\cdots b_l}\,
    B^{c_1\cdots c_{k'}}{}_{d_1\cdots d_{l'}}
  \bigr) \\
  = (\covd_e A^{a_1\cdots a_k}{}_{b_1\cdots b_l})\,
    B^{c_1\cdots c_{k'}}{}_{d_1\cdots d_{l'}}
  + A^{a_1\cdots a_k}{}_{b_1\cdots b_l}\,
    (\covd_e B^{c_1\cdots c_{k'}}{}_{d_1\cdots d_{l'}})\,.
\end{multline}

\subsubsection*{Axiom~(3): Commutativity with contraction}

For all $A\in\ttype{k}{l}$ (with $k\geq 1$, $l\geq 1$):
\begin{equation}\label{eq:axiom3}
  \covd_X(C_{j,j'}\, A) = C_{j,j'}(\covd_X A)\,.
\end{equation}
In abstract index notation, contraction is implemented by
repeating an index, and the axiom states that $\covd_d$
commutes with this operation.

\subsubsection*{Axiom~(4): Reduction to directional derivative
on scalars}

For all $f\in\mathscr{F}(\M)$:
\begin{equation}\label{eq:axiom4}
  \eqbox{\covd_X(f) = X(f)}\,.
\end{equation}
In abstract index notation: $X(f) = X^a\covd_a f$.

\begin{remark}
  Axiom~(4) says that on scalar functions the covariant
  derivative reduces to the ordinary directional derivative.
  No extra data is needed to differentiate scalars.
\end{remark}

\subsubsection*{Axiom~(5): Torsion-free}

For all $X,Y\in\mathscr{X}(\M)$:
\begin{equation}\label{eq:axiom5-torsionfree}
  \eqbox{\covd_X Y - \covd_Y X - [X,Y] = 0}\,.
\end{equation}
Equivalently, in abstract index notation: for all
$f\in\mathscr{F}(\M)$,
\begin{equation}\label{eq:torsion-free-ain}
  \covd_a\covd_b f = \covd_b\covd_a f\,.
\end{equation}

% ──────────────────────────────────────────────────────────────
\subsection{Simple consequences}\label{sec:connection-consequences}

Two useful identities follow directly from the axioms.

\begin{theorem}[Leibniz rule on products]\label{thm:leibniz-fY}
  For all $X\in\mathscr{X}(\M)$,
  $Y\in\mathscr{X}(\M)$, and $f\in\mathscr{F}(\M)$:
  \begin{equation}\label{eq:leibniz-fY}
    \covd_X(fY) = X(f)\,Y + f\,\covd_X Y\,.
  \end{equation}
\end{theorem}

\begin{proof}
  View $fY$ as the tensor product
  $f\tp Y$ (with $f\in\ttype{0}{0}$, $Y\in\ttype{1}{0}$).
  By axiom~(2):
  \[
    \covd_X(f\tp Y) = (\covd_X f)\tp Y + f\tp(\covd_X Y)\,.
  \]
  By axiom~(4), $\covd_X f = X(f)$.  The result follows.
\end{proof}

\begin{theorem}[Lie bracket via $\covd$]\label{thm:lie-bracket-covd}
  For all $v,w\in\mathscr{X}(\M)$:
  \begin{equation}\label{eq:lie-bracket-covd}
    \eqbox{[v,w]^b = v^a\covd_a w^b - w^a\covd_a v^b}\,.
  \end{equation}
\end{theorem}

\begin{proof}
  By axiom~(5), $\covd_v w - \covd_w v = [v,w]$.  Writing
  $\covd_v \equiv v^a\covd_a$ and $\covd_w \equiv w^a\covd_a$
  gives~\eqref{eq:lie-bracket-covd}.
\end{proof}

\begin{remark}
  Equation~\eqref{eq:lie-bracket-covd} shows that the Lie
  bracket, which is defined purely in terms of the manifold
  structure, can be computed using \emph{any} torsion-free
  connection.  The result is independent of the choice of~$\covd$.
\end{remark}

% ──────────────────────────────────────────────────────────────
\subsection{Existence and non-uniqueness}%
\label{sec:existence-nonuniqueness}

There are \emph{many} derivative operators satisfying
axioms~(0)--(5).  We first exhibit the simplest one, then
characterise the freedom in choosing~$\covd$.

\subsubsection{The ordinary derivative}\label{sec:ordinary-deriv}

Let $\psi$ be a chart on~$\M$ with coordinate
functions~$(x^1,\dots,x^n)$.  For a tensor
$T\in\ttype{k}{l}$ with components
$T^{\mu_1\cdots\mu_k}{}_{\nu_1\cdots\nu_l}$ in the coordinate
basis, define $\partial_a$ by letting it act via the partial
derivative $\partial/\partial x^\sigma$ on these components.

\begin{remark}
  The operator $\partial_a$ satisfies axioms~(0)--(5), but it is
  \textbf{chart-dependent}: it is defined only with respect to
  the chart~$\psi$.  A different chart generally gives a
  different operator~$\partial_a$.  A true covariant derivative
  is chart-independent.
\end{remark}

% ──────────────────────────────────────────────────────────────
\subsubsection{Difference of two derivative operators}%
\label{sec:difference-operators}

Suppose $\covd_a$ and $\tilde\covd_a$ are two derivative
operators both satisfying axioms~(0)--(5).

\medskip\noindent
\textbf{Step~1: On scalars.}\;
For all $f\in\mathscr{F}(\M)$, axiom~(4) gives
\begin{equation}\label{eq:diff-on-scalars}
  (\tilde\covd_a - \covd_a)f
    = \tilde\covd_a f - \covd_a f = 0\,.
\end{equation}

\medskip\noindent
\textbf{Step~2: On covectors.}\;
Consider the map
$(\tilde\covd_a - \covd_a)\colon\ttype{0}{1}\to\ttype{0}{2}$.
Let $f\in\mathscr{F}(\M)$ and $\omega\in\ttype{0}{1}$.  Then
$f\omega_a\in\ttype{0}{1}$, and by the Leibniz
rule~\eqref{eq:axiom2} and~\eqref{eq:diff-on-scalars}:
\begin{equation}\label{eq:diff-covector-flinearity}
  (\tilde\covd_a - \covd_a)(f\omega_b)
    = f\,(\tilde\covd_a - \covd_a)\omega_b\,.
\end{equation}

\begin{theorem}\label{thm:diff-local}
  The quantity $(\tilde\covd_a - \covd_a)\omega_b$
  depends only on the value of $\omega_b$ at the
  point~$p$.
\end{theorem}

\begin{proof}
  Let $\omega'_b\in\ttype{0}{1}$ agree with $\omega_b$ at~$p$,
  i.e.\ $\omega'_b|_p = \omega_b|_p$.  We must show that
  \[
    (\tilde\covd_a - \covd_a)\omega'_c\big|_p
    = (\tilde\covd_a - \covd_a)\omega_c\big|_p\,.
  \]
  Write the difference $\omega'_b - \omega_b$ in terms of smooth
  functions~$f_\alpha$ vanishing at~$p$ and smooth covector
  fields~$\mu_b^{(\alpha)}$:
  \[
    \omega'_b - \omega_b
      = \sum_\alpha f_\alpha\, \mu_b^{(\alpha)}\,,
    \qquad f_\alpha|_p = 0\,.
  \]
  (Such a decomposition exists by the smoothness of
  $\omega' - \omega$ and the fact that it vanishes at~$p$.)
  Applying $(\tilde\covd_a - \covd_a)$ and
  using~\eqref{eq:diff-covector-flinearity}:
  \[
    (\tilde\covd_a - \covd_a)(\omega'_b - \omega_b)\big|_p
    = \sum_\alpha
      f_\alpha|_p\,
      (\tilde\covd_a - \covd_a)\mu_b^{(\alpha)}\big|_p
    = 0\,,
  \]
  since $f_\alpha|_p = 0$.
\end{proof}

% ──────────────────────────────────────────────────────────────
\subsection{The $C$ tensor}\label{sec:C-tensor}

By Theorem~\ref{thm:diff-local}, the map
$(\tilde\covd_a - \covd_a)$ on covectors at~$p$ is a map
\[
  V_p^* \;\longrightarrow\; V_p^*\tp V_p^*\,.
\]
By duality ($V_p^{**}\cong V_p$), this can be interpreted as an
element of $V_p\tp V_p^*\tp V_p^*$.  We write its components as
$C^c{}_{ab}$ and record:
\begin{equation}\label{eq:C-tensor-covector}
  \eqbox{\covd_a\omega_b
    = \tilde\covd_a\omega_b - C^c{}_{ab}\,\omega_c}\,.
\end{equation}

\begin{remark}
  \textbf{Warning:} if one takes $\tilde\covd_a = \partial_a$,
  then $C^c{}_{ab}$ does \emph{not} transform as a tensor field
  of type~$\ttype{1}{2}$ under coordinate changes.  Only the
  \emph{difference} of two genuine covariant derivatives
  transforms tensorially.
\end{remark}

% ──────────────────────────────────────────────────────────────
\subsubsection{Symmetry of $C^c{}_{ab}$}%
\label{sec:C-symmetry}

Let $f\in\mathscr{F}(\M)$ and consider the covector
$\omega_b = \covd_b f = \tilde\covd_b f$ (by axiom~(4) both
operators agree on scalars).  Then
\begin{equation}\label{eq:C-symmetry-derivation}
  \covd_a\covd_b f
    = \tilde\covd_a\tilde\covd_b f
      - C^c{}_{ab}\,\covd_c f\,.
\end{equation}
The left-hand side is symmetric in $a,b$ (torsion-free,
axiom~(5)), and the first term on the right-hand side is
likewise symmetric.  Therefore
\begin{equation}\label{eq:C-symmetric}
  \eqbox{C^c{}_{ab} = C^c{}_{ba}}\,.
\end{equation}

% ──────────────────────────────────────────────────────────────
\subsubsection{Action on vectors}\label{sec:C-on-vectors}

We now determine how $(\tilde\covd_a - \covd_a)$ acts on
vectors $t^b\in\ttype{1}{0}$.  Let $\omega\in\ttype{0}{1}$ and
$t\in\ttype{1}{0}$.  The contraction $\omega_b\,t^b$ is a
scalar, so by axiom~(4):
\begin{equation}\label{eq:diff-scalar-zero}
  (\tilde\covd_a - \covd_a)(\omega_b\, t^b) = 0\,.
\end{equation}
Expanding by the Leibniz rule:
\[
  t^b\,(\tilde\covd_a - \covd_a)\omega_b
  + \omega_b\,(\tilde\covd_a - \covd_a)t^b = 0\,.
\]
Substituting~\eqref{eq:C-tensor-covector}
($(\tilde\covd_a - \covd_a)\omega_b = C^c{}_{ab}\,\omega_c$
from the \emph{opposite} sign convention):
\[
  -t^b\,C^c{}_{ab}\,\omega_c
  + \omega_b\,(\tilde\covd_a - \covd_a)t^b = 0\,.
\]
Relabelling the dummy index $c\to b$ in the first term and
requiring this to hold for all~$\omega_b$, we obtain
\begin{equation}\label{eq:C-tensor-vector}
  \eqbox{\covd_a t^b
    = \tilde\covd_a t^b + C^b{}_{ac}\, t^c}\,.
\end{equation}

\begin{keyresult}[Difference of connections on vectors]
  Two torsion-free derivative operators $\covd_a$ and
  $\tilde\covd_a$ are related on vector fields by
  \[
    \covd_a t^b = \tilde\covd_a t^b + C^b{}_{ac}\, t^c\,,
  \]
  where $C^c{}_{ab} = C^c{}_{ba}$ is symmetric in its lower
  indices.
\end{keyresult}

% ──────────────────────────────────────────────────────────────
\subsection{Extension to general tensors}\label{sec:C-general}

The Leibniz rule and axiom~(4) determine the action of
$(\tilde\covd_a - \covd_a)$ on \emph{all} tensor fields.  For a
general $T\in\ttype{k}{l}$:
\begin{equation}\label{eq:C-general-tensor}
  \eqbox{
  \covd_a T^{b_1\cdots b_k}{}_{c_1\cdots c_l}
  = \tilde\covd_a T^{b_1\cdots b_k}{}_{c_1\cdots c_l}
  + \sum_{j=1}^{k}
    C^{b_j}{}_{ad}\,
    T^{b_1\cdots d\cdots b_k}{}_{c_1\cdots c_l}
  - \sum_{j=1}^{l}
    C^d{}_{ac_j}\,
    T^{b_1\cdots b_k}{}_{c_1\cdots d\cdots c_l}
  }
\end{equation}
In the first sum, $d$ replaces $b_j$; in the second sum, $d$
replaces~$c_j$.

\begin{remark}
  The rule is: one $+C$ term for each upper index, one $-C$
  term for each lower index.  The action of $\tilde\covd_a -
  \covd_a$ on an arbitrary tensor of type~$(k,l)$ is thus
  completely determined by the single object
  $C^c{}_{ab}\in V_p\tp V_p^*\tp V_p^*$.
\end{remark}

\begin{theorem}[Converse]\label{thm:C-converse}
  Conversely, given \emph{any} derivative operator
  $\tilde\covd_a$ satisfying axioms~(0)--(5) and \emph{any}
  smooth tensor field $C^c{}_{ab}$ symmetric in $a,b$,
  the operator $\covd_a$ defined
  by~\eqref{eq:C-general-tensor} is also a derivative operator
  satisfying axioms~(0)--(5).
\end{theorem}

\begin{intuition}[The space of connections]
  The set of all torsion-free connections on~$\M$ is an
  \emph{affine space}: fixing any one connection
  $\tilde\covd_a$, every other connection is obtained by
  adding a symmetric tensor $C^c{}_{ab}$.  There is no
  preferred ``origin''---no canonical choice of
  connection---without additional structure such as a metric.
\end{intuition}

% ──────────────────────────────────────────────────────────────
\subsection{Christoffel symbols}\label{sec:christoffel}

The most important special case arises when we choose
$\tilde\covd_a = \partial_a$, the ordinary derivative in some
coordinate system.  The corresponding $C^c{}_{ab}$ is then
written $\chris{c}{ab}$ and called the \textbf{Christoffel
symbols} (of the second kind).

\begin{definition}[Christoffel symbols]\label{def:christoffel}
  Let $\partial_a$ denote the ordinary (coordinate) derivative
  in a chart~$\psi$.  The \textbf{Christoffel symbols}
  $\chris{c}{ab}$ are defined by the relation
  \begin{equation}\label{eq:christoffel-def-vector}
    \covd_a t^b = \partial_a t^b + \chris{b}{ac}\, t^c\,.
  \end{equation}
\end{definition}

\noindent
In coordinate components (replacing abstract indices
$a,b,c,\dots$ by coordinate indices $\mu,\nu,\lambda,\dots$),
this reads
\begin{equation}\label{eq:covd-coord}
  \eqbox{\covd_\mu t^\nu
    = \pd{t^\nu}{x^\mu} + \chris{\nu}{\mu\lambda}\, t^\lambda}\,.
\end{equation}

\begin{remark}
  Although $\chris{c}{ab}$ has three indices, it does
  \textbf{not} transform as a tensor under coordinate changes.
  Under a change of coordinates $x^\mu\to x'^{\mu'}$, the
  Christoffel symbols pick up an inhomogeneous term involving
  second derivatives of the coordinate transformation.  It is
  only the \emph{difference} of two connections that transforms
  tensorially.
\end{remark}

For a covector field $\omega_b$, the corresponding formula is
\begin{equation}\label{eq:covd-covector-coord}
  \covd_\mu\omega_\nu
    = \pd{\omega_\nu}{x^\mu}
      - \chris{\lambda}{\mu\nu}\,\omega_\lambda\,.
\end{equation}
For a general tensor $T\in\ttype{k}{l}$:
\begin{multline}\label{eq:covd-general-coord}
  \covd_\sigma T^{\mu_1\cdots\mu_k}{}_{\nu_1\cdots\nu_l}
  = \pd{T^{\mu_1\cdots\mu_k}{}_{\nu_1\cdots\nu_l}}{x^\sigma}
  + \sum_{j=1}^{k}
    \chris{\mu_j}{\sigma\lambda}\,
    T^{\mu_1\cdots\lambda\cdots\mu_k}{}_{\nu_1\cdots\nu_l} \\
  - \sum_{j=1}^{l}
    \chris{\lambda}{\sigma\nu_j}\,
    T^{\mu_1\cdots\mu_k}{}_{\nu_1\cdots\lambda\cdots\nu_l}\,.
\end{multline}

\begin{exercise}\label{ex:covd-metric-components}
  Using~\eqref{eq:covd-general-coord}, write out the
  explicit expression for $\covd_\sigma g_{\mu\nu}$ in terms
  of partial derivatives and Christoffel symbols.
\end{exercise}

\begin{exercise}\label{ex:christoffel-not-tensor}
  Let $x^\mu$ and $x'^{\mu'}$ be two coordinate systems.
  Derive the transformation law for the Christoffel symbols
  $\chris{\nu}{\mu\lambda}$ under the coordinate change and
  verify that it contains an inhomogeneous (non-tensorial) term.
\end{exercise}
