%!TEX root = ../GeneralRelativity.tex
% ──────────────────────────────────────────────────────────────
%  Lecture 8 — Parallel transport continued
% ──────────────────────────────────────────────────────────────

\section{Parallel transport continued}\label{sec:parallel-transport-contd}

In the previous lecture we introduced derivative operators
(covariant derivatives) on a manifold~$\M$.  We now use them to
define \emph{parallel transport}---the notion of moving a vector
(or tensor) along a curve ``without rotating it.''  When a metric
is present, requiring that parallel transport preserves inner
products singles out a unique derivative operator: the
\emph{Levi-Civita connection}.  We derive its explicit form in
coordinates (the Christoffel symbols) and conclude with the
geodesic equation.

% ──────────────────────────────────────────────────────────────
\subsection{Parallel transport}\label{sec:parallel-transport-def}

Let $\M$ be a manifold equipped with a derivative operator
$\covd_a$.  Let $C$ be a smooth curve in~$\M$ with tangent
vector~$t^a$.

% Diagram: curve C from p to q with tangent and transported vectors
\begin{center}
\begin{tikzpicture}[scale=0.9]
  % Manifold blob
  \draw[thick, spacecadet, rounded corners=14pt]
    (0,0.3) to[out=25,in=160] (6.5,0.6)
    to[out=-20,in=50] (7.0,-1.0)
    to[out=230,in=-10] (0.5,-1.2)
    to[out=170,in=250] cycle;
  \node[font=\sf\small, text=cgblue] at (6.8,0.9) {$\M$};
  % Curve
  \draw[thick, munsell]
    (1.0,-0.3) to[out=15,in=200] (3.0,0.1)
    to[out=20,in=190] (5.5,-0.1);
  \node[font=\scriptsize, munsell, below] at (3.0,-0.2) {$C$};
  % Point p
  \node[circle, fill=banana, inner sep=1.8pt] (P) at (1.0,-0.3) {};
  \node[font=\small, below left] at (P) {$p$};
  % Point q
  \node[circle, fill=banana, inner sep=1.8pt] (Q) at (5.5,-0.1) {};
  \node[font=\small, below right] at (Q) {$q$};
  % Tangent vector at midpoint
  \node[circle, fill=banana, inner sep=1.2pt] (M) at (3.0,0.1) {};
  \draw[vecstyle] (M) -- ++(0.7,0.15)
    node[above, font=\small]{$t^a$};
  % Transported vector at p
  \draw[vecstyle, spacecadet] (P) -- ++(0.3,0.7)
    node[above, font=\small]{$v^a$};
  % Transported vector at midpoint
  \draw[vecstyle, spacecadet] (M) -- ++(0.3,0.7)
    node[above, font=\small]{$v^a$};
  % Transported vector at q
  \draw[vecstyle, spacecadet] (Q) -- ++(0.3,0.7)
    node[above, font=\small]{$v^a$};
\end{tikzpicture}
\end{center}

The curve $C$ connects a point~$p$ to a point~$q$.  We wish to
define a linear map
\[
  U_C\colon V_p \to V_q
\]
that ``transports'' vectors from~$p$ to~$q$ along~$C$.

\begin{definition}[Parallel transport of a vector]%
\label{def:parallel-transport}
  Let $C$ be a smooth curve in~$\M$ with tangent vector~$t^a$.
  A vector field~$v^a$ defined along~$C$ is said to be
  \textbf{parallelly transported} along~$C$ if
  \begin{equation}\label{eq:parallel-transport-vec}
    \eqbox{t^a \covd_a v^b = 0}
  \end{equation}
  at every point of~$C$.
\end{definition}

\noindent
More generally, if
$T^{a_1\cdots a_k}{}_{c_1\cdots c_l}$ is a tensor field of
type~$(k,l)$ defined along~$C$, we say it is parallelly
transported if
\begin{equation}\label{eq:parallel-transport-tensor}
  t^a \covd_a T^{a_1\cdots a_k}{}_{c_1\cdots c_l} = 0\,.
\end{equation}

% ──────────────────────────────────────────────────────────────
\subsubsection{Parallel transport in coordinates}%
\label{sec:parallel-transport-coords}

Let $\psi$ be a coordinate chart determining the ordinary
derivative operator $\partial_a$.  Recall that the covariant
derivative of a vector is related to the ordinary derivative by
\[
  \covd_a v^b = \partial_a v^b
    + \chris{b}{ac}\, v^c\,.
\]
In the chart $\psi$, the tangent vector to the curve has
components $t^\mu = \dd x^\mu/\dd t$.
The parallel transport condition
$t^a \covd_a v^b = 0$ becomes
\[
  \frac{\dd x^\mu}{\dd t}\,
  \pd{v^\nu}{x^\mu}
  + \frac{\dd x^\mu}{\dd t}\,
  \chris{\nu}{\mu\lambda}\, v^\lambda = 0\,.
\]
The first factor in each term combines to give
$\dd v^\nu/\dd t$ along the curve.  Hence:
\begin{equation}\label{eq:parallel-transport-coord}
  \eqbox{\frac{\dd v^\nu}{\dd t}
    + \frac{\dd x^\mu}{\dd t}\,
      \chris{\nu}{\mu\lambda}\, v^\lambda = 0}
\end{equation}

\begin{remark}
  The parallel transport condition~\eqref{eq:parallel-transport-coord}
  only involves the values of~$v^a$ on the curve~$C$.  We do
  \emph{not} need $v^a$ to be defined as a vector field on all
  of~$\M$---it suffices for $v^a$ to be given along~$C$.
\end{remark}

\begin{remark}
  Equation~\eqref{eq:parallel-transport-coord} is a system of
  linear first-order ordinary differential equations for the
  components $v^\nu(t)$.  By the standard existence and uniqueness
  theorem for ODEs, given an initial vector $v^a|_p$, there exists
  a unique solution along~$C$.  This defines the
  \textbf{parallel transporter}
  \[
    U_C\colon V_p \to V_q\,,
    \qquad v^a(p) \mapsto \tilde{v}^a(q)\,,
  \]
  which is a linear isomorphism (linearity follows from the
  linearity of the ODE).
\end{remark}

% ──────────────────────────────────────────────────────────────
\subsection{The Levi-Civita connection}%
\label{sec:levi-civita}

If the manifold $\M$ is equipped with a metric $g_{ab}$, we can
make a \emph{canonical} choice of derivative operator $\covd_a$.
The guiding physical requirement is that parallel transport should
preserve lengths and angles.

\subsubsection{Metric compatibility}\label{sec:metric-compat}

Consider two vectors $v^a, w^a \in V_p$ that are parallelly
transported along a curve~$C$ with tangent~$t^a$:
\[
  t^a \covd_a v^b = 0\,,
  \qquad
  t^a \covd_a w^b = 0\,.
\]
We require that the inner product
$(v,w)_p = g_{ab}\, v^a w^b$ is preserved during transport.
That is,
\[
  t^a \covd_a \bigl(g_{bc}\, v^b w^c\bigr) = 0\,.
\]
Applying the Leibniz rule:
\begin{align}
  0 &= (t^a \covd_a g_{bc})\, v^b w^c
     + g_{bc}\,(t^a \covd_a v^b)\, w^c
     + g_{bc}\, v^b\,(t^a \covd_a w^c)\,.
     \label{eq:leibniz-metric}
\end{align}
The last two terms vanish by the parallel transport condition.
Since this must hold for \emph{all} vectors $v^b$, $w^c$, all
curves, and all tangent vectors~$t^a$, we conclude:

\begin{equation}\label{eq:metric-compatibility}
  \eqbox{\covd_a g_{bc} = 0}
\end{equation}

\begin{definition}[Metric compatibility]\label{def:metric-compat}
  A derivative operator $\covd_a$ is said to be
  \textbf{metric-compatible} (or \textbf{compatible with the
  metric} $g_{ab}$) if $\covd_a g_{bc} = 0$.
\end{definition}

\begin{intuition}[Why metric compatibility?]
  The metric encodes the notions of length and angle.  If
  $\covd_a g_{bc} = 0$, then parallel transport preserves the
  inner product: two vectors that are orthogonal at~$p$ remain
  orthogonal after transport to~$q$, and their lengths do not
  change.  This is the natural analogue of the familiar fact
  that, in Euclidean space, translating a vector does not alter
  its length or direction.
\end{intuition}

% ──────────────────────────────────────────────────────────────
\subsubsection{Existence and uniqueness of the Levi-Civita
  connection}\label{sec:levi-civita-unique}

\begin{theorem}[Levi-Civita connection]\label{thm:levi-civita}
  Let $g_{ab}$ be a metric on~$\M$.  Then there exists a
  \textbf{unique} torsion-free derivative operator $\covd_a$
  satisfying $\covd_a g_{bc} = 0$.
\end{theorem}

\begin{proof}
  Let $\tilde\covd_a$ be \emph{any} derivative operator
  (for instance, the ordinary derivative $\partial_a$ associated
  with some coordinate chart).  Any other derivative operator
  $\covd_a$ differs from~$\tilde\covd_a$ by a tensor
  $C^d{}_{ab}$:
  \[
    \covd_a \omega_b
      = \tilde\covd_a \omega_b - C^d{}_{ab}\, \omega_d\,.
  \]
  Since $\covd_a$ is torsion-free, $C^d{}_{ab} = C^d{}_{ba}$.

  \smallskip
  \noindent\textbf{Step 1: Impose metric compatibility.}\;
  The condition $\covd_a g_{bc} = 0$ reads
  \begin{equation}\label{eq:metric-compat-expand}
    0 = \covd_a g_{bc}
      = \tilde\covd_a g_{bc}
        - C^d{}_{ab}\, g_{dc}
        - C^d{}_{ac}\, g_{bd}\,.
  \end{equation}
  Writing $C_{cab} = g_{dc}\, C^d{}_{ab}$, this becomes
  \begin{equation}\label{eq:compat-lowered}
    \tilde\covd_a g_{bc} = C_{cab} + C_{bac}\,.
  \end{equation}

  \noindent\textbf{Step 2: Cyclic permutations.}\;
  Write~\eqref{eq:compat-lowered} with three different index
  orderings:
  \begin{align}
    \tilde\covd_a g_{bc} &= C_{cab} + C_{bac}\,,
      \label{eq:perm1}\\
    \tilde\covd_b g_{ac} &= C_{cba} + C_{abc}\,,
      \label{eq:perm2}\\
    \tilde\covd_c g_{ab} &= C_{bca} + C_{acb}\,.
      \label{eq:perm3}
  \end{align}
  Form the combination
  $\eqref{eq:perm1} + \eqref{eq:perm2} - \eqref{eq:perm3}$:
  \begin{align}
    \tilde\covd_a g_{bc}
    + \tilde\covd_b g_{ac}
    - \tilde\covd_c g_{ab}
    &= C_{cab} + C_{bac}
     + C_{cba} + C_{abc}
     - C_{bca} - C_{acb}\,.
     \notag
  \end{align}
  Using the torsion-free symmetry
  $C_{cab} = C_{cba}$, the right-hand side simplifies to
  $2C_{cab}$.  Therefore
  \begin{equation}\label{eq:C-solved}
    C_{cab} = \tfrac{1}{2}\bigl(
      \tilde\covd_a g_{bc}
      + \tilde\covd_b g_{ac}
      - \tilde\covd_c g_{ab}\bigr)\,.
  \end{equation}
  Raising the first index:
  \begin{equation}\label{eq:C-solved-raised}
    C^c{}_{ab} = \tfrac{1}{2}\, g^{cd}\bigl(
      \tilde\covd_a g_{bd}
      + \tilde\covd_b g_{ad}
      - \tilde\covd_d g_{ab}\bigr)\,.
  \end{equation}
  Since $C^c{}_{ab}$ is uniquely determined
  by~$\tilde\covd_a$ and~$g_{ab}$, the derivative operator
  $\covd_a$ is unique.
\end{proof}

\begin{remark}
  From now on, whenever a metric $g_{ab}$ is given, we choose
  $\covd_a$ to be the unique Levi-Civita connection determined
  by~$g_{ab}$.
\end{remark}

% ──────────────────────────────────────────────────────────────
\subsection{Christoffel symbols}\label{sec:christoffel}

In the special case where $\tilde\covd_a = \partial_a$ is the
ordinary derivative associated with a coordinate chart~$\psi$,
the tensor $C^c{}_{ab}$ becomes the Christoffel symbol
$\chris{c}{ab}$.  Since $\partial_a g_{bd}$ simply
differentiates the metric components, we obtain
(in abstract index notation):
\begin{equation}\label{eq:christoffel-abstract}
  \chris{c}{ab}
    = \tfrac{1}{2}\, g^{cd}\bigl(
      \partial_a g_{bd}
      + \partial_b g_{ad}
      - \partial_d g_{ab}\bigr)\,.
\end{equation}

In the coordinate chart $\psi$ with coordinates
$(x^1,\dots,x^n)$, this reads:
\begin{equation}\label{eq:christoffel-coord}
  \eqbox{\chris{\rho}{\mu\nu}
    = \frac{1}{2}\sum_\sigma g^{\rho\sigma}
      \!\left(
        \pd{g_{\sigma\mu}}{x^\nu}
        + \pd{g_{\sigma\nu}}{x^\mu}
        - \pd{g_{\mu\nu}}{x^\sigma}
      \right)}
\end{equation}

\begin{keyresult}[Christoffel symbols from the metric]
  The Christoffel symbols~\eqref{eq:christoffel-coord} are
  determined entirely by the metric and its first derivatives.
  They are symmetric in the lower two indices:
  $\chris{\rho}{\mu\nu} = \chris{\rho}{\nu\mu}$.
  Note that they are \emph{not} the components of a tensor
  (see~\S\ref{sec:connection-transform} below).
\end{keyresult}

% ──────────────────────────────────────────────────────────────
\subsection{Example: $\Rn{2}$ in Cartesian and polar coordinates}%
\label{sec:R2-example}

\subsubsection{Cartesian coordinates}\label{sec:R2-cartesian}

\begin{example}[Flat metric in Cartesian coordinates]%
\label{ex:R2-cartesian}
  Let $\M = \Rn{2}$ with Cartesian coordinates $(x,y)$.  A
  vector is expanded as
  \[
    v = \sum_{\mu = x,y} v^\mu\,
      \pd{}{x^\mu}\bigg|_p\,.
  \]
  The Euclidean metric is
  \[
    g_{\mu\nu}\,\dd x^\mu \tp \dd x^\nu
      = \dd x \tp \dd x + \dd y \tp \dd y\,,
  \]
  so $g_{\mu\nu} = \delta_{\mu\nu}$ and all derivatives of
  $g_{\mu\nu}$ vanish.  Hence
  \[
    \chris{\lambda}{\mu\nu} = 0\,.
  \]
  Choose $\tilde\covd_a = \partial_a$ with respect to Cartesian
  coordinates.  Then $\covd_a = \partial_a$, and the parallel
  transport equation $t^a \covd_a v^b = 0$ becomes
  \[
    t^\nu\,\pd{v^\mu}{x^\nu} = 0
    \quad\Longrightarrow\quad
    v^\mu \text{ is constant along } C\,.
  \]
  This is the familiar result: in flat space with Cartesian
  coordinates, parallel transport simply keeps the components
  of the vector constant.
\end{example}

% ──────────────────────────────────────────────────────────────
\subsubsection{Polar coordinates}\label{sec:R2-polar}

\begin{example}[Flat metric in polar coordinates]%
\label{ex:R2-polar}
  Now introduce polar coordinates $(r,\varphi)$ on $\Rn{2}$
  via $(x,y) = (r\cos\varphi,\; r\sin\varphi)$.
  Using $\dd x = \cos\varphi\,\dd r - r\sin\varphi\,\dd\varphi$
  and $\dd y = \sin\varphi\,\dd r + r\cos\varphi\,\dd\varphi$,
  the metric becomes
  \begin{equation}\label{eq:polar-metric}
    g = \dd r\tp\dd r + r^2\,\dd\varphi\tp\dd\varphi\,.
  \end{equation}
  In matrix form:
  \[
    g_{\mu'\nu'}
      = \begin{pmatrix} 1 & 0 \\ 0 & r^2 \end{pmatrix},
    \qquad
    g^{\mu'\nu'}
      = \begin{pmatrix} 1 & 0 \\ 0 & 1/r^2 \end{pmatrix}.
  \]

  \noindent\textbf{Christoffel symbols.}\;
  From~\eqref{eq:christoffel-coord}, the only nonvanishing
  derivatives of $g_{\mu'\nu'}$ involve
  $\partial g_{\varphi\varphi}/\partial r = 2r$.
  A direct computation gives:
  \begin{equation}\label{eq:chris-polar}
    \chris{r}{\varphi\varphi} = -r\,,
    \qquad
    \chris{\varphi}{r\varphi}
      = \chris{\varphi}{\varphi r}
      = \frac{1}{r}\,,
  \end{equation}
  with all other components vanishing.
\end{example}

\begin{exercise}\label{ex:chris-polar}
  Verify the Christoffel
  symbols~\eqref{eq:chris-polar} by direct computation
  from~\eqref{eq:christoffel-coord}.
\end{exercise}

% ──────────────────────────────────────────────────────────────
\subsubsection{Parallel transport along a radial line}%
\label{sec:radial-transport}

Take the curve $C$ to be a displacement along the radial
direction: the tangent vector is
$t^a = (\partial/\partial r)^a$, so $t^r = 1$ and
$t^\varphi = 0$.

The parallel transport
equation~\eqref{eq:parallel-transport-coord} yields two
component equations:
\begin{align}
  \frac{\dd v^r}{\dd r}
    + \chris{r}{rr}\, v^r
    + \chris{r}{r\varphi}\, v^\varphi
    &= 0
  \quad\Longrightarrow\quad
  \frac{\dd v^r}{\dd r} = 0\,,
  \label{eq:radial-vr}\\[4pt]
  \frac{\dd v^\varphi}{\dd r}
    + \chris{\varphi}{rr}\, v^r
    + \chris{\varphi}{r\varphi}\, v^\varphi
    &= 0
  \quad\Longrightarrow\quad
  \frac{\dd v^\varphi}{\dd r} = -\frac{v^\varphi}{r}\,.
  \label{eq:radial-vphi}
\end{align}

From~\eqref{eq:radial-vr}, $v^r$ is constant.
Equation~\eqref{eq:radial-vphi} is separable:
\[
  \frac{1}{v^\varphi}\,\dd v^\varphi = -\frac{1}{r}\,\dd r
  \quad\Longrightarrow\quad
  \log v^\varphi = -\log r + \text{const}
  \quad\Longrightarrow\quad
  v^\varphi = \frac{A}{r}\,,
\]
where $A$ is a constant determined by the initial conditions.

\begin{remark}
  The components $v^\varphi$ change along the curve, but this
  is entirely because the coordinate basis vectors
  $\partial/\partial r$ and $\partial/\partial\varphi$ themselves
  change from point to point.  The vector $v^a$ is ``constant''
  in the Cartesian sense---the parallel transport equation
  correctly accounts for the variation of the basis.
\end{remark}

% ──────────────────────────────────────────────────────────────
\subsection{Covariant derivative of one-forms and connection
  one-forms}\label{sec:covd-oneforms}

Suppose $\covd_a$ is a derivative operator.  Let $\omega$ be a
covector (a tensor of type~$(0,1)$) and let $\psi$ be a
coordinate chart.  In the coordinate basis, the covariant
derivative of the dual basis element $\dd x^\nu$ in the
direction $\partial/\partial x^\mu$ is
\begin{equation}\label{eq:covd-dual-basis}
  \covd_{\partial/\partial x^\mu}(\dd x^\nu)
    = -\chris{\nu}{\mu\lambda}\,\dd x^\lambda\,.
\end{equation}
For each choice of $\mu$ and $\nu$, this defines a one-form
called a \textbf{connection one-form}.

% ──────────────────────────────────────────────────────────────
\subsection{Transformation of connection coefficients}%
\label{sec:connection-transform}

Let $(U,\psi)$ and $(V,\psi')$ be two overlapping charts with
coordinates $x^\mu = \psi(p)$ and $y^\alpha = \psi'(p)$.
Write $\chris{\gamma}{\alpha\beta}$ for the Christoffel symbols
in the $x$-chart and $\tilde\Gamma^\gamma{}_{\alpha\beta}$ for
those in the $y$-chart.

Starting from the definition
$\chris{\gamma}{\alpha\beta}\,\partial/\partial x^\gamma
  = \covd_{\partial/\partial x^\alpha}
    (\partial/\partial x^\beta)$,
one can express the basis vectors
$\partial/\partial y^\alpha
  = (\partial x^\mu/\partial y^\alpha)\,
    \partial/\partial x^\mu$
and compute:
\begin{equation}\label{eq:connection-transform}
  \eqbox{\tilde\Gamma^\gamma{}_{\alpha\beta}
    = \pd{x^\lambda}{y^\alpha}\,
      \pd{x^n}{y^\beta}\,
      \pd{y^\gamma}{x^\nu}\,
      \chris{\nu}{\lambda n}
    + \frac{\partial^2 x^\nu}{\partial y^\alpha\,
        \partial y^\beta}\,
      \pd{y^\gamma}{x^\nu}}
\end{equation}

\begin{remark}
  The first term in~\eqref{eq:connection-transform} is the
  standard tensorial transformation law.  The second term,
  involving second derivatives of the coordinate
  transformation, shows that $\chris{\nu}{\mu\lambda}$ does
  \emph{not} transform as a tensor.  It is precisely this
  inhomogeneous term that allows $\covd_a v^b$ to be a tensor
  even though $v^b$ and $\chris{b}{ac}$ individually are not.
\end{remark}

\begin{exercise}\label{ex:covd-invariance}
  Show that $\covd_X Y$ is independent of the choice of
  coordinates: if $X = X^\mu\,\partial/\partial x^\mu$,
  $Y = Y^\nu\,\partial/\partial x^\nu$ in one chart and
  $X = \tilde X^\alpha\,\partial/\partial y^\alpha$,
  $Y = \tilde Y^\beta\,\partial/\partial y^\beta$ in another,
  then $X^\mu \covd_{\partial/\partial x^\mu}
    (Y^\nu\,\partial/\partial x^\nu)
  = \tilde X^\alpha \covd_{\partial/\partial y^\alpha}
    (\tilde Y^\beta\,\partial/\partial y^\beta)$.
\end{exercise}

% ──────────────────────────────────────────────────────────────
\subsection{Geodesics}\label{sec:geodesics}

We now arrive at one of the central concepts in general
relativity.

\begin{definition}[Geodesic]\label{def:geodesic}
  Let $(\M, g_{ab})$ be a manifold with metric.  A
  \textbf{geodesic} is a curve whose tangent vector $T^a$ is
  parallelly transported along itself:
  \begin{equation}\label{eq:geodesic-abstract}
    \eqbox{T^a \covd_a T^b = 0}
  \end{equation}
\end{definition}

\begin{intuition}[Geodesics as straightest curves]
  A geodesic is the ``straightest possible'' curve one can
  trace through~$\M$.  On a curved manifold there is in general
  no global notion of a ``straight line,'' but the geodesic
  equation captures the idea locally: the tangent vector does
  not turn relative to the connection.  In flat space, geodesics
  reduce to straight lines.
\end{intuition}

% Diagram: geodesic on a curved surface
\begin{center}
\begin{tikzpicture}[scale=0.85]
  % Curved manifold
  \draw[thick, spacecadet, rounded corners=14pt]
    (0,0.5) to[out=20,in=160] (5.5,0.8)
    to[out=-20,in=40] (6.0,-0.6)
    to[out=220,in=-10] (0.5,-0.5)
    to[out=170,in=250] cycle;
  \node[font=\sf\small, text=cgblue] at (5.8,1.0) {$\M$};
  % Geodesic curve
  \draw[thick, munsell]
    (1.0,0.1) to[out=10,in=190] (3.0,0.3)
    to[out=10,in=200] (5.0,0.2);
  % Points
  \node[circle, fill=banana, inner sep=1.8pt] (P) at (1.0,0.1) {};
  \node[font=\small, below left] at (P) {$p$};
  \node[circle, fill=banana, inner sep=1.8pt] (Q) at (5.0,0.2) {};
  \node[font=\small, below right] at (Q) {$q$};
  % Tangent vectors
  \draw[vecstyle] (P) -- ++(0.8,0.12)
    node[above, font=\small]{$T^a$};
  \node[circle, fill=banana, inner sep=1.2pt] (M) at (3.0,0.3) {};
  \draw[vecstyle] (M) -- ++(0.8,0.08)
    node[above, font=\small]{$T^a$};
  \draw[vecstyle] (Q) -- ++(0.7,-0.05)
    node[above, font=\small]{$T^a$};
\end{tikzpicture}
\end{center}

\subsubsection{Geodesic equation in coordinates}%
\label{sec:geodesic-coords}

Choose a coordinate chart~$\psi$.  A curve is described by
functions $x^\mu(t)$, and the tangent vector has components
$T^\mu = \dd x^\mu/\dd t$.  Applying the parallel transport
equation~\eqref{eq:parallel-transport-coord} with
$v^\mu = T^\mu$:
\[
  \frac{\dd T^\mu}{\dd t}
    + \chris{\mu}{\sigma\nu}\,
      \frac{\dd x^\sigma}{\dd t}\,
      T^\nu = 0\,.
\]
Since $T^\mu = \dd x^\mu/\dd t$, we write
$\dd T^\mu/\dd t = \dd^2 x^\mu/\dd t^2$ to obtain:

\begin{equation}\label{eq:geodesic-coord}
  \eqbox{\frac{\dd^2 x^\mu}{\dd t^2}
    + \sum_{\sigma,\nu}
      \chris{\mu}{\sigma\nu}\,
      \frac{\dd x^\sigma}{\dd t}\,
      \frac{\dd x^\nu}{\dd t}
    = 0}
\end{equation}

\begin{remark}
  Equation~\eqref{eq:geodesic-coord} is a system of $n$
  coupled second-order ordinary differential equations for the
  $n$ functions $x^\mu(t)$.  Given initial position
  $x^\mu(0)$ and initial velocity
  $\dd x^\mu/\dd t|_{t=0}$, the ODE existence and uniqueness
  theorem guarantees a unique geodesic (at least locally).
\end{remark}

\begin{exercise}\label{ex:geodesic-flat}
  Show that the geodesics of Euclidean $\Rn{n}$ (with the
  standard flat metric in Cartesian coordinates) are straight
  lines.
\end{exercise}

\begin{exercise}\label{ex:geodesic-sphere}
  Compute the Christoffel symbols for the round metric
  $g = \dd\theta\tp\dd\theta
    + \sin^2\!\theta\;\dd\varphi\tp\dd\varphi$
  on $S^2$, and write down the geodesic equations.
  Verify that great circles satisfy these equations.
\end{exercise}
