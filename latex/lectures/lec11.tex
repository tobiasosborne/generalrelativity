%!TEX root = ../GeneralRelativity.tex
% ──────────────────────────────────────────────────────────────
%  Lecture 11 — Curvature continued
% ──────────────────────────────────────────────────────────────

\section{Curvature continued}\label{sec:curvature-continued}

In the previous lecture we introduced the Riemann curvature tensor
$\Riem_{abc}{}^d$ via the commutator of covariant derivatives
acting on a covector field.  We now extend this to vectors and
general tensors, establish the fundamental symmetry and Bianchi
identities, define the Ricci tensor and scalar curvature, derive
the geodesic deviation equation, and give the explicit coordinate
formula for the Riemann tensor.

% ──────────────────────────────────────────────────────────────
\subsection{Action of the Riemann tensor on vectors and tensors}%
\label{sec:riemann-on-vectors}

Recall that for a covector field~$\omega_c$:
\begin{equation}\label{eq:riemann-on-covector-recall}
  (\covd_a \covd_b - \covd_b \covd_a)\omega_c
    = \Riem_{abc}{}^d\, \omega_d\,.
\end{equation}
We now determine the action of the commutator
$[\covd_a, \covd_b] \equiv \covd_a \covd_b - \covd_b \covd_a$
on vectors and general tensor fields.

\subsubsection{Action on a vector field}%
\label{sec:riemann-action-vector}

Let $t^c$ be an arbitrary vector field and $\omega_c$ an
arbitrary covector field.  The contraction $t^c \omega_c$ is a
scalar, so by the torsion-free property:
\begin{equation}\label{eq:commutator-scalar-zero}
  0 = (\covd_a \covd_b - \covd_b \covd_a)(t^c \omega_c)\,.
\end{equation}
Expanding by the Leibniz rule:
\begin{align}
  0
    &= \covd_a(\omega_c\, \covd_b t^c + t^c\, \covd_b \omega_c)
     - \covd_b(\omega_c\, \covd_a t^c + t^c\, \covd_a \omega_c)
     \notag\\
    &= \omega_c\,
       (\covd_a \covd_b - \covd_b \covd_a) t^c
     + t^c\,
       (\covd_a \covd_b - \covd_b \covd_a) \omega_c
     \notag\\
    &= \omega_c\,
       (\covd_a \covd_b - \covd_b \covd_a) t^c
     + t^c\, \omega_d\, \Riem_{abc}{}^d\,.
     \label{eq:riemann-vector-derivation}
\end{align}
Since $\omega_c$ is arbitrary, we conclude:
\begin{equation}\label{eq:riemann-on-vector}
  \eqbox{(\covd_a \covd_b - \covd_b \covd_a) t^c
    = -\Riem_{abd}{}^c\, t^d}
\end{equation}
Note the crucial minus sign: for vectors the Riemann tensor
appears with the opposite sign compared to covectors.

\subsubsection{Action on a general tensor field}%
\label{sec:riemann-action-general}

\begin{exercise}\label{ex:riemann-general-tensor}
  By repeated application of the Leibniz rule, show that for a
  tensor field
  $T^{c_1 \cdots c_k}{}_{d_1 \cdots d_l}$ of type~$(k,l)$:
\end{exercise}

\begin{equation}\label{eq:riemann-on-general}
  \eqbox{
  (\covd_a \covd_b - \covd_b \covd_a)
    T^{c_1 \cdots c_k}{}_{d_1 \cdots d_l}
  = -\sum_{j=1}^{k}
      \Riem_{abe}{}^{c_j}\,
      T^{c_1 \cdots e \cdots c_k}{}_{d_1 \cdots d_l}
    + \sum_{j=1}^{l}
      \Riem_{ab d_j}{}^{e}\,
      T^{c_1 \cdots c_k}{}_{d_1 \cdots e \cdots d_l}
  }
\end{equation}
In the first sum, the dummy index~$e$ replaces~$c_j$ in the
$j$-th contravariant slot; in the second sum, $e$ replaces~$d_j$
in the $j$-th covariant slot.

\begin{remark}
  The pattern is: one $-\Riem$ term for each upper index (just as
  for a single vector), one $+\Riem$ term for each lower index
  (just as for a single covector).  This is the universal formula
  for the ``failure of covariant derivatives to commute'' on
  tensors of arbitrary type.
\end{remark}

% ──────────────────────────────────────────────────────────────
\subsection{Symmetries of the Riemann tensor}%
\label{sec:riemann-symmetries}

We now establish four fundamental properties of
$\Riem_{abc}{}^d$.  Properties~(1)--(2) hold for \emph{any}
torsion-free derivative operator; properties~(3)--(4) require in
addition that $\covd_a$ be the metric-compatible (Levi-Civita)
connection.

% ──────────────────────────────────────────────────────────────
\subsubsection{Property 1: Antisymmetry in the first two indices}%
\label{sec:riemann-prop1}

\begin{theorem}\label{thm:riemann-antisym-ab}
  \begin{equation}\label{eq:riemann-antisym-ab}
    \Riem_{abc}{}^d = -\Riem_{bac}{}^d\,.
  \end{equation}
\end{theorem}

\begin{proof}
  This follows directly from the definition
  $\Riem_{abc}{}^d\, \omega_d
    = (\covd_a \covd_b - \covd_b \covd_a)\omega_c$.
  The commutator $[\covd_a, \covd_b]$ is manifestly
  antisymmetric in $a$ and~$b$.
\end{proof}

% ──────────────────────────────────────────────────────────────
\subsubsection{Property 2: First Bianchi identity (algebraic
  Bianchi identity)}\label{sec:riemann-prop2}

\begin{theorem}[First Bianchi identity]\label{thm:first-bianchi}
  For any torsion-free derivative operator:
  \begin{equation}\label{eq:first-bianchi}
    \Riem_{[abc]}{}^d = 0\,.
  \end{equation}
\end{theorem}

\begin{proof}
  Let $\omega_a$ be an arbitrary covector field and $\covd_a$ a
  torsion-free derivative operator.  We claim that
  \begin{equation}\label{eq:double-covd-antisym}
    \covd_{[a} \covd_b \omega_{c]} = 0\,.
  \end{equation}
  To verify~\eqref{eq:double-covd-antisym}, let
  $\tilde\covd_a = \partial_a$ be the ordinary derivative in
  some coordinate chart.  Then
  $\covd_b \omega_c
    = \partial_b \omega_c - C^d{}_{bc}\, \omega_d$,
  and hence
  \[
    \covd_a \covd_b \omega_c
      = \partial_a \partial_b \omega_c
        + C^d{}_{ab}\, \partial_{|d|} \omega_c
        + C^e{}_{ac}\, C^d{}_{|e|b}\, \omega_d
        + \cdots\,.
  \]
  Antisymmetrising over $a,b,c$ and using the torsion-free
  condition $C^d{}_{ab} = C^d{}_{ba}$, all terms
  vanish, confirming~\eqref{eq:double-covd-antisym}.

  Now write
  \begin{align}
    0 &= 2\,\covd_{[a} \covd_b \omega_{c]}
       = \covd_{[a} \covd_b \omega_{c]}
         - \covd_{[b} \covd_a \omega_{c]}
       = \Riem_{[abc]}{}^d\, \omega_d\,.
       \label{eq:first-bianchi-proof}
  \end{align}
  Since $\omega_d$ is arbitrary,
  $\Riem_{[abc]}{}^d = 0$.
\end{proof}

% ──────────────────────────────────────────────────────────────
\subsubsection{Property 3: Antisymmetry in the last two indices}%
\label{sec:riemann-prop3}

\begin{theorem}\label{thm:riemann-antisym-cd}
  If $\covd_a$ is the Levi-Civita connection
  ($\covd_a g_{bc} = 0$), then
  \begin{equation}\label{eq:riemann-antisym-cd}
    \Riem_{abcd} = -\Riem_{abdc}\,,
  \end{equation}
  where $\Riem_{abcd} \equiv g_{de}\, \Riem_{abc}{}^e$.
\end{theorem}

\begin{proof}
  Since $\covd_a g_{cd} = 0$, applying the commutator formula to
  the metric tensor gives
  \begin{align}
    0 &= (\covd_a \covd_b - \covd_b \covd_a) g_{cd}
       = \Riem_{abc}{}^e\, g_{ed}
         + \Riem_{abd}{}^e\, g_{ce}
       = \Riem_{abcd} + \Riem_{abdc}\,.
       \label{eq:antisym-cd-proof}
  \end{align}
  Hence $\Riem_{abcd} = -\Riem_{abdc}$.
\end{proof}

\begin{remark}
  Together, properties~(1)--(3) imply that $\Riem_{abcd}$ is
  antisymmetric in the pair $(a,b)$ and in the pair $(c,d)$, and
  symmetric under exchange of the two pairs:
  $\Riem_{abcd} = \Riem_{cdab}$.
\end{remark}

% ──────────────────────────────────────────────────────────────
\subsubsection{Property 4: Differential Bianchi identity}%
\label{sec:riemann-prop4}

\begin{theorem}[Bianchi identity]\label{thm:bianchi-identity}
  For any torsion-free derivative operator:
  \begin{equation}\label{eq:bianchi-identity}
    \covd_{[a}\, \Riem_{bc]d}{}^e = 0\,.
  \end{equation}
\end{theorem}

\begin{proof}[Proof sketch]
  Let $\omega_d$ be an arbitrary covector field.  Consider the
  two expressions:
  \begin{enumerate}[(i)]
    \item $(\covd_a \covd_b - \covd_b \covd_a)\covd_c \omega_d
      = \Riem_{abc}{}^e\, \covd_e \omega_d
        + \Riem_{abd}{}^f\, \covd_c \omega_f$.
    \item $\covd_a(\covd_b \covd_c - \covd_c \covd_b)\omega_d
      = \covd_a(\Riem_{bcd}{}^e\, \omega_e)
      = \omega_e\, \covd_a \Riem_{bcd}{}^e
        + \Riem_{bcd}{}^e\, \covd_a \omega_e$.
  \end{enumerate}
  Antisymmetrise both expressions over $a,b,c$.  In~(i), the
  term $\Riem_{[abc]}{}^e\, \covd_e \omega_d$ vanishes by the
  first Bianchi identity, and the remaining terms cancel with
  corresponding terms from~(ii).  This yields
  \[
    \omega_e\, \covd_{[a}\, \Riem_{bc]d}{}^e = 0\,.
  \]
  Since $\omega_e$ is arbitrary:
  $\covd_{[a}\, \Riem_{bc]d}{}^e = 0$.
\end{proof}

% ──────────────────────────────────────────────────────────────
\subsection{Ricci tensor, scalar curvature, and the Einstein
  tensor}\label{sec:ricci-einstein}

\subsubsection{Geometric interpretation of the Riemann tensor}%
\label{sec:riemann-interpretation}

Given two vectors $v^a, w^a \in V_p$, the quantity
$\Riem(v, w, \cdot, \cdot)$ defines a linear transformation
$M_a{}^b$ of~$V_p$.  The two vectors $v^a$ and $w^a$ span a
two-dimensional subspace of~$V_p$; the Riemann tensor ``probes''
the curvature of~$\M$ in the corresponding two-dimensional
plane.

\begin{intuition}[Riemann as sectional curvature probe]
  Each pair of tangent vectors at~$p$ specifies a 2-plane
  in~$V_p$.  The Riemann tensor, when fed these two directions,
  returns a linear map that encodes how vectors are rotated
  when parallel-transported around an infinitesimal loop in
  that 2-plane.  This is the essence of \emph{sectional
  curvature}.
\end{intuition}

% ──────────────────────────────────────────────────────────────
\subsubsection{The Ricci tensor}\label{sec:ricci-tensor}

\begin{definition}[Ricci tensor]\label{def:ricci-tensor}
  The \textbf{Ricci tensor} is the contraction of the Riemann
  tensor over its second and fourth indices:
  \begin{equation}\label{eq:ricci-tensor}
    \Ric_{ac} = \Riem_{abc}{}^b\,.
  \end{equation}
\end{definition}

\begin{remark}
  By the symmetries of $\Riem_{abcd}$ (properties 1--3), the
  Ricci tensor is symmetric:
  $\Ric_{ab} = \Ric_{ba}$.
\end{remark}

% ──────────────────────────────────────────────────────────────
\subsubsection{Scalar curvature}\label{sec:scalar-curvature}

\begin{definition}[Scalar curvature]\label{def:scalar-curvature}
  The \textbf{scalar curvature} (or \textbf{Ricci scalar}) is
  the trace of the Ricci tensor:
  \begin{equation}\label{eq:scalar-curvature}
    \eqbox{\Ric = \Ric_a{}^a = g^{ab}\, \Ric_{ab}}
  \end{equation}
\end{definition}

% ──────────────────────────────────────────────────────────────
\subsubsection{The contracted Bianchi identity and the Einstein
  tensor}\label{sec:einstein-tensor}

We now contract the differential Bianchi
identity~\eqref{eq:bianchi-identity} to obtain a fundamental
divergence condition.

\medskip\noindent
\textbf{Step 1: First contraction.}\;
Write out the Bianchi identity explicitly:
\[
  \covd_a \Riem_{bcd}{}^a
  + \covd_b \Riem_{cd}
  - \covd_c \Riem_{bd}
  = 0\,,
\]
where we used $\Riem_{bcd}{}^a$ contracted on the first and
last indices to relate to the Ricci tensor.

\medskip\noindent
\textbf{Step 2: Second contraction.}\;
Contract with $g^{bd}$:
\[
  \covd_a \Ric_c{}^a
  + \covd_b \Ric_c{}^b
  - \covd_c \Ric
  = 0\,,
\]
which gives
\begin{equation}\label{eq:contracted-bianchi}
  2\,\covd_a \Ric_c{}^a - \covd_c \Ric = 0\,.
\end{equation}
This can be rewritten as $\covd^a \Ein_{ab} = 0$, where:

\begin{definition}[Einstein tensor]\label{def:einstein-tensor}
  The \textbf{Einstein tensor} is defined by
  \begin{equation}\label{eq:einstein-tensor}
    \eqbox{\Ein_{ab}
      = \Ric_{ab} - \tfrac{1}{2}\,\Ric\, g_{ab}}
  \end{equation}
\end{definition}

\begin{keyresult}[Divergence-free Einstein tensor]
  The Einstein tensor is divergence-free:
  \begin{equation}\label{eq:einstein-divergence-free}
    \covd^a \Ein_{ab} = 0\,.
  \end{equation}
  This identity holds purely as a consequence of the Bianchi
  identity---it is a geometric identity, not a field equation.
  It will play a central role when we write down Einstein's
  field equations.
\end{keyresult}

% ──────────────────────────────────────────────────────────────
\subsection{Geodesic deviation}\label{sec:geodesic-deviation}

The Riemann tensor has a direct physical interpretation: it
controls how nearby geodesics accelerate toward or away from
each other.  This is captured by the \textbf{geodesic deviation
equation}.

% ──────────────────────────────────────────────────────────────
\subsubsection{Setup: a one-parameter family of geodesics}%
\label{sec:geodesic-family}

Consider a one-parameter family of geodesics
$\gamma_s(t)$, where $s\in(-\varepsilon,\bar\varepsilon)$
labels the different geodesics and $t$ is an affine parameter
along each one.  We require:
\begin{enumerate}
  \item $\gamma_s$ is a geodesic for every~$s$, and
  \item the map $(s,t) \mapsto \gamma_s(t)$ is smooth,
    one-to-one, with smooth inverse.
\end{enumerate}

\noindent
The image of this family is a two-dimensional submanifold
$\Sigma \subset \M$, and $(s,t)$ serve as a coordinate system
on~$\Sigma$.

% Diagram: one-parameter family of geodesics
\begin{center}
\begin{tikzpicture}[scale=0.9]
  % Manifold blob
  \draw[thick, spacecadet, rounded corners=14pt]
    (0,0.5) to[out=20,in=160] (7.0,0.8)
    to[out=-20,in=50] (7.5,-1.2)
    to[out=230,in=-10] (0.5,-1.0)
    to[out=170,in=250] cycle;
  \node[font=\sf\small, text=cgblue] at (7.2,1.0) {$\M$};
  % Geodesics
  \draw[thick, munsell]
    (1.0,0.4) to[out=5,in=175] (6.0,0.5);
  \draw[thick, munsell]
    (1.0,0.0) to[out=5,in=175] (6.0,-0.1);
  \draw[thick, munsell]
    (1.0,-0.4) to[out=5,in=175] (6.0,-0.7);
  % Labels
  \node[font=\scriptsize, munsell, right] at (6.0,0.5)
    {$\gamma_{s+\delta s}$};
  \node[font=\scriptsize, munsell, right] at (6.0,-0.1)
    {$\gamma_s$};
  \node[font=\scriptsize, munsell, right] at (6.0,-0.7)
    {$\gamma_{s-\delta s}$};
  % Point on middle geodesic
  \node[circle, fill=banana, inner sep=1.8pt] (P) at (3.5,-0.05) {};
  % T^a tangent vector
  \draw[vecstyle] (P) -- ++(0.9,0.02)
    node[above, font=\small]{$T^a$};
  % X^a deviation vector
  \draw[vecstyle, spacecadet] (P) -- ++(0.05,0.45)
    node[left, font=\small]{$X^a$};
\end{tikzpicture}
\end{center}

Define two vector fields on~$\Sigma$:
\begin{itemize}
  \item $T^a = (\partial/\partial t)^a$: the tangent vector to
    each geodesic.  Since $\gamma_s$ is a geodesic:
    $T^a \covd_a T^b = 0$.
  \item $X^a = (\partial/\partial s)^a$: the \textbf{deviation
    vector}, pointing from $\gamma_s(t)$ to the infinitesimally
    nearby geodesic $\gamma_{s+\delta s}(t)$.
\end{itemize}

\noindent
Since $T^a$ and $X^a$ are coordinate vector fields on~$\Sigma$,
they commute:
\begin{equation}\label{eq:TX-commute}
  [T, X]^a = 0
  \qquad\Longleftrightarrow\qquad
  T^b \covd_b X^a = X^b \covd_b T^a\,.
\end{equation}

\begin{exercise}\label{ex:XdotT-constant}
  Using~\eqref{eq:TX-commute} and the geodesic equation
  $T^a \covd_a T^b = 0$, show that $X^a T_a$ is constant
  along each geodesic.  (By a reparametrisation of~$s$,
  we may set $X^a T_a = 0$, so that $X^a$ is purely
  transverse.)
\end{exercise}

% ──────────────────────────────────────────────────────────────
\subsubsection{Relative velocity and acceleration}%
\label{sec:relative-accel}

Define the \textbf{relative velocity} of nearby geodesics:
\begin{equation}\label{eq:relative-velocity}
  v^a = T^b \covd_b X^a\,.
\end{equation}
This measures the rate of change of the deviation vector~$X^a$
along the geodesic.

\medskip\noindent
The \textbf{relative acceleration} is
\begin{equation}\label{eq:relative-accel-def}
  A^a = T^c \covd_c v^a
      = T^c \covd_c (T^b \covd_b X^a)\,.
\end{equation}

\noindent
We now simplify~\eqref{eq:relative-accel-def} using the
commutativity condition~\eqref{eq:TX-commute}:
\begin{align}
  A^a
    &= T^c \covd_c (X^b \covd_b T^a)
       \notag\\
    &= (T^c \covd_c X^b)(\covd_b T^a)
       + X^b\, T^c \covd_c \covd_b T^a\,.
       \label{eq:accel-expand1}
\end{align}
Using the commutativity relation again on the first term,
$(T^c \covd_c X^b) = (X^c \covd_c T^b)$:
\begin{align}
  A^a
    &= (X^c \covd_c T^b)(\covd_b T^a)
       + X^b\, T^c \covd_c \covd_b T^a\,.
       \label{eq:accel-expand2}
\end{align}
The first term equals
$X^c \covd_c(T^b \covd_b T^a)
  - X^b(\text{extra terms})$.
But $T^b \covd_b T^a = 0$ (geodesic equation), so
\[
  (X^c \covd_c T^b)(\covd_b T^a)
    = X^c \covd_c(T^b \covd_b T^a)
      - X^b(T^c \covd_b \covd_c T^a - \text{cancelled})\,.
\]
The simplification, using the definition of the Riemann tensor
$T^c \covd_c \covd_b T^a - T^c \covd_b \covd_c T^a
  = -\Riem_{cbd}{}^a\, T^c T^d$,
and the geodesic equation, yields:

\begin{equation}\label{eq:geodesic-deviation}
  \eqbox{A^a = -\Riem_{cbd}{}^a\, X^b\, T^c\, T^d}
\end{equation}

\noindent
This is the \textbf{geodesic deviation equation} (also called
the \textbf{Jacobi equation}).

\begin{intuition}[Tidal forces]
  In general relativity, freely falling particles follow
  geodesics.  The geodesic deviation equation says that the
  relative acceleration between two nearby freely falling
  particles is proportional to the Riemann tensor.  This is the
  precise formulation of \emph{tidal forces}: the Riemann
  tensor is what an observer actually \emph{measures} as the
  differential gravitational acceleration across a small region
  of spacetime.
\end{intuition}

\begin{keyresult}[Curvature $\Leftrightarrow$ geodesic deviation]
  A manifold is flat ($\Riem_{abc}{}^d = 0$ everywhere) if and
  only if $A^a = 0$ for \emph{all} one-parameter families of
  geodesics.  Conversely, if there exist relatively
  accelerating geodesics, then $\Riem_{abc}{}^d \neq 0$: the
  manifold is curved.
\end{keyresult}

% ──────────────────────────────────────────────────────────────
\subsection{Computing curvature in coordinates}%
\label{sec:curvature-coordinates}

Given a manifold $\M$ with metric $g_{ab}$, the Riemann tensor
$\Riem_{abc}{}^d$ is uniquely determined by the Levi-Civita
connection~$\covd_a$.  We now derive the explicit coordinate
expression.

\subsubsection{Derivation}\label{sec:riemann-coord-derivation}

Choose a coordinate chart $(\psi, x^\mu)$.  For a covector
field~$\omega_c$, recall that the covariant derivative is
\[
  \covd_b \omega_c = \partial_b \omega_c
    - \chris{d}{bc}\, \omega_d\,.
\]
Applying a second covariant derivative:
\begin{align}
  \covd_a \covd_b \omega_c
    &= \partial_a(\partial_b \omega_c
       - \chris{d}{bc}\, \omega_d)
     - \chris{e}{ab}(\partial_e \omega_c
       - \chris{d}{ec}\, \omega_d)
     - \chris{e}{ac}(\partial_b \omega_e
       - \chris{d}{be}\, \omega_d)\,.
     \label{eq:double-covd-coords}
\end{align}
The Riemann tensor is obtained by antisymmetrising:
\[
  \Riem_{abc}{}^d\, \omega_d
    = 2\,\covd_{[a} \covd_{b]} \omega_c
    = (-2\,\partial_{[a} \chris{d}{b]c}
       + 2\,\chris{e}{c[a}\, \chris{d}{b]e})\,\omega_d\,.
\]
Since $\omega_d$ is arbitrary, we read off the coordinate
components:

\begin{equation}\label{eq:riemann-coord}
  \eqbox{\Riem^\sigma{}_{\mu\nu\rho}
    = \pd{\chris{\sigma}{\nu\rho}}{x^\mu}
    - \pd{\chris{\sigma}{\mu\rho}}{x^\nu}
    + \sum_\alpha
      \bigl(
        \chris{\alpha}{\nu\rho}\,\chris{\sigma}{\mu\alpha}
        - \chris{\alpha}{\mu\rho}\,\chris{\sigma}{\nu\alpha}
      \bigr)}
\end{equation}

\begin{remark}
  The Riemann tensor involves the Christoffel symbols and their
  first derivatives, hence the metric and its first \emph{and}
  second derivatives.  This is consistent with the fact that the
  Christoffel symbols involve first derivatives of the metric,
  and the Riemann tensor involves first derivatives of the
  Christoffel symbols.
\end{remark}

\subsubsection{Ricci tensor in coordinates}%
\label{sec:ricci-coord}

By contraction, the Ricci tensor in coordinates is
\begin{equation}\label{eq:ricci-coord}
  \Ric_{\mu\rho}
    = \sum_\nu \Riem^\nu{}_{\mu\nu\rho}
    = \sum_\nu \biggl(
        \pd{\chris{\nu}{\nu\rho}}{x^\mu}
        - \pd{\chris{\nu}{\mu\rho}}{x^\nu}
        + \sum_\alpha
          \bigl(
            \chris{\alpha}{\nu\rho}\,\chris{\nu}{\mu\alpha}
            - \chris{\alpha}{\mu\rho}\,\chris{\nu}{\nu\alpha}
          \bigr)
      \biggr)\,.
\end{equation}

% ──────────────────────────────────────────────────────────────
\subsection{Useful formulas}\label{sec:useful-formulas}

We collect several results that simplify practical computations
involving the Christoffel symbols.

\subsubsection{The metric determinant}\label{sec:metric-det}

Let $g = \det(g_{\mu\nu})$ denote the determinant of the metric
in a coordinate chart.  A standard identity from matrix calculus
gives:
\begin{equation}\label{eq:det-deriv}
  \sum_{\nu,\alpha} g^{\nu\alpha}\,
    \pd{g_{\nu\alpha}}{x^\mu}
    = \frac{1}{g}\,\pd{g}{x^\mu}
    = \pd{\log|g|}{x^\mu}\,.
\end{equation}

% ──────────────────────────────────────────────────────────────
\subsubsection{Trace of the Christoffel symbols}%
\label{sec:christoffel-trace}

\begin{theorem}\label{thm:christoffel-trace}
  The trace of the Christoffel symbols over the first and second
  indices is
  \begin{equation}\label{eq:christoffel-trace}
    \chris{\alpha}{\alpha\mu}
      = \sum_\nu \chris{\nu}{\nu\mu}
      = \frac{1}{2} \sum_{\nu,\alpha} g^{\nu\alpha}\,
          \pd{g_{\nu\alpha}}{x^\mu}
      = \pd{\log\sqrt{|g|}}{x^\mu}\,.
  \end{equation}
\end{theorem}

\begin{proof}
  From the definition of the Christoffel
  symbols~\eqref{eq:christoffel-trace}:
  \[
    \chris{\alpha}{\alpha\mu}
      = \frac{1}{2}\, g^{\alpha\sigma}\biggl(
          \pd{g_{\sigma\alpha}}{x^\mu}
          + \pd{g_{\sigma\mu}}{x^\alpha}
          - \pd{g_{\alpha\mu}}{x^\sigma}
        \biggr)\,.
  \]
  The last two terms cancel when summed over $\alpha$ (by
  relabelling $\alpha\leftrightarrow\sigma$ and using the
  symmetry $g^{\alpha\sigma} = g^{\sigma\alpha}$), leaving
  \[
    \chris{\alpha}{\alpha\mu}
      = \frac{1}{2}\, g^{\alpha\sigma}\,
          \pd{g_{\sigma\alpha}}{x^\mu}
      = \frac{1}{2g}\,\pd{g}{x^\mu}
      = \pd{\log\sqrt{|g|}}{x^\mu}\,.
      \qedhere
  \]
\end{proof}

% ──────────────────────────────────────────────────────────────
\subsubsection{Divergence formula}\label{sec:divergence-formula}

\begin{theorem}[Covariant divergence in coordinates]%
\label{thm:divergence-formula}
  For a vector field $T^a$, the covariant divergence is
  \begin{equation}\label{eq:divergence-formula}
    \eqbox{\covd_a T^a
      = \sum_\mu
        \frac{1}{\sqrt{|g|}}\,
        \pd{(\sqrt{|g|}\, T^\mu)}{x^\mu}}
  \end{equation}
\end{theorem}

\begin{proof}
  By definition,
  \[
    \covd_a T^a
      = \partial_a T^a + \chris{a}{ab}\, T^b
      = \sum_\mu \pd{T^\mu}{x^\mu}
        + \sum_\mu \chris{\mu}{\mu\nu}\, T^\nu\,.
  \]
  Substituting the Christoffel
  trace~\eqref{eq:christoffel-trace}:
  \[
    \covd_a T^a
      = \sum_\mu \pd{T^\mu}{x^\mu}
        + \sum_\nu
          \frac{1}{\sqrt{|g|}}\,
          \pd{\sqrt{|g|}}{x^\nu}\, T^\nu
      = \sum_\mu
        \frac{1}{\sqrt{|g|}}\,
        \pd{(\sqrt{|g|}\, T^\mu)}{x^\mu}\,,
  \]
  where the last step uses the product rule.
\end{proof}

\begin{remark}
  The divergence formula~\eqref{eq:divergence-formula} is
  extremely useful in practice.  It shows that the covariant
  divergence can be computed without explicitly evaluating
  Christoffel symbols: one needs only the metric determinant and
  partial derivatives.
\end{remark}

\begin{exercise}\label{ex:divergence-spherical}
  Using the divergence
  formula~\eqref{eq:divergence-formula}, compute $\covd_a T^a$
  for the metric of $\Rn{3}$ in spherical coordinates
  $(r,\theta,\varphi)$ and verify that it reproduces the
  standard expression for the divergence in spherical
  coordinates.
\end{exercise}
