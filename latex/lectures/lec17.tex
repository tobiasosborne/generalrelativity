%!TEX root = ../GeneralRelativity.tex
% ──────────────────────────────────────────────────────────────
%  Lecture 17 — The Friedmann equations
% ──────────────────────────────────────────────────────────────

\subsection{Solving Einstein's equations for FLRW}%
\label{sec:friedmann-derivation}

We now carry out the explicit calculation: starting from the FLRW
metric~\eqref{eq:flrw-unified}, we compute the Christoffel symbols,
the Ricci tensor, the Ricci scalar, and the Einstein tensor, then
substitute into the field
equations~\eqref{eq:friedmann-tt}--\eqref{eq:friedmann-ss} to obtain
the \textbf{Friedmann equations}---ordinary differential equations
governing the scale factor~$a(\tau)$.

Basis-free reasoning was invaluable for deducing the \emph{form} of
the metric, but it is hopeless for doing any real calculation.  To
model a physical process, we must choose coordinates and compute.

% ──────────────────────────────────────────────────────────────
\subsubsection{Christoffel symbols}%
\label{sec:flrw-christoffel}

For the FLRW metric~\eqref{eq:flrw-unified} in coordinates
$(\tau, r, \theta, \phi)$, a straightforward (if lengthy) exercise
yields the following non-vanishing Christoffel symbols.  We write
$\dot a = da/d\tau$.

\medskip\noindent\textbf{Temporal--spatial:}
\begin{equation}\label{eq:flrw-chris-1}
  \chris{\tau}{rr} = \frac{a\,\dot a}{1 - k\,r^2}\,,\qquad
  \chris{\tau}{\theta\theta} = a\,\dot a\, r^2\,,\qquad
  \chris{\tau}{\phi\phi} = a\,\dot a\, r^2 \sin^2\!\theta\,.
\end{equation}

\noindent\textbf{Spatial--temporal:}
\begin{equation}\label{eq:flrw-chris-2}
  \chris{r}{\tau r} = \chris{\theta}{\tau\theta}
    = \chris{\phi}{\tau\phi} = \frac{\dot a}{a}\,.
\end{equation}

\noindent\textbf{Purely spatial:}
\begin{equation}\label{eq:flrw-chris-3}
  \chris{r}{rr} = \frac{k\,r}{1 - k\,r^2}\,,\qquad
  \chris{r}{\theta\theta} = -r\,(1 - k\,r^2)\,,\qquad
  \chris{r}{\phi\phi} = -r\,(1 - k\,r^2)\sin^2\!\theta\,,
\end{equation}
\begin{equation}\label{eq:flrw-chris-4}
  \chris{\theta}{r\theta} = \chris{\phi}{r\phi}
    = \frac{1}{r}\,,\qquad
  \chris{\theta}{\phi\phi} = -\sin\theta\cos\theta\,,\qquad
  \chris{\phi}{\theta\phi} = \cot\theta\,.
\end{equation}
All other components either vanish or are determined by the symmetry
$\chris{\lambda}{\mu\nu} = \chris{\lambda}{\nu\mu}$.

\begin{remark}
  The individual Christoffel symbols are not difficult to compute;
  the main challenge is bookkeeping---detecting the inevitable sign
  errors and missed factors before they propagate.  Computer algebra
  (Mathematica, SageMath, etc.) is an invaluable tool for verifying
  these results.
\end{remark}

% ──────────────────────────────────────────────────────────────
\subsubsection{Ricci tensor and scalar}%
\label{sec:flrw-ricci}

From the Christoffel symbols one computes the Ricci tensor.  The
non-vanishing components are (exercise):
\begin{align}
  R_{\tau\tau} &= -3\,\frac{\ddot a}{a}\,,
    \label{eq:ricci-tt}\\[4pt]
  R_{rr} &= \frac{\ddot a\, a + 2\dot a^2 + 2k}{1 - k\,r^2}\,,
    \label{eq:ricci-rr}\\[4pt]
  R_{\theta\theta} &= (\ddot a\, a + 2\dot a^2 + 2k)\, r^2\,,
    \label{eq:ricci-thth}\\[4pt]
  R_{\phi\phi} &= (\ddot a\, a + 2\dot a^2 + 2k)\, r^2
    \sin^2\!\theta\,.
    \label{eq:ricci-phph}
\end{align}
Note that the Ricci tensor is diagonal, and the spatial components
all share the common factor $(\ddot a\, a + 2\dot a^2 + 2k)$
---confirming by direct calculation the isotropy that we
built in by assumption.

The Ricci scalar is obtained by tracing with the inverse metric:
\begin{equation}\label{eq:ricci-scalar-flrw}
  R = 6\biggl(\frac{\ddot a}{a}
    + \frac{\dot a^2}{a^2}
    + \frac{k}{a^2}\biggr)\,.
\end{equation}

% ──────────────────────────────────────────────────────────────
\subsubsection{Einstein tensor}%
\label{sec:flrw-einstein}

The Einstein tensor $\Ein_{\mu\nu} = R_{\mu\nu} - \frac{1}{2}\,
g_{\mu\nu}\, R$ has the following non-vanishing components
(exercise):
\begin{align}
  \Ein_{\tau\tau} &= 3\,\frac{\dot a^2}{a^2}
    + 3\,\frac{k}{a^2}\,,
    \label{eq:einstein-tt-flrw}\\[4pt]
  \Ein_{rr} &= \frac{-2\ddot a\, a - \dot a^2 - k}
    {1 - k\,r^2}\,,
    \label{eq:einstein-rr-flrw}\\[4pt]
  \Ein_{\theta\theta} &= r^2\bigl(-2\ddot a\, a
    - \dot a^2 - k\bigr)\,,
    \label{eq:einstein-thth-flrw}\\[4pt]
  \Ein_{\phi\phi} &= r^2 \sin^2\!\theta
    \bigl(-2\ddot a\, a - \dot a^2 - k\bigr)\,.
    \label{eq:einstein-phph-flrw}
\end{align}

\begin{exercise}\label{ex:flrw-einstein}
  Derive~\eqref{eq:einstein-tt-flrw}--\eqref{eq:einstein-phph-flrw}
  from~\eqref{eq:ricci-tt}--\eqref{eq:ricci-phph}
  and~\eqref{eq:ricci-scalar-flrw}.
\end{exercise}

% ──────────────────────────────────────────────────────────────
\subsubsection{The Friedmann equations}%
\label{sec:friedmann-equations}

Substituting the Einstein tensor and the perfect-fluid
stress-energy tensor~\eqref{eq:perfect-fluid-cosmo} into the field
equations~$\Ein_{ab} + \Lambda\, g_{ab} = 8\pi\, T_{ab}$:

\medskip\noindent
\textbf{$\tau$--$\tau$ component:}
\begin{equation}\label{eq:friedmann-1-prime}
  3\,\frac{\dot a^2}{a^2}
    + 3\,\frac{k}{a^2} - \Lambda = 8\pi\, \rho\,.
    \tag{i$'$}
\end{equation}

\noindent
\textbf{$r$--$r$ component} (after dividing out common factors):
\begin{equation}\label{eq:friedmann-2-prime}
  -2\,\frac{\ddot a}{a}
    - \frac{\dot a^2}{a^2}
    - \frac{k}{a^2} + \Lambda = 8\pi\, p\,.
    \tag{ii$'$}
\end{equation}
The $\theta$--$\theta$ and $\phi$--$\phi$ components yield the same
equation as~\eqref{eq:friedmann-2-prime}, confirming that only two
equations are independent.

By differentiating~\eqref{eq:friedmann-1-prime} and combining
with~\eqref{eq:friedmann-2-prime} to eliminate common terms, one
arrives at the equivalent---and more commonly cited---pair:

\begin{keyresult}[Friedmann equations]
  \begin{align}
    3\,\frac{\dot a^2}{a^2}
      &= 8\pi\,\rho - \frac{3k}{a^2} + \Lambda\,,
      \label{eq:friedmann-I}
      \tag{I}\\[6pt]
    3\,\frac{\ddot a}{a}
      &= -4\pi\,(\rho + 3p) + \Lambda\,.
      \label{eq:friedmann-II}
      \tag{II}
  \end{align}
\end{keyresult}

\noindent
Together with the \textbf{fluid equation} (which follows from
$\covd^a T_{ab} = 0$, or equivalently from differentiating~(I) and
substituting~(II)):
\begin{equation}\label{eq:fluid-equation}
  \dot\rho = -3\,\frac{\dot a}{a}\,(\rho + p)\,,
\end{equation}
these govern the evolution of the scale factor~$a(\tau)$ and the
energy density~$\rho(\tau)$ for any equation of state $p = p(\rho)$.

% ──────────────────────────────────────────────────────────────
\subsubsection{Absorbing the cosmological constant as dark energy}%
\label{sec:dark-energy}

Observe that the cosmological constant~$\Lambda$ can be
\emph{eliminated} from~\eqref{eq:friedmann-I}--\eqref{eq:friedmann-II}
by redefining the energy density and pressure:
\begin{equation}\label{eq:lambda-absorption}
  \rho \;\longmapsto\; \rho + \frac{\Lambda}{8\pi}\,,
  \qquad
  p \;\longmapsto\; p - \frac{\Lambda}{8\pi}\,.
\end{equation}
After this substitution, the Friedmann equations take the same form
but with $\Lambda = 0$.  In this way, the cosmological constant is
interpreted as a form of energy with \textbf{negative pressure}
equal in magnitude to its energy density:
\begin{equation}\label{eq:dark-energy-eos}
  p_\Lambda = -\rho_\Lambda\,.
\end{equation}
This exotic equation of state defines what is known as
\textbf{dark energy}.

% ──────────────────────────────────────────────────────────────
\subsection{Qualitative consequences of the Friedmann equations}%
\label{sec:friedmann-qualitative}

Before solving the Friedmann equations quantitatively, we can extract
striking qualitative predictions by simple inspection.

% ──────────────────────────────────────────────────────────────
\subsubsection{The universe cannot be static}%
\label{sec:not-static}

Suppose that $\rho > 0$ and $p \geq 0$ (i.e.\ we temporarily ignore
the cosmological constant / dark energy contribution).  Then
equation~\eqref{eq:friedmann-II} gives
\[
  \ddot a = -\frac{4\pi}{3}\,(\rho + 3p)\, a < 0\,.
\]
Since $a > 0$, the scale factor has strictly negative second
derivative: $a(\tau)$ is \emph{concave}.  In particular,
$\dot a \neq 0$---the universe must always be either expanding
($\dot a > 0$) or contracting ($\dot a < 0$).  A static universe
($\dot a = 0$, $\ddot a = 0$) is impossible for ordinary matter.

\begin{intuition}[Why Einstein introduced $\Lambda$]
  This was precisely the result that Einstein found unacceptable in
  1917: the prevailing belief was that the universe is eternal and
  static.  The cosmological constant term was introduced to balance
  the gravitational attraction and permit a static solution.  Of
  course, Hubble's observations later confirmed that the universe
  is indeed expanding.  One should note, however, that the principle
  of including all terms permitted by the symmetries of the theory
  is sound---the motivation may have been questionable, but the
  cosmological constant itself is a perfectly legitimate (and, as it
  turns out, physically realised) contribution to the field
  equations.
\end{intuition}

% ──────────────────────────────────────────────────────────────
\subsubsection{Hubble's law}%
\label{sec:hubble}

Let $\mathcal{R}$ denote the proper distance between two isotropic
observers at cosmological time~$\tau$.  Since the FLRW metric
rescales all spatial distances by the common factor~$a(\tau)$, we
have $\mathcal{R}(\tau) = a(\tau)\, d$, where $d$ is the
\emph{comoving distance} (constant in time).  Differentiating:
\begin{equation}\label{eq:hubble-law}
  \eqbox{v \equiv \dot{\mathcal{R}}
    = \frac{\dot a}{a}\,\mathcal{R}
    = H(\tau)\,\mathcal{R}}
\end{equation}
where
\begin{equation}\label{eq:hubble-parameter}
  H(\tau) = \frac{\dot a}{a}
\end{equation}
is the \textbf{Hubble parameter} (often called ``Hubble's constant,''
although it is \emph{not} constant in time---it is better described
as Hubble's \emph{function}).

Equation~\eqref{eq:hubble-law} is \textbf{Hubble's law}: the
recession velocity of a distant galaxy is proportional to its
distance, with proportionality constant~$H$.  For sufficiently large
separations, $v$ can exceed the speed of light.  This entails no
violation of relativity: in general relativity, only \emph{locally}
measured relative velocities (at a single event, between
infinitesimally close observers) are required not to exceed~$c$.
Global recession velocities, determined by the geodesic distance
between widely separated observers, are not so constrained.

\begin{intuition}[The balloon analogy]
  One should not picture the expansion as originating from a
  distinguished centre.  Rather, think of galaxies as dots painted
  on the surface of a balloon that is being inflated: all distances
  between dots increase uniformly, with no dot occupying a preferred
  position.  The expansion is an expansion \emph{of space itself},
  not an expansion of matter into pre-existing space.
\end{intuition}

% ──────────────────────────────────────────────────────────────
\subsubsection{The Big Bang singularity}%
\label{sec:big-bang}

Given that the universe is expanding ($\dot a > 0$) and that
$\ddot a < 0$ (for ordinary matter with $\rho + 3p > 0$), the
concavity of $a(\tau)$ has a dramatic consequence.  Even at
\emph{constant} expansion rate (which underestimates the past rate,
since the expansion was faster in the past), the scale factor would
reach zero at a finite time
\[
  T = \frac{a}{\dot a} = \frac{1}{H}\,.
\]
Since the actual expansion was faster in the past ($\ddot a < 0$
means $\dot a$ was larger), the time to $a = 0$ is even shorter
than $1/H$.

At $a = 0$, all spatial distances vanish, the energy density
diverges (assuming any matter is present), and the spacetime metric
becomes singular.  This is the \textbf{Big Bang}: the initial
singularity from which the universe began.

\begin{keyresult}[The Big Bang]
  In an expanding FLRW universe with $\rho + 3p > 0$, the scale
  factor $a(\tau)$ must have been zero at a finite time in the past.
  At this instant, all distances between events vanish and the
  energy density is infinite.  This is a genuine singularity of the
  spacetime: it is impossible to extend the manifold in a
  non-singular, connected way to ``before'' the Big Bang.
  The Big Bang is \emph{not} an explosion of pre-existing matter
  into pre-existing space---it is the origin of space and time
  themselves.
\end{keyresult}

\medskip
In the next lecture, we will solve the Friedmann equations
explicitly for various equations of state and then turn to the
final major topic of this course: the \textbf{Schwarzschild
solution}---an exact solution of Einstein's equations describing
the geometry outside a spherically symmetric, non-rotating mass.
