%!TEX root = ../GeneralRelativity.tex
% ──────────────────────────────────────────────────────────────
%  Lecture 10 — Geodesics continued; curvature
% ──────────────────────────────────────────────────────────────

\section{Geodesics continued; curvature}\label{sec:geodesics-curvature}

In the previous lectures we defined geodesics as curves whose
tangent vector is parallelly transported along itself.  In this
lecture we study the \emph{arc length} and \emph{proper time}
functionals, show that geodesics are \emph{extremal curves} of
these functionals, and then introduce the central object of
Riemannian geometry: the \textbf{Riemann curvature tensor}.

% ──────────────────────────────────────────────────────────────
\subsection{Arc length and proper time}\label{sec:arc-length}

Let $(\M,g_{ab})$ be a manifold with metric, and let $\covd_a$
be the Levi-Civita connection.  Consider a smooth curve~$C$
from a point~$p$ to a point~$q$ in~$\M$, parametrised by~$t$,
with tangent vector~$T^a$.

% Diagram: curve C from p to q with tangent vector
\begin{center}
\begin{tikzpicture}[scale=0.9]
  % Manifold blob
  \draw[thick, spacecadet, rounded corners=14pt]
    (0,0.3) to[out=25,in=160] (6.5,0.6)
    to[out=-20,in=50] (7.0,-1.0)
    to[out=230,in=-10] (0.5,-1.2)
    to[out=170,in=250] cycle;
  \node[font=\sf\small, text=cgblue] at (6.8,0.9) {$\M$};
  % Labels for metric and connection
  \node[font=\small, text=spacecadet] at (1.5,0.7) {$g_{ab},\;\covd_a$};
  % Curve
  \draw[thick, munsell]
    (1.2,-0.3) to[out=15,in=200] (3.5,0.1)
    to[out=20,in=190] (5.8,-0.1);
  \node[font=\scriptsize, munsell, below] at (3.5,-0.2) {$C$};
  % Point p
  \node[circle, fill=banana, inner sep=1.8pt] (P) at (1.2,-0.3) {};
  \node[font=\small, below left] at (P) {$p$};
  % Point q
  \node[circle, fill=banana, inner sep=1.8pt] (Q) at (5.8,-0.1) {};
  \node[font=\small, below right] at (Q) {$q$};
  % Tangent vector at midpoint
  \node[circle, fill=banana, inner sep=1.2pt] (M) at (3.5,0.1) {};
  \draw[vecstyle] (M) -- ++(0.8,0.12)
    node[above, font=\small]{$T^a$};
\end{tikzpicture}
\end{center}

\begin{definition}[Arc length]\label{def:arc-length}
  Assume for now that $g_{ab}$ has Riemannian signature
  $(+,+,\dots,+)$.  The \textbf{length} (or \textbf{arc length})
  of~$C$ with respect to~$g_{ab}$ is
  \begin{equation}\label{eq:arc-length}
    \eqbox{\ell = \int_a^b \sqrt{g_{ab}\,T^a T^b}\;\dd t}\,,
  \end{equation}
  where $T^a$ is the tangent vector to~$C$ and $t$ is the curve
  parameter, with $C(a)=p$ and $C(b)=q$.
\end{definition}

For a Lorentzian metric with signature $(-,+,+,\dots,+)$, the
quantity $g_{ab}\,T^a T^b$ need not be positive.  We
distinguish three cases:

\begin{definition}[Causal character of curves]%
\label{def:causal-character}
  Let $C$ be a smooth curve with tangent~$T^a$ on a Lorentzian
  manifold $(\M,g_{ab})$.
  \begin{itemize}
    \item $C$ is \textbf{timelike} if
      $g_{ab}\,T^a T^b < 0$ everywhere along~$C$.
    \item $C$ is \textbf{null} (or \textbf{lightlike}) if
      $g_{ab}\,T^a T^b = 0$ everywhere along~$C$.
    \item $C$ is \textbf{spacelike} if
      $g_{ab}\,T^a T^b > 0$ everywhere along~$C$.
  \end{itemize}
\end{definition}

\begin{definition}[Proper time]\label{def:proper-time}
  For a \textbf{timelike} curve~$C$ on a Lorentzian manifold,
  the \textbf{proper time} along~$C$ is
  \begin{equation}\label{eq:proper-time}
    \eqbox{\tau = \int_a^b \sqrt{-g_{ab}\,T^a T^b}\;\dd t}\,.
  \end{equation}
\end{definition}

A key observation is that the causal character of a geodesic
cannot change.

\begin{theorem}[Constancy of the norm along a geodesic]%
\label{thm:geodesic-norm-constant}
  Let $C$ be a geodesic with tangent~$T^a$.  Then the quantity
  $g_{ab}\,T^a T^b$ is constant along~$C$.  In particular, a
  geodesic cannot change from timelike to null, or from null to
  spacelike, etc.
\end{theorem}

\begin{proof}
  Compute the derivative of the norm along~$C$:
  \begin{align}
    T^c\covd_c\bigl(g_{ab}\,T^a T^b\bigr)
      &= (T^c\covd_c g_{ab})\,T^a T^b
        + g_{ab}\,(T^c\covd_c T^a)\,T^b
        + g_{ab}\,T^a\,(T^c\covd_c T^b)\,.
        \label{eq:norm-deriv}
  \end{align}
  The first term vanishes because the Levi-Civita connection is
  metric-compatible: $\covd_c g_{ab} = 0$.  The second and third
  terms vanish by the geodesic equation $T^c\covd_c T^a = 0$.
  Hence
  \[
    T^c\covd_c\bigl(g_{ab}\,T^a T^b\bigr) = 0\,,
  \]
  i.e.\ the inner product $(T,T) = g_{ab}\,T^a T^b$ is constant
  along the geodesic.
\end{proof}

\begin{intuition}[Causal character is preserved]
  Since for a geodesic $T^a$ is parallelly transported along
  itself, and parallel transport (with respect to a
  metric-compatible connection) preserves inner products, the
  ``squared speed'' $g_{ab}\,T^a T^b$ cannot change sign.  A
  geodesic that starts timelike remains timelike forever.
\end{intuition}

% ──────────────────────────────────────────────────────────────
\subsection{Reparametrisation invariance}%
\label{sec:reparam-invariance}

The arc length and proper time are \textbf{reparametrisation
invariant}: they do not depend on the choice of curve parameter.

\begin{theorem}[Reparametrisation invariance of arc length]%
\label{thm:reparam-invariance}
  Let $s = s(t)$ be a monotone reparametrisation of~$C$, and
  let $S^a$ denote the tangent vector with respect to~$s$.  Then
  $\ell$ computed using parameter~$s$ equals~$\ell$ computed
  using parameter~$t$.
\end{theorem}

\begin{proof}
  The new tangent vector is related to the old one by
  \[
    S^a = \td{t}{s}\, T^a\,.
  \]
  Therefore
  \begin{align}
    \ell'
      &= \int \sqrt{g_{ab}\,S^a S^b}\;\dd s
       = \int \sqrt{g_{ab}\,T^a T^b}\,
         \left|\td{t}{s}\right|\;\dd s
       = \int \sqrt{g_{ab}\,T^a T^b}\;\dd t
       = \ell\,.
         \notag
  \end{align}
  The same argument applies to the proper time~$\tau$.
\end{proof}

% ──────────────────────────────────────────────────────────────
\subsection{Geodesics as extremal curves}%
\label{sec:geodesics-extremal}

We now prove the fundamental result connecting geodesics to the
calculus of variations.

\begin{theorem}[Geodesics extremise arc length]%
\label{thm:geodesics-extremise}
  A curve~$C$ joining two points $p$ and~$q$ extremises the arc
  length functional~$\ell$ if and only if $C$ is a geodesic.
\end{theorem}

\noindent\textbf{Setup.}\;
Let $p$ and~$q$ lie in a common chart
$\psi\colon\M\to\Rn{n}$, and assume~$C$ is spacelike.  In
coordinates, the arc length is
\begin{equation}\label{eq:arc-length-coord}
  \ell = \int_a^b
    \sqrt{\sum_{\mu,\nu}
      g_{\mu\nu}\,\frac{\dd x^\mu}{\dd t}\,
        \frac{\dd x^\nu}{\dd t}}
    \;\dd t\,,
  \qquad C(a)=p,\; C(b)=q\,.
\end{equation}

% Diagram: curve C from p to q on manifold and in chart
\begin{center}
\begin{tikzpicture}[scale=0.85]
  % Manifold
  \draw[thick, spacecadet, rounded corners=12pt]
    (0,0.3) to[out=30,in=160] (4.0,0.5)
    to[out=-20,in=30] (4.2,-0.6)
    to[out=210,in=-10] (0.2,-0.5) -- cycle;
  \node[font=\sf\small, text=cgblue] at (3.8,0.8) {$\M$};
  % Curve
  \draw[thick, munsell]
    (0.8,-0.1) to[out=20,in=200] (2.0,0.15)
    to[out=20,in=190] (3.2,0.0);
  \node[circle, fill=banana, inner sep=1.5pt] (P) at (0.8,-0.1) {};
  \node[font=\small, below left] at (P) {$p$};
  \node[circle, fill=banana, inner sep=1.5pt] (Q) at (3.2,0.0) {};
  \node[font=\small, below right] at (Q) {$q$};
  \node[font=\scriptsize, munsell] at (2.0,-0.2) {$C$};
  % Arrow psi
  \draw[thick, -{Stealth[length=6pt]}, cgblue]
    (4.6,0.0) -- (6.0,0.0)
    node[midway, above, font=\small]{$\psi$};
  % R^n
  \begin{scope}[shift={(6.5,-0.5)}]
    \draw[axisstyle, spacecadet] (0,-0.5) -- (0,1.3);
    \draw[axisstyle, spacecadet] (-0.3,0) -- (2.5,0);
    \node[font=\sf\small, text=spacecadet] at (2.2,1.1) {$\Rn{n}$};
    \draw[thick, munsell]
      (0.3,0.2) to[out=30,in=210] (1.2,0.7)
      to[out=30,in=200] (2.0,0.5);
    \node[circle, fill=banana, inner sep=1.2pt] at (0.3,0.2) {};
    \node[font=\scriptsize, below left] at (0.3,0.2) {$\psi(p)$};
    \node[circle, fill=banana, inner sep=1.2pt] at (2.0,0.5) {};
    \node[font=\scriptsize, below right] at (2.0,0.5) {$\psi(q)$};
  \end{scope}
\end{tikzpicture}
\end{center}

\subsubsection{Variational derivation}\label{sec:variational-derivation}

Consider an infinitesimal variation of the curve:
\[
  x^\mu(t) \;\longmapsto\; x^\mu(t) + \delta x^\mu(t)\,,
  \qquad \delta x^\mu(a) = \delta x^\mu(b) = 0\,.
\]
The endpoints are held fixed.  We assume a parametrisation with
\textbf{unit speed}: $g_{ab}\,T^a T^b = 1$.

\begin{proof}[Proof of Theorem~\ref{thm:geodesics-extremise}]
  The extremality condition $\delta\ell = 0$ for all variations
  $\delta x^\alpha$ yields, after integration by parts,
  \begin{equation}\label{eq:euler-lagrange-length}
    0 = -\sum_\alpha g_{\alpha\beta}\,
        \frac{\dd^2 x^\alpha}{\dd t^2}
      - \sum_{\alpha,\lambda}
        \pd{g_{\alpha\beta}}{x^\lambda}\,
        \frac{\dd x^\alpha}{\dd t}\,
        \frac{\dd x^\lambda}{\dd t}
      + \frac{1}{2}\sum_{\alpha,\lambda}
        \pd{g_{\alpha\lambda}}{x^\beta}\,
        \frac{\dd x^\alpha}{\dd t}\,
        \frac{\dd x^\lambda}{\dd t}\,.
  \end{equation}
  Multiplying through by $g^{\mu\beta}$ and using the formula
  for the Christoffel symbols, this is equivalent to
  \begin{equation}\label{eq:geodesic-from-variation}
    \eqbox{\frac{\dd^2 x^\mu}{\dd t^2}
      + \sum_{\alpha,\lambda}
        \chris{\mu}{\alpha\lambda}\,
        \frac{\dd x^\alpha}{\dd t}\,
        \frac{\dd x^\lambda}{\dd t}
      = 0}
  \end{equation}
  This is precisely the \textbf{geodesic equation}.  Hence a
  curve extremises arc length if and only if it is a geodesic.
\end{proof}

\begin{keyresult}[Geodesics from the Lagrangian]
  The geodesic equation can also be obtained by applying
  the Euler--Lagrange equations to the \textbf{Lagrangian}
  \begin{equation}\label{eq:geodesic-lagrangian}
    L = \sum_{\mu,\nu}
      g_{\mu\nu}\,\frac{\dd x^\mu}{\dd t}\,
        \frac{\dd x^\nu}{\dd t}\,,
  \end{equation}
  via $\partial L/\partial x^\alpha
    - \dd/\dd t(\partial L/\partial\dot{x}^\alpha) = 0$.
  This provides a very efficient method for computing the
  Christoffel symbols: one simply reads them off from the
  Euler--Lagrange equations.
\end{keyresult}

\begin{exercise}\label{ex:geodesic-lagrangian-sphere}
  Use the Lagrangian method to derive the geodesic equations on
  the sphere $S^2$ with metric
  $g = \dd\theta\tp\dd\theta
    + \sin^2\!\theta\;\dd\varphi\tp\dd\varphi$,
  and read off the Christoffel symbols.
\end{exercise}

% ──────────────────────────────────────────────────────────────
\subsection{Extrema versus minima}\label{sec:extrema-vs-minima}

\begin{remark}
  For a Riemannian metric (positive-definite signature), a
  geodesic locally \emph{minimises} arc length.  The situation
  is more subtle for Lorentzian signature: a timelike curve
  joining two points may have \textbf{arbitrarily small} proper
  time (consider a ``zigzag'' path close to null segments).  If
  a curve of \emph{greatest} proper time between two events
  exists, it must be a timelike geodesic---but a timelike
  geodesic need not maximise proper time globally (it is only a
  local extremum).
\end{remark}

\begin{intuition}[The twin paradox revisited]
  In special relativity, the twin who travels along a
  non-geodesic (accelerated) worldline accumulates less proper
  time than the twin who follows the geodesic (inertial)
  worldline.  This is because the inertial worldline
  \emph{maximises} proper time among all timelike curves joining
  the same two events.
\end{intuition}

% ══════════════════════════════════════════════════════════════
%  CURVATURE
% ══════════════════════════════════════════════════════════════

% ──────────────────────────────────────────────────────────────
\subsection{Curvature: the Riemann tensor}%
\label{sec:riemann-tensor}

We now turn to the second major topic of this lecture:
\textbf{curvature}.

The curvature of a manifold~$\M$ equipped with a connection
$\covd_a$ measures the \textbf{path-dependence of parallel
transport}.  On a flat manifold, parallel-transporting a vector
around a closed loop returns the original vector.  On a curved
manifold, the transported vector generally \emph{differs} from
the original.  The Riemann curvature tensor provides a precise,
intrinsic measure of this failure.

\begin{intuition}[Curvature as non-commutativity]
  The Riemann curvature tensor measures the failure of
  successive covariant derivatives to commute.  In flat space,
  $\covd_a\covd_b = \covd_b\covd_a$ when acting on any tensor;
  curvature is the obstruction to this commutativity.
\end{intuition}

% ──────────────────────────────────────────────────────────────
\subsubsection{The commutator of covariant derivatives on
  dual vectors}\label{sec:commutator-covd}

We begin by studying the action of the commutator
$\covd_a\covd_b - \covd_b\covd_a$ on an arbitrary dual vector
field $\omega_c$.

\begin{theorem}\label{thm:commutator-local}
  For any $f\in\mathscr{F}(\M)$ and any dual vector
  field~$\omega_c$,
  \begin{equation}\label{eq:commutator-f-omega}
    (\covd_a\covd_b - \covd_b\covd_a)(f\omega_c)
      = f\,(\covd_a\covd_b - \covd_b\covd_a)\omega_c\,.
  \end{equation}
  Consequently, $(\covd_a\covd_b - \covd_b\covd_a)\omega_c$
  depends only on the value of~$\omega_c$ at the point~$p$.
\end{theorem}

\begin{proof}
  Apply the Leibniz rule:
  \begin{align}
    \covd_a\covd_b(f\omega_c)
      &= \covd_a\bigl((\covd_b f)\,\omega_c
        + f\,\covd_b\omega_c\bigr) \notag\\
      &= (\covd_a\covd_b f)\,\omega_c
        + (\covd_b f)\,\covd_a\omega_c
        + (\covd_a f)\,\covd_b\omega_c
        + f\,\covd_a\covd_b\omega_c\,.
        \label{eq:commutator-expand-ab}
  \end{align}
  The corresponding expression with $a$ and $b$ swapped is
  \begin{equation}\label{eq:commutator-expand-ba}
    \covd_b\covd_a(f\omega_c)
      = (\covd_b\covd_a f)\,\omega_c
        + (\covd_a f)\,\covd_b\omega_c
        + (\covd_b f)\,\covd_a\omega_c
        + f\,\covd_b\covd_a\omega_c\,.
  \end{equation}
  Subtracting~\eqref{eq:commutator-expand-ba}
  from~\eqref{eq:commutator-expand-ab} and using the
  torsion-free condition
  $\covd_a\covd_b f = \covd_b\covd_a f$ (axiom~(5)):
  \[
    (\covd_a\covd_b - \covd_b\covd_a)(f\omega_c)
      = f\,(\covd_a\covd_b - \covd_b\covd_a)\omega_c\,.
  \]
  Since the operator $(\covd_a\covd_b - \covd_b\covd_a)$ is
  $\mathscr{F}(\M)$-linear on dual vector fields, by the same
  argument as in Theorem~\ref{thm:diff-local} (localisation), it
  depends only on the pointwise value of~$\omega_c$.
\end{proof}

% ──────────────────────────────────────────────────────────────
\subsubsection{Definition of the Riemann tensor}%
\label{sec:riemann-def}

Since the map
$\omega_c \mapsto (\covd_a\covd_b - \covd_b\covd_a)\omega_c$
is a linear map $\ttype{0}{1}\to\ttype{0}{3}$ that depends only
on the pointwise value of~$\omega_c$, there exists a tensor
$\Riem_{abc}{}^d\in\ttype{1}{3}$ such that:

\begin{definition}[Riemann curvature tensor]\label{def:riemann}
  The \textbf{Riemann curvature tensor}
  $\Riem_{abc}{}^d$ is defined by
  \begin{equation}\label{eq:riemann-def}
    \eqbox{(\covd_a\covd_b - \covd_b\covd_a)\omega_c
      = \Riem_{abc}{}^d\,\omega_d}
  \end{equation}
  for all dual vector fields~$\omega_c$.
\end{definition}

\begin{remark}
  The Riemann tensor is a tensor of type~$(1,3)$.  It is
  determined entirely by the derivative operator $\covd_a$ (and
  hence, when $\covd_a$ is the Levi-Civita connection, entirely
  by the metric $g_{ab}$).  If $\Riem_{abc}{}^d = 0$, the
  connection is \textbf{flat}: covariant derivatives commute on
  all tensor fields, and parallel transport is path-independent.
\end{remark}

\begin{exercise}\label{ex:riemann-on-vectors}
  Show that for a vector field $v^c$:
  \[
    (\covd_a\covd_b - \covd_b\covd_a)v^c
      = -\Riem_{abd}{}^c\,v^d\,.
  \]
  \emph{Hint:} Apply the commutator to the scalar
  $\omega_c\,v^c$ and use the Leibniz rule.
\end{exercise}

% ──────────────────────────────────────────────────────────────
\subsection{Curvature and parallel transport around a loop}%
\label{sec:curvature-loop}

We now give a geometric interpretation of the Riemann tensor as
measuring the change in a vector under parallel transport
around an infinitesimal closed loop.

% Diagram: infinitesimal loop on 2D surface
\begin{center}
\begin{tikzpicture}[scale=1.1]
  % Surface patch
  \draw[thick, spacecadet, rounded corners=6pt,
        fill=isabelline!40]
    (-0.3,-0.3) -- (4.3,-0.3) -- (4.3,3.3) -- (-0.3,3.3)
    -- cycle;
  \node[font=\sf\small, text=cgblue] at (4.6,3.1) {$S$};
  % Coordinate labels
  \node[font=\small, text=spacecadet] at (4.6,-0.1) {$s$};
  \node[font=\small, text=spacecadet] at (-0.5,3.1) {$t$};
  % Loop
  \draw[ultra thick, munsell, -{Stealth[length=6pt]}]
    (0.5,0.5) -- (0.5,2.5)
    node[midway, left, font=\scriptsize]{(i)};
  \draw[ultra thick, munsell, -{Stealth[length=6pt]}]
    (0.5,2.5) -- (3.5,2.5)
    node[midway, above, font=\scriptsize]{(ii)};
  \draw[ultra thick, munsell, -{Stealth[length=6pt]}]
    (3.5,2.5) -- (3.5,0.5)
    node[midway, right, font=\scriptsize]{(iii)};
  \draw[ultra thick, munsell, -{Stealth[length=6pt]}]
    (3.5,0.5) -- (0.5,0.5)
    node[midway, below, font=\scriptsize]{(iv)};
  % Corner labels
  \node[circle, fill=banana, inner sep=1.8pt] (A) at (0.5,0.5) {};
  \node[font=\scriptsize, below left] at (A)
    {$(0,0)$};
  \node[circle, fill=banana, inner sep=1.2pt] at (0.5,2.5) {};
  \node[font=\scriptsize, above left] at (0.5,2.5)
    {$(0,\Delta t)$};
  \node[circle, fill=banana, inner sep=1.2pt] at (3.5,2.5) {};
  \node[font=\scriptsize, above right] at (3.5,2.5)
    {$(\Delta s,\Delta t)$};
  \node[circle, fill=banana, inner sep=1.2pt] at (3.5,0.5) {};
  \node[font=\scriptsize, below right] at (3.5,0.5)
    {$(\Delta s,0)$};
  % Vector at p
  \draw[vecstyle, spacecadet] (A) -- ++(0.4,0.6)
    node[above right, font=\small]{$v^a$};
\end{tikzpicture}
\end{center}

Consider a two-dimensional surface~$S\subset\M$ parametrised by
coordinates $(s,t)$.  Let $T^a = (\partial/\partial t)^a$ and
$S^a = (\partial/\partial s)^a$ be the coordinate tangent
vectors.  We parallel-transport a vector $v^a$ around the
infinitesimal loop:
\[
  (0,0) \;\xrightarrow{\;\text{(i)}\;}\; (0,\Delta t)
  \;\xrightarrow{\;\text{(ii)}\;}\; (\Delta s,\Delta t)
  \;\xrightarrow{\;\text{(iii)}\;}\; (\Delta s,0)
  \;\xrightarrow{\;\text{(iv)}\;}\; (0,0)\,.
\]
We test the change in~$v^a$ by contracting with an arbitrary
dual vector~$\omega_a$.  Since $v^a$ is parallelly transported,
only $\omega_a$ varies along each leg.

\medskip\noindent
\textbf{Legs (i) and (iii).}\;
Along leg~(i) (transport in the $t$-direction at $s=0$), the
change in $v^a\omega_a$ is
\[
  \delta_1 = \Delta t \cdot T^b\covd_b(v^a\omega_a)
    \Big|_{(\Delta t/2,\,0)}
  = \Delta t \cdot v^a\,T^b\covd_b\omega_a
    \Big|_{(\Delta t/2,\,0)}\,,
\]
where we used $T^b\covd_b v^a = 0$ (parallel transport).
Along leg~(iii) (transport in the $(-t)$-direction at
$s=\Delta s$):
\[
  \delta_3 = -\Delta t \cdot v^a\,T^b\covd_b\omega_a
    \Big|_{(\Delta t/2,\,\Delta s)}\,.
\]
Combining:
\begin{equation}\label{eq:delta13}
  \delta_1 + \delta_3
    = \Delta t\Bigl(
        v^a\,T^b\covd_b\omega_a\Big|_{(\Delta t/2,\,0)}
      - v^a\,T^b\covd_b\omega_a\Big|_{(\Delta t/2,\,\Delta s)}
      \Bigr)
    = -\Delta s\,\Delta t\;
      S^c\covd_c\bigl(v^a\,T^b\covd_b\omega_a\bigr)\,.
\end{equation}

\medskip\noindent
\textbf{Legs (ii) and (iv).}\;
By an analogous calculation (transport in the $s$- and
$(-s)$-directions):
\begin{equation}\label{eq:delta24}
  \delta_2 + \delta_4
    = \Delta s\,\Delta t\;
      T^c\covd_c\bigl(v^a\,S^b\covd_b\omega_a\bigr)\,.
\end{equation}

\medskip\noindent
\textbf{Total change.}\;
The net change in $v^a\omega_a$ around the loop is
\begin{align}
  \delta(v^a\omega_a)
    &= (\delta_1+\delta_3) + (\delta_2+\delta_4) \notag\\
    &= \Delta s\,\Delta t\; v^a
      \bigl(
        T^c\,S^b\,\covd_c\covd_b\omega_a
        - S^c\,T^b\,\covd_c\covd_b\omega_a
      \bigr) \notag\\
    &= \Delta s\,\Delta t\; v^a\,T^c\,S^b\,
      (\covd_c\covd_b - \covd_b\covd_c)\omega_a\,.
      \label{eq:total-change}
\end{align}
By the definition of the Riemann
tensor~\eqref{eq:riemann-def}:
\begin{equation}\label{eq:loop-riemann}
  \delta(v^a\omega_a)
    = \Delta s\,\Delta t\;
      v^a\,T^c\,S^b\,\Riem_{cba}{}^d\,\omega_d\,.
\end{equation}
Since this holds for \emph{all} dual vectors~$\omega_a$, we
conclude:

\begin{keyresult}[Parallel transport around an infinitesimal
  loop]\label{res:loop-transport}
  The change in a vector $v^a$ after parallel transport around
  an infinitesimal loop spanned by $T^a$ and $S^a$ is
  \begin{equation}\label{eq:delta-v-loop}
    \eqbox{\delta v^a
      = \Delta s\,\Delta t\;
        v^d\,T^c\,S^b\,\Riem_{cbd}{}^a}
  \end{equation}
\end{keyresult}

\begin{remark}
  Equation~\eqref{eq:delta-v-loop} is the precise statement
  that curvature equals the path-dependence of parallel
  transport.  On a flat manifold, $\Riem_{abc}{}^d = 0$ and
  $\delta v^a = 0$: parallel transport around any closed loop
  returns the original vector.
\end{remark}

\begin{exercise}\label{ex:riemann-sphere-holonomy}
  Consider the unit sphere $S^2$ with the round metric.
  Parallel-transport a vector around the triangle formed by the
  equator from $\varphi=0$ to $\varphi=\pi/2$, the meridian
  from the equator to the north pole, and the meridian from the
  north pole back to the starting point.  Verify that the vector
  rotates by~$\pi/2$, consistent with the curvature of~$S^2$.
\end{exercise}

\begin{exercise}\label{ex:riemann-christoffel}
  Show that in a coordinate chart, the components of the
  Riemann tensor are given by
  \[
    \Riem^\rho{}_{\sigma\mu\nu}
      = \pd{\chris{\rho}{\nu\sigma}}{x^\mu}
      - \pd{\chris{\rho}{\mu\sigma}}{x^\nu}
      + \chris{\rho}{\mu\lambda}\,\chris{\lambda}{\nu\sigma}
      - \chris{\rho}{\nu\lambda}\,\chris{\lambda}{\mu\sigma}\,.
  \]
  \emph{Hint:} Apply the
  definition~\eqref{eq:riemann-def} to~$\omega_c$
  expressed in the coordinate basis and use the formula for
  $\covd_\mu\omega_\nu$ in terms of Christoffel symbols.
\end{exercise}
