%!TEX root = ../GeneralRelativity.tex
% ──────────────────────────────────────────────────────────────
%  Lecture 23 — Retrospective and outlook
% ──────────────────────────────────────────────────────────────

\section{Retrospective and outlook}%
\label{sec:retrospective}

We close these lecture notes by stepping back from the details to
survey the logical architecture of general relativity, to appreciate
what we have built, and to glimpse what lies beyond.

% ──────────────────────────────────────────────────────────────
\subsection{The logical structure of the theory}%
\label{sec:logical-structure}

General relativity rests on a remarkably small number of physical
principles.  The entire theory can be summarised in four lines:

\begin{keyresult}[The axioms of general relativity]
  \begin{enumerate}
    \item[\textup{(i)}] Spacetime is a four-dimensional Lorentzian
      manifold $(\M, g_{ab})$.
    \item[\textup{(ii)}] Freely falling test bodies move on
      geodesics of $g_{ab}$.
    \item[\textup{(iii)}] The laws of physics, expressed as tensor
      equations, take the same form in all coordinate systems
      (\emph{general covariance}).
    \item[\textup{(iv)}] The metric is determined by the matter
      content via Einstein's field equation:
      $\Ein_{ab} + \Lambda\, g_{ab} = 8\pi\, T_{ab}$.
  \end{enumerate}
\end{keyresult}

\noindent
From these four statements, the entire course follows.
Axiom~(i) required us to develop the theory of manifolds, tangent
spaces, and tensor fields (Lectures~3--6).  Axiom~(ii) led us to
connections, parallel transport, and the geodesic equation
(Lectures~7--10).  The requirement that the connection be metric-compatible
and torsion-free singled out the Levi-Civita connection and its
curvature, the Riemann tensor (Lectures~10--11).  Axiom~(iii) was
our guiding principle throughout, ensuring that every equation we
wrote was intrinsic.  And axiom~(iv)---arrived at by demanding
consistency with the Bianchi identity and the Newtonian
limit---gave us the field equations whose solutions occupied the
second half of the course (Lectures~12--22).

\begin{intuition}[The miracle of self-consistency]
  The most striking feature of this logical structure is its
  \emph{self-consistency}.  The geodesic hypothesis~(ii) was an
  \emph{input} to the theory---we assumed it as a physical postulate
  to motivate the field equations.  But Einstein's field
  equations~(iv), via the Bianchi identity
  $\covd^a \Ein_{ab} = 0$, automatically imply stress-energy
  conservation $\covd^a T_{ab} = 0$, which in turn implies the
  geodesic motion of pressureless matter
  (\S\ref{sec:geodesic-self-consistency}).  The postulate is
  recovered as a \emph{theorem}.  This kind of bootstrapping is
  extraordinarily rare in physics and is one of the strongest pieces
  of internal evidence that the theory is on the right track.
\end{intuition}

% ──────────────────────────────────────────────────────────────
\subsection{What we solved}%
\label{sec:what-we-solved}

The general Einstein equations are ten coupled, nonlinear, second-order
partial differential equations---a system of formidable complexity.
We attacked them through three strategies:

\begin{enumerate}
  \item \textbf{Perturbation theory} (Lectures~12--14).
    Linearising around flat spacetime reduced the field equations to
    a wave equation, yielding the Newtonian limit ($\phi = -M/r$),
    gravitational radiation (two polarizations propagating at~$c$),
    and the gauge structure of the theory.

  \item \textbf{Maximal symmetry} (Lectures~14--18).
    Imposing homogeneity and isotropy reduced the ten PDEs to two
    ODEs---the Friedmann equations---for a single function $a(\tau)$.
    The solutions describe the Big Bang, the expansion of the
    universe, and (for $k = +1$) the Big Crunch.

  \item \textbf{Spherical symmetry} (Lectures~18--22).
    Imposing staticity and spherical symmetry reduced the equations
    to two functions of one variable, $f(r)$ and $h(r)$, which we
    solved \emph{exactly}: the Schwarzschild metric.  Its geodesics
    gave us the precession of Mercury's perihelion ($43''$/century)
    and the deflection of starlight by the Sun ($1.75''$).
\end{enumerate}

\noindent
In each case, symmetry was the key that unlocked the equations.
Without it, we would have been helpless.

% ──────────────────────────────────────────────────────────────
\subsection{The experimental pillars}%
\label{sec:experimental-pillars}

General relativity is not merely a mathematical edifice; it is a
physical theory, and its predictions have been confirmed with
extraordinary precision:

\begin{center}
\renewcommand{\arraystretch}{1.5}
\begin{tabular}{lll}
  \toprule
  \textbf{Prediction} & \textbf{Confirmation} &
    \textbf{Precision}\\
  \midrule
  Perihelion precession of Mercury &
    Le Verrier (1859), Einstein (1915) &
    $\sim 0.1\%$ \\
  Deflection of starlight &
    Eddington (1919), radio VLBI &
    $\sim 0.01\%$ \\
  Gravitational redshift &
    Pound--Rebka (1960), GPS &
    $\sim 10^{-4}$ \\
  Shapiro time delay &
    Cassini spacecraft (2003) &
    $\sim 10^{-5}$ \\
  Gravitational waves &
    LIGO/Virgo (2015--) &
    waveform match \\
  Expansion of the universe &
    Hubble (1929), Planck (2018) &
    $\sim 1\%$ \\
  Black hole shadow &
    Event Horizon Telescope (2019) &
    image \\
  \bottomrule
\end{tabular}
\end{center}

\noindent
Every prediction of general relativity that has been tested has been
confirmed.  No other classical theory of gravity comes close to this
record.

% ──────────────────────────────────────────────────────────────
\subsection{What we did not cover}%
\label{sec:what-we-missed}

A one-semester course can only scratch the surface.  Major topics
that await the reader in more advanced treatments include:

\begin{itemize}
  \item \textbf{Black holes.}
    The Schwarzschild coordinate singularity at $r = 2M$ conceals
    the rich causal structure of black holes: the event horizon,
    the Penrose diagram, the Kerr solution for rotating black holes,
    the laws of black hole mechanics, and Hawking radiation.
  \item \textbf{Gravitational wave physics.}
    Beyond the linearised theory: post-Newtonian approximations,
    numerical relativity, the two-body problem, and the extraction
    of astrophysical parameters from LIGO/Virgo signals.
  \item \textbf{Singularity theorems.}
    The Penrose--Hawking theorems show that singularities are
    \emph{generic} in general relativity, not artefacts of high
    symmetry.  Their proof requires the theory of causal structure
    and the Raychaudhuri equation.
  \item \textbf{Initial value formulation.}
    General relativity admits a well-posed initial value problem
    (the ADM formalism), which is the foundation for numerical
    relativity and the mathematical study of existence and uniqueness
    of solutions.
  \item \textbf{Cosmological perturbation theory.}
    The seeds of structure in the universe---galaxies, clusters,
    the CMB anisotropies---arise from perturbations of the FLRW
    background.  Their study connects general relativity to
    inflationary cosmology and observational astrophysics.
  \item \textbf{Quantum gravity.}
    General relativity and quantum mechanics are individually
    successful but mutually incompatible.  Approaches to their
    unification---string theory, loop quantum gravity, causal
    set theory, and others---remain among the deepest open problems
    in physics.
\end{itemize}

% ──────────────────────────────────────────────────────────────
\subsection{The enduring mystery}%
\label{sec:enduring-mystery}

We began this course (\S\ref{sec:prerelativity}) by noting that
general relativity is, by wide consensus, one of the most beautiful
theories in physics---and also that it is \emph{wrong}, or at least
incomplete.  It describes the large-scale structure of the universe
with breathtaking accuracy, yet it breaks down at singularities and
cannot be reconciled with quantum mechanics.

The deepest questions remain open.  What is the quantum structure of
spacetime?  What happens at the singularity inside a black hole, or
at the Big Bang?  What is the nature of dark energy, which drives
the accelerating expansion of the universe?  Is spacetime
fundamental, or does it emerge from something deeper?

These are questions for the next generation of physicists.  The
tools developed in this course---manifolds, tensors, curvature,
geodesics, Einstein's equations---are the language in which these
questions are posed and in which their answers will be expressed.

\begin{intuition}[A final thought]
  Einstein's general relativity is a theory of unparalleled beauty
  and scope.  From a handful of physical principles---the
  equivalence of gravitational and inertial mass, the requirement
  that physics be independent of coordinates---it constructs a
  framework that describes the expansion of the universe, the
  dynamics of black holes, and the propagation of gravitational
  waves, all as consequences of the curvature of spacetime.  That
  the same equation, $\Ein_{ab} = 8\pi\, T_{ab}$, predicts both
  the $43''$-per-century precession of Mercury and the existence of
  gravitational waves from merging black holes detected a century
  later is, by any measure, extraordinary.

  We have only scratched the surface.  There has been active research
  in general relativity for over a century, and the largest mysteries
  that remain open in physics---the nature of dark energy, the
  quantum structure of spacetime, the origin of the universe---are
  those that arise at the intersection of gravity and quantum theory.
  The journey continues.
\end{intuition}
