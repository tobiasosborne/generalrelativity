%!TEX root = ../GeneralRelativity.tex
% ──────────────────────────────────────────────────────────────
%  Lecture 20 — Interior solutions and geodesics of the
%              Schwarzschild metric
% ──────────────────────────────────────────────────────────────

\subsection{Interior solution: a static spherically symmetric star}%
\label{sec:interior-solution}

The Schwarzschild metric~\eqref{eq:schwarzschild} describes the
gravitational field in the \emph{vacuum region} outside a
spherically symmetric body.  We now find the solution
\emph{inside} the body, where the stress-energy tensor is non-zero.

We model the star as a static, spherically symmetric perfect
fluid---a ``big hot boiling liquid''---with stress-energy tensor
\begin{equation}\label{eq:star-tab}
  T_{ab} = \rho\, u_a\, u_b + p\,(g_{ab} + u_a\, u_b)\,,
\end{equation}
where $\rho(r)$ is the energy density, $p(r)$ the pressure, and
$u^a = -\sqrt{f}\,(dt)^a$ points in the time direction (compatible
with staticity).  The metric retains the general static spherically
symmetric form~\eqref{eq:schwarzschild-ansatz}.

% ──────────────────────────────────────────────────────────────
\subsubsection{Einstein's equations for the interior}%
\label{sec:interior-einstein}

Applying $\Ein_{ab} = 8\pi\, T_{ab}$ in an orthonormal
basis $(e_0, e_1, e_2, e_3)$ adapted to the metric (where
$(e_0)_a = \sqrt{f}\,(dt)_a$, $(e_1)_a = \sqrt{h}\,(dr)_a$,
$(e_2)_a = r\,(d\theta)_a$,
$(e_3)_a = r\sin\theta\,(d\phi)_a$), one obtains three independent
equations (exercise):
\begin{align}
  8\pi\,\rho &= (r\,h^2)^{-1}\, h'
    + r^{-2}(1 - h^{-1})\,,
    \label{eq:interior-1}
    \tag{1}\\[4pt]
  8\pi\, p &= (r\,h)^{-1}\, f'/f
    - r^{-2}(1 - h^{-1})\,,
    \label{eq:interior-2}
    \tag{2}\\[4pt]
  8\pi\, p &= \tfrac{1}{2}(fh)^{-1/2}\,
    \frac{d}{dr}\bigl[(fh)^{-1/2}\, f'\bigr]
    + \tfrac{1}{2}(r\,f\,h)^{-1}\, f'
    - \tfrac{1}{2}(r\,h^2)^{-1}\, h'\,.
    \label{eq:interior-3}
    \tag{3}
\end{align}

% ──────────────────────────────────────────────────────────────
\subsubsection{Solving for $h$: the mass function}%
\label{sec:mass-function}

Equation~\eqref{eq:interior-1} involves $h$ alone and can be
rewritten as
\[
  \frac{1}{r^2}\,\frac{d}{dr}\bigl[r\,(1 - h^{-1})\bigr]
    = 8\pi\,\rho\,.
\]
Integrating with the boundary condition $h(0) = 1$ (smoothness at
the centre):
\begin{equation}\label{eq:h-interior}
  h(r) = \biggl(1 - \frac{2\,m(r)}{r}\biggr)^{-1}\,,
\end{equation}
where the \textbf{mass function}
\begin{equation}\label{eq:mass-function}
  m(r) = 4\pi \int_0^r \rho(r')\, r'^2\, dr'
\end{equation}
represents the total mass-energy enclosed within radius~$r$.
Since the spatial hypersurfaces $\Sigma_t$ must be spacelike, we
require $h > 0$, i.e.\ $r > 2\,m(r)$ throughout the interior.

At the surface $r = R$ of the star (where $\rho = 0$ for
$r > R$), the interior solution must match the exterior
Schwarzschild solution.  This requires $M = m(R)$: the
Schwarzschild mass parameter equals the integrated mass function
evaluated at the stellar radius.

% ──────────────────────────────────────────────────────────────
\subsubsection{Solving for $f$: the gravitational potential}%
\label{sec:interior-f}

Writing $f = e^{2\phi}$ (so that $\phi$ will reduce to the
Newtonian gravitational potential in the weak-field limit),
equation~\eqref{eq:interior-2} with~\eqref{eq:h-interior}
becomes
\begin{equation}\label{eq:phi-equation}
  \frac{d\phi}{dr}
    = \frac{m(r) + 4\pi\, r^3\, p}{r\bigl(r - 2\,m(r)\bigr)}\,.
\end{equation}
In the Newtonian limit ($r^3\, p \ll m(r)$ and $m(r) \ll r$), this
reduces to $d\phi/dr \approx m(r)/r^2$---Poisson's equation for the
gravitational potential, confirming the identification of $\phi$
with the Newtonian potential.

% ──────────────────────────────────────────────────────────────
\subsubsection{The Tolman--Oppenheimer--Volkoff equation}%
\label{sec:tov}

Substituting~\eqref{eq:h-interior} and~\eqref{eq:phi-equation}
into the remaining equation~\eqref{eq:interior-3} yields, after
considerable algebra, the \textbf{Tolman--Oppenheimer--Volkoff (TOV)
equation} of hydrostatic equilibrium:
\begin{equation}\label{eq:tov}
  \eqbox{\frac{dp}{dr}
    = -(\rho + p)\,\frac{m(r) + 4\pi\, r^3\, p}
      {r\bigl(r - 2\,m(r)\bigr)}}
\end{equation}
This is the relativistic generalisation of the Newtonian equation of
hydrostatic equilibrium $dp/dr = -\rho\, m(r)/r^2$.  Every factor
in~\eqref{eq:tov} has a clear physical interpretation:
\begin{itemize}
  \item $\rho + p$ replaces $\rho$: in relativity, pressure
    contributes to the gravitational mass.
  \item $m(r) + 4\pi\, r^3\, p$ replaces $m(r)$: the pressure
    itself gravitates.
  \item $r - 2\,m(r)$ replaces $r$: a relativistic correction from
    the spatial curvature.
\end{itemize}
All three corrections \emph{strengthen} gravity compared to the
Newtonian prediction, making relativistic stars more prone to
gravitational collapse.

\begin{keyresult}[Interior Schwarzschild metric]
  The spacetime geometry inside a static, spherically symmetric,
  perfect-fluid star is
  \begin{equation}\label{eq:interior-metric}
    ds^2 = -e^{2\phi(r)}\, dt^2
      + \biggl(1 - \frac{2\,m(r)}{r}\biggr)^{-1}\, dr^2
      + r^2\, d\Omega^2\,,
  \end{equation}
  where $m(r)$ is given by~\eqref{eq:mass-function} and
  $\phi(r)$ by~\eqref{eq:phi-equation}, subject to the TOV
  equation~\eqref{eq:tov} and an equation of state $p = p(\rho)$.
\end{keyresult}

% ══════════════════════════════════════════════════════════════
%  Part 2: Geodesics of the Schwarzschild solution
% ══════════════════════════════════════════════════════════════

\subsection{Geodesics of the Schwarzschild solution}%
\label{sec:schwarzschild-geodesics}

We now turn to the motion of test bodies and light rays in the
exterior Schwarzschild spacetime ($r > 2M$).  These are the
geodesics of the metric~\eqref{eq:schwarzschild}, and their study
leads to some of the most celebrated predictions of general
relativity.

% ──────────────────────────────────────────────────────────────
\subsubsection{Killing vectors and constants of motion}%
\label{sec:killing-constants}

Before computing geodesics explicitly, we derive a powerful
general result that will simplify the analysis considerably.

\begin{proposition}\label{prop:killing-constant}
  Let $\xi^a$ be a Killing vector field and $\gamma$ a geodesic
  with tangent $u^a$.  Then $\xi_a\, u^a$ is constant along
  $\gamma$.
\end{proposition}

\begin{proof}
  Differentiate along the geodesic:
  \[
    u^b\, \covd_b(\xi_a\, u^a)
      = u^a\, u^b\, \covd_b \xi_a
        + \xi_a\, u^b\, \covd_b u^a\,.
  \]
  The first term vanishes because
  $\covd_{(a}\xi_{b)} = 0$ (Killing's equation) and $u^a u^b$ is
  symmetric.  The second term vanishes by the geodesic equation
  $u^b\, \covd_b u^a = 0$.
\end{proof}

The Schwarzschild metric has two manifest Killing vectors:
\begin{itemize}
  \item $\xi^a = (\partial/\partial t)^a$ (time-translation
    invariance, stationarity),
  \item $\psi^a = (\partial/\partial \phi)^a$ (axial rotational
    invariance).
\end{itemize}
By Proposition~\ref{prop:killing-constant}, each yields a conserved
quantity along any geodesic.  We will exploit these in the next
lecture to reduce the geodesic equation to a single ODE.

% ──────────────────────────────────────────────────────────────
\subsubsection{Restriction to the equatorial plane}%
\label{sec:equatorial-plane}

The Schwarzschild metric is also invariant under the parity
reflection $\theta \mapsto \pi - \theta$.  This symmetry has an
important consequence:

\begin{keyresult}[Geodesics are planar]
  If the initial position and initial tangent vector of a geodesic
  both lie in the equatorial plane $\theta = \pi/2$, then the
  entire geodesic lies in that plane.

  Since any initial conditions can be rotated into the equatorial
  plane by an $SO(3)$ rotation (which is an isometry of the
  Schwarzschild metric), it follows that \emph{every} geodesic in
  the Schwarzschild spacetime is planar.  We may therefore set
  $\theta = \pi/2$ without loss of generality when analysing
  geodesic motion.
\end{keyresult}

This reduces the effective dimension of the problem: instead of
four coordinates $(t, r, \theta, \phi)$, we work with
$(t, r, \phi)$ at $\theta = \pi/2$.  Combined with the two
conserved quantities from the Killing vectors, the geodesic
equation will reduce to a single first-order ODE for~$r$---an
``effective one-dimensional problem'' that we will solve in the
next lecture.
