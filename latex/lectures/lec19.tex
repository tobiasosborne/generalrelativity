%!TEX root = ../GeneralRelativity.tex
% ──────────────────────────────────────────────────────────────
%  Lecture 19 — Deriving the Schwarzschild metric
% ──────────────────────────────────────────────────────────────

\subsection{Solving the vacuum Einstein equations}%
\label{sec:schwarzschild-derivation}

We now carry out the purely mechanical---but highly
rewarding---procedure of substituting the
ansatz~\eqref{eq:schwarzschild-ansatz} into the vacuum Einstein
equations and solving for the unknown functions $f(r)$ and~$h(r)$.
The remarkable thing about the Schwarzschild solution is how easy
the equations are to solve, despite the formidable appearance of the
intermediate expressions.  One should think of the Schwarzschild
solution as the \emph{hydrogen atom of general relativity}: both
pertain to spherically symmetric systems, and both yield an
extraordinary amount of physics from an extraordinarily small amount
of work.

% ──────────────────────────────────────────────────────────────
\subsubsection{Metric components and their inverses}%
\label{sec:schw-metric-components}

The ansatz~\eqref{eq:schwarzschild-ansatz} gives the diagonal metric
components
\[
  g_{tt} = -f(r)\,,\quad
  g_{rr} = h(r)\,,\quad
  g_{\theta\theta} = r^2\,,\quad
  g_{\phi\phi} = r^2 \sin^2\!\theta\,,
\]
with inverse
\[
  g^{tt} = -f^{-1}\,,\quad
  g^{rr} = h^{-1}\,,\quad
  g^{\theta\theta} = r^{-2}\,,\quad
  g^{\phi\phi} = r^{-2}\sin^{-2}\!\theta\,.
\]

% ──────────────────────────────────────────────────────────────
\subsubsection{Christoffel symbols}%
\label{sec:schw-christoffel}

The non-vanishing Christoffel symbols are (exercise---the
main challenge is bookkeeping, not conceptual difficulty):
\begin{align}
  &\chris{r}{rr} = \frac{h'}{2h}\,,\qquad
  \chris{r}{\theta\theta} = -\frac{r}{h}\,,\qquad
  \chris{r}{\phi\phi} = -\frac{r\sin^2\!\theta}{h}\,,
    \label{eq:schw-chris-1}\\[4pt]
  &\chris{r}{tt} = \frac{f'}{2h}\,,\qquad
  \chris{t}{tr} = \frac{f'}{2f}\,,
    \label{eq:schw-chris-2}\\[4pt]
  &\chris{\theta}{r\theta} = \chris{\phi}{r\phi}
    = \frac{1}{r}\,,\qquad
  \chris{\theta}{\phi\phi} = -\sin\theta\cos\theta\,,\qquad
  \chris{\phi}{\theta\phi} = \cot\theta\,,
    \label{eq:schw-chris-3}
\end{align}
where $f' = df/dr$ and $h' = dh/dr$.

\begin{remark}
  No individual Christoffel symbol is difficult to compute---one
  could assign any one of them as a first-year calculus exercise.
  The real challenge is ensuring that the inevitable sign errors and
  missed factors are caught before they propagate.  Computer algebra
  systems are an invaluable complement to hand calculation for this
  purpose.
\end{remark}

% ──────────────────────────────────────────────────────────────
\subsubsection{Ricci tensor}%
\label{sec:schw-ricci}

The Ricci tensor is computed from
$R_{\mu\nu} = \partial_\kappa\, \chris{\kappa}{\mu\nu}
  - \partial_\nu\, \chris{\kappa}{\mu\kappa}
  + \chris{\kappa}{\kappa\lambda}\,\chris{\lambda}{\mu\nu}
  - \chris{\kappa}{\nu\lambda}\,\chris{\lambda}{\mu\kappa}$.
One first verifies that $R_{\mu\nu} = 0$ for $\mu \neq \nu$
(exercise---this considerably reduces the work).  The diagonal
components are:
\begin{align}
  R_{rr} &= -\frac{f''}{2f}
    - \frac{1}{4}\,\frac{f'}{f}\Bigl(\frac{h'}{h}
      + \frac{f'}{f}\Bigr)
    - \frac{1}{r}\,\frac{h'}{h}\,,
    \label{eq:schw-Rrr}\\[6pt]
  R_{\theta\theta} &= -\Bigl(-1
    + \frac{r}{2h}\Bigl(-\frac{h'}{h}
      + \frac{f'}{f}\Bigr) + \frac{1}{h}\Bigr)\,,
    \label{eq:schw-Rthth}\\[6pt]
  R_{\phi\phi} &= \sin^2\!\theta\; R_{\theta\theta}\,,
    \label{eq:schw-Rphph}\\[6pt]
  R_{tt} &= -\Bigl(\frac{f''}{2h}
    - \frac{1}{4}\,\frac{f'}{f}\Bigl(\frac{h'}{h}
      + \frac{f'}{f}\Bigr)
    - \frac{1}{r}\,\frac{f'}{h}\Bigr)\,.
    \label{eq:schw-Rtt}
\end{align}

\begin{exercise}\label{ex:schw-ricci}
  Derive the Ricci tensor
  components~\eqref{eq:schw-Rrr}--\eqref{eq:schw-Rtt} from
  the Christoffel
  symbols~\eqref{eq:schw-chris-1}--\eqref{eq:schw-chris-3}.
  Start by confirming that $R_{\mu\nu} = 0$ for $\mu \neq \nu$.
\end{exercise}

% ──────────────────────────────────────────────────────────────
\subsubsection{Vacuum equations and the solution}%
\label{sec:schw-vacuum-solution}

We model the gravitational field \emph{outside} a spherically
symmetric body (a star, say).  In this exterior region the
stress-energy tensor vanishes, and using the trace-reversed form of
Einstein's equations~\eqref{eq:einstein-alternate} (recalled from
\S\ref{sec:einstein-trace}), the vacuum field equations reduce to
\begin{equation}\label{eq:vacuum-ricci}
  R_{\mu\nu} = 0\,.
\end{equation}
Since $R_{\phi\phi} = \sin^2\!\theta\, R_{\theta\theta}$, it
suffices to solve the three equations
$R_{tt} = R_{rr} = R_{\theta\theta} = 0$.

\medskip\noindent
\textbf{Step 1: Relate $f$ and $h$.}\;
A key observation is that the combination
\begin{equation}\label{eq:fh-relation}
  \frac{R_{tt}}{f} + \frac{R_{rr}}{h}
    = -\frac{1}{r\,h}\Bigl(\frac{h'}{h} + \frac{f'}{f}\Bigr)
    = 0
\end{equation}
yields the remarkably simple equation
\[
  \frac{f'}{f} + \frac{h'}{h} = 0\,,
  \qquad\text{i.e.}\quad
  \frac{d}{dr}\bigl(\ln(f\, h)\bigr) = 0\,,
\]
so that $f\, h = K$ for some constant~$K$.  By rescaling the time
coordinate $t \to \sqrt{K}\, t$, we may set $K = 1$:
\begin{equation}\label{eq:f-h-inverse}
  \eqbox{f = h^{-1}}
\end{equation}
This immediately eliminates one of the two unknown functions.

\medskip\noindent
\textbf{Step 2: Solve for $f$.}\;
Substituting $h = f^{-1}$ into $R_{\theta\theta} = 0$ gives
(exercise):
\begin{equation}\label{eq:schw-ode}
  -f' + \frac{1 - f}{r} = 0\,,
  \qquad\text{i.e.}\quad
  \frac{d}{dr}(r\, f) = 1\,.
\end{equation}
Integrating:
\begin{equation}\label{eq:f-solved}
  f(r) = 1 + \frac{C}{r}\,,
\end{equation}
where $C$ is a constant of integration.

\medskip\noindent
\textbf{Step 3: Interpret the constant.}\;
As $r \to \infty$, the metric~\eqref{eq:schwarzschild-ansatz} with
$f = 1 + C/r$ approaches the Minkowski metric: the $C/r$ terms are
small perturbations.  Comparing with the Newtonian limit derived in
\S\ref{sec:newtonian-limit}---specifically the metric perturbation
$\gamma_{00} = -2\phi$ where $\phi = -M/r$ is the Newtonian
potential---we identify
\begin{equation}\label{eq:C-identification}
  C = -2M\,,
\end{equation}
where $M$ is the mass of the central body (in geometrised units
$G = c = 1$).

\begin{exercise}\label{ex:newtonian-comparison}
  Carry out the comparison explicitly: write the Schwarzschild
  metric to leading order in $M/r$ and compare with the
  linearised metric~\eqref{eq:newtonian-metric} from the Newtonian
  limit.
\end{exercise}

% ──────────────────────────────────────────────────────────────
\subsection{The Schwarzschild metric}%
\label{sec:schwarzschild-metric}

Substituting $C = -2M$ and defining the \textbf{Schwarzschild
radius} $r_s = 2M$:

\begin{keyresult}[Schwarzschild metric]
  \begin{equation}\label{eq:schwarzschild}
    \eqbox{ds^2 = -\Bigl(1 - \frac{2M}{r}\Bigr)\, dt^2
      + \Bigl(1 - \frac{2M}{r}\Bigr)^{-1}\, dr^2
      + r^2\, d\Omega^2}
  \end{equation}
  where $d\Omega^2 = d\theta^2 + \sin^2\!\theta\, d\phi^2$ is the
  standard metric on~$S^2$.  This is the unique vacuum, static,
  spherically symmetric solution of Einstein's field equations,
  obtained by Karl Schwarzschild in 1916.
\end{keyresult}

\begin{intuition}[The hydrogen atom of general relativity]
  We selected, in this course, the two very best exact solutions of
  Einstein's equations: the FLRW cosmologies and the Schwarzschild
  metric.  Both exploit symmetry to reduce the full complexity of
  the field equations to tractable problems, and both yield an
  extraordinary amount of physics.  The Schwarzschild solution
  describes the spacetime outside any spherically symmetric,
  non-rotating mass---stars, planets, non-rotating black holes---and
  is the foundation for the classical tests of general relativity
  (perihelion precession, light deflection, gravitational redshift)
  that we will study in the coming lectures.  It went rather
  painlessly; unfortunately, this is not the general nature of
  things---solving Einstein's equations without high symmetry is an
  extraordinarily intricate task.
\end{intuition}

% ──────────────────────────────────────────────────────────────
\subsection{Singularities of the Schwarzschild metric}%
\label{sec:schw-singularities}

The metric~\eqref{eq:schwarzschild} has apparent singularities at
two values of~$r$:
\begin{enumerate}
  \item $r = 0$: both $g_{tt}$ and $g_{rr}$ diverge.
  \item $r = r_s = 2M$: $g_{tt} = 0$ and
    $g_{rr} \to \infty$.
\end{enumerate}

The singularity at $r = 2M$ is a \textbf{coordinate singularity}:
the spacetime manifold is perfectly smooth there, and the apparent
divergence is an artefact of the Schwarzschild coordinates (much as
the apparent singularity of polar coordinates at the origin of
$\Rn{2}$).  Alternative coordinate systems (Eddington--Finkelstein,
Kruskal--Szekeres) are regular at $r = 2M$.  We will not prove this
here.

The singularity at $r = 0$, by contrast, is a genuine
\textbf{curvature singularity}: the Kretschner scalar
$R_{abcd}\, R^{abcd}$ diverges as $r \to 0$, indicating that
the tidal forces become infinite.

\begin{remark}
  For ``usual'' objects (stars, planets)---as opposed to black
  holes---the Schwarzschild radius $r_s = 2M$ lies deep within the
  interior of the body, where the vacuum solution does not apply.
  For the Sun, $r_s \approx 2.95\,\text{km}$, compared to the
  solar radius of $\approx 7\times 10^5\,\text{km}$.  In such
  cases, the vacuum Schwarzschild solution is joined to an
  \emph{interior solution} (with non-zero stress-energy) at the
  surface of the body, and neither singularity is physically
  relevant.  The singularities become significant only for
  sufficiently compact objects---\textbf{black holes}---which are
  the subject of more advanced courses.
\end{remark}
