%!TEX root = ../GeneralRelativity.tex
% ──────────────────────────────────────────────────────────────
%  Lecture 14 — Gravitational radiation and the beginning of
%              cosmology
% ──────────────────────────────────────────────────────────────

\section{Gravitational radiation}%
\label{sec:gravitational-radiation}

In the previous two lectures we linearised Einstein's field equations
around flat spacetime and showed that in the Lorenz
gauge~\eqref{eq:lorenz-gauge} they reduce to the wave
equation~\eqref{eq:linearised-wave},
$\dalem\,\bar\gamma_{ab} = -16\pi\, T_{ab}$.
We then studied the \emph{static} Newtonian limit, where time
derivatives are negligible, and recovered Newton's law of gravitation.

In general relativity, however, the gravitational field is a
\emph{dynamical} entity: unlike in Newtonian theory, the geometry
itself can change with time.  This opens the possibility of
\textbf{gravitational radiation}---wave-like solutions of the
linearised vacuum Einstein equations that propagate at the speed
of light, analogous to electromagnetic radiation in Maxwell's theory.

\begin{intuition}[Gravitational waves as ripples in spacetime]
  In electrodynamics, accelerating charges produce electromagnetic
  waves.  In general relativity, accelerating masses produce
  gravitational waves: ripples in the spacetime metric that propagate
  outward at speed~$c$.  Just as electromagnetic waves carry energy
  and momentum, gravitational waves carry energy away from their
  source, causing (for instance) binary star systems to spiral
  inward.  The key difference is that gravitational waves are
  perturbations of the \emph{metric itself}, not of a field
  propagating \emph{on} a fixed background.
\end{intuition}

% ──────────────────────────────────────────────────────────────
\subsection{Residual gauge freedom}%
\label{sec:residual-gauge}

Recall from \S\ref{sec:gauge-lorenz} that in the Lorenz gauge we
have:
\begin{align}
  &\text{Gauge condition:} &
    \partial^a \bar\gamma_{ab} &= 0\,,
      \label{eq:lorenz-recall}\\[4pt]
  &\text{Field equation:} &
    \partial^c \partial_c\, \bar\gamma_{ab} &= -16\pi\, T_{ab}\,.
      \label{eq:wave-recall}
\end{align}
The Lorenz condition does \emph{not} completely fix the gauge.  We
are still free to perform gauge transformations
\begin{equation}\label{eq:residual-gauge}
  \gamma_{ab} \;\longrightarrow\;
    \gamma_{ab} + \partial_a \xi_b + \partial_b \xi_a
\end{equation}
provided the vector field $\xi^a$ satisfies
\begin{equation}\label{eq:residual-condition}
  \partial^b \partial_b\, \xi^a = 0\,.
\end{equation}
This is precisely the condition that ensures the new perturbation
still satisfies the Lorenz gauge~\eqref{eq:lorenz-recall}.

\begin{remark}
  The situation is directly analogous to electrodynamics in the
  Lorenz gauge $\partial_\mu A^\mu = 0$: one may still perform gauge
  transformations $A_\mu \to A_\mu + \partial_\mu \Lambda$ provided
  $\dalem\, \Lambda = 0$.  The residual gauge freedom allows us to
  impose additional simplifying conditions on the perturbation.
\end{remark}

% ──────────────────────────────────────────────────────────────
\subsection{The radiation gauge (transverse-traceless gauge)}%
\label{sec:radiation-gauge}

We exploit the residual freedom~\eqref{eq:residual-condition} to
impose the \textbf{radiation gauge} conditions:
\begin{equation}\label{eq:radiation-gauge}
  \gamma = 0\,,\qquad
  \gamma_{0j} = 0\,,
  \qquad j = 1, 2, 3\,,
\end{equation}
where $\gamma = \gamma^a{}_a = \eta^{ab}\,\gamma_{ab}$ is the trace.
The condition $\gamma = 0$ implies $\bar\gamma_{ab} = \gamma_{ab}$
(the trace-reversal is trivial when the trace vanishes), and the
Lorenz condition $\partial^a \bar\gamma_{ab} = 0$ then further
constrains the spatial components.

The radiation gauge is achieved by solving an initial-value problem
for~$\xi^\mu$ on a fixed initial surface~$t = t_0$.  One solves:
\begin{alignat}{2}
  2\Bigl(-\pd{\xi_0}{t} + \nabla\cdot\boldsymbol{\xi}\Bigr)
    &= -\gamma\,,
    &\qquad&\text{(fixes trace)}
    \label{eq:rad-gauge-trace}\\[4pt]
  \pd{\xi_\mu}{t} + \pd{\xi_0}{x^\mu}
    &= -\gamma_{0\mu}\,,
    &\qquad&\mu = 1, 2, 3\,,
    \label{eq:rad-gauge-0j}\\[4pt]
  \nabla^2 \xi_\mu
    + \pd{}{x^\mu}\Bigl(\pd{\xi_0}{t}\Bigr)
    &= -\pd{\gamma_{0\mu}}{t}\,,
    &\qquad&\mu = 1, 2, 3\,.
    \label{eq:rad-gauge-deriv}
\end{alignat}
Given initial data $\xi_\mu$ and $\partial_t \xi_\mu$ on
$t = t_0$, one solves the wave equation $\dalem\,\xi^a = 0$
to propagate the gauge transformation throughout spacetime.  One
then verifies (exercise) that the radiation gauge conditions hold
not just on the initial surface, but at all times.

\begin{keyresult}[Complete gauge fixing in vacuum]
  If $T_{ab} = 0$ \textbf{everywhere} (not just in a region), then
  one can further achieve
  \begin{equation}\label{eq:gamma00-vanish}
    \gamma_{00} = 0\,.
  \end{equation}
  This follows because $\gamma = 0$ implies
  $\bar\gamma_{0b} = \gamma_{0b}$, and the Lorenz
  condition~\eqref{eq:lorenz-recall} then gives
  $\partial_t \gamma_{00} = 0$.  The field
  equation~\eqref{eq:wave-recall} with $T_{00} = 0$ reduces to
  $\nabla^2 \gamma_{00} = 0$, whose unique bounded solution is
  $\gamma_{00} = \text{const}$.  A constant shift of $\gamma_{00}$
  is always permissible, so we set $\gamma_{00} = 0$.
  In this \textbf{transverse-traceless (TT) gauge}, the only
  non-vanishing components of $\gamma_{ab}$ are the purely spatial
  ones $\gamma_{ij}$, which are traceless and transverse
  ($\partial^i \gamma_{ij} = 0$).
\end{keyresult}

% ──────────────────────────────────────────────────────────────
\subsection{Plane wave solutions}%
\label{sec:plane-waves}

We now seek wave-like solutions of the vacuum equation
$\dalem\,\gamma_{ab} = 0$ (recall $\bar\gamma = \gamma$ in the
radiation gauge).  We make the standard plane-wave ansatz:
\begin{equation}\label{eq:plane-wave-ansatz}
  \gamma_{ab} = H_{ab}\, e^{i\, k_\mu x^\mu}\,,
\end{equation}
where $H_{ab}$ is a constant symmetric tensor (the
\textbf{polarization tensor}) and $k_\mu$ is the wave four-vector.

Substituting into the wave equation $\dalem\,\gamma_{ab} = 0$:
\begin{equation}\label{eq:null-wavevector}
  \eqbox{k^\mu k_\mu = 0}
\end{equation}
---the wave vector is \textbf{null}.  Gravitational waves propagate
at the speed of light.

The gauge conditions~\eqref{eq:lorenz-recall}
and~\eqref{eq:radiation-gauge} impose the following constraints on
the polarization tensor:
\begin{alignat}{2}
  &\text{(a)}\quad k^\mu\, H_{\mu\nu} = 0\,,
    &\qquad&\text{(Lorenz gauge)}
    \label{eq:pol-lorenz}\\[3pt]
  &\text{(b)}\quad H_{0\nu} = 0\,,
    &\qquad&\text{(radiation gauge)}
    \label{eq:pol-0nu}\\[3pt]
  &\text{(c)}\quad H^\mu{}_\mu = 0\,.
    &\qquad&\text{(tracelessness)}
    \label{eq:pol-traceless}
\end{alignat}

% ──────────────────────────────────────────────────────────────
\subsubsection{Counting the degrees of freedom}%
\label{sec:polarization-count}

A symmetric $(0,2)$ tensor in four dimensions has $10$ independent
components.  How many survive the gauge and transversality
conditions?

\begin{itemize}
  \item Condition~(a): $4$ equations (one for each value of~$\nu$).
  \item Condition~(b): $4$ equations ($\nu = 0, 1, 2, 3$).
  \item Condition~(c): $1$ equation.
  \item Total: $9$ equations, but (a) and~(b) together imply
    $H_{0\nu}\, k^\nu = 0$, so only $\mathbf{8}$ are independent.
\end{itemize}

\begin{keyresult}[Two polarizations]
  A plane gravitational wave has
  \begin{equation}\label{eq:two-polarizations}
    10 - 8 = 2
  \end{equation}
  linearly independent polarization states.  These are the
  $+$ (``plus'') and $\times$ (``cross'') polarizations.
\end{keyresult}

\begin{exercise}\label{ex:verify-dep}
  Show that conditions~(a) and~(b) are not independent by verifying
  that both imply $H_{0\nu}\, k^\nu = 0$.
\end{exercise}

% ──────────────────────────────────────────────────────────────
\subsection{No gravitational waves in $(2{+}1)$ dimensions}%
\label{sec:no-waves-2+1}

The degree-of-freedom count is dimension-dependent.  In
$(2{+}1)$-dimensional spacetime:
\begin{itemize}
  \item A symmetric $(0,2)$ tensor has $6$ independent components.
  \item The gauge conditions impose $6$ independent constraints.
  \item Result: $6 - 6 = 0$ degrees of freedom---\textbf{no
    gravitational waves} in $(2{+}1)$~dimensions.
\end{itemize}

This can also be understood from the structure of the curvature
tensor.  In $(2{+}1)$ dimensions, the Riemann tensor is completely
determined by the Ricci tensor and scalar:
\begin{equation}\label{eq:riemann-2+1}
  R_{abcd} = g_{ac}\, R_{bd} + g_{bd}\, R_{ac}
    - g_{bc}\, R_{ad} - g_{ad}\, R_{bc}
    - \tfrac{1}{2}\,(g_{ac}\, g_{bd} - g_{ad}\, g_{bc})\, R\,.
\end{equation}
This identity holds because the \textbf{Weyl tensor}~$C_{abcd}$
(the completely trace-free part of the Riemann tensor) vanishes
identically in three dimensions.

\begin{intuition}[Why the Weyl tensor matters]
  The Riemann tensor can be decomposed into a part determined by the
  Ricci tensor (and hence, via Einstein's equations, by the local
  matter content) and a trace-free part---the Weyl tensor---that
  encodes the ``free'' gravitational field.  In four dimensions, the
  Weyl tensor has $10$ independent components, which carry the two
  polarization degrees of freedom of gravitational waves.  In three
  dimensions, the Weyl tensor has $0$ components: there is no room
  for freely propagating gravitational degrees of freedom.

  The vacuum Einstein equations in $(2{+}1)$ dimensions give
  $R_{ab} = 0$ and hence $R = 0$;
  equation~\eqref{eq:riemann-2+1} then forces
  $R_{abcd} = 0$---the spacetime is \emph{flat}.  Gravity in
  $(2{+}1)$ dimensions has no local degrees of freedom whatsoever.
\end{intuition}

% ──────────────────────────────────────────────────────────────
\subsection{Experimental detection}%
\label{sec:gw-detection}

\begin{historical}[LIGO and the first direct detection]
  Gravitational waves were predicted by Einstein in 1916 and
  indirectly confirmed by the Hulse--Taylor binary pulsar
  observations (Nobel Prize 1993), which showed the orbital decay
  rate matching the energy loss to gravitational radiation predicted
  by general relativity.

  The first \emph{direct} detection of gravitational waves was
  announced on 11~February 2016 by the LIGO and Virgo
  collaborations.  The signal, designated GW150914, was detected on
  14~September 2015 and matched the waveform predicted for the
  inspiral and merger of two black holes of approximately $36$ and
  $29$ solar masses.  The detection was made possible by a laser
  interferometer with arm lengths of $4\,\text{km}$, capable of
  measuring displacements of order $10^{-18}\,\text{m}$---roughly
  one-thousandth the diameter of a proton.  The 2017 Nobel Prize in
  Physics was awarded to Rainer Weiss, Barry Barish, and Kip Thorne
  for this discovery.
\end{historical}

% ══════════════════════════════════════════════════════════════
%  Part 2: Beginning of cosmology
% ══════════════════════════════════════════════════════════════

\section{Homogeneous isotropic cosmology}%
\label{sec:cosmology}

We now turn from perturbative solutions to the ambitious programme
of finding \emph{exact} solutions of Einstein's field equations that
describe the large-scale structure of the entire universe.

An arbitrary matter distribution makes Einstein's equations
hopelessly complicated.  Progress requires symmetry assumptions.
The key observational fact is that, on sufficiently large scales
(hundreds of megaparsecs), the distribution of matter in the
universe appears remarkably uniform: there is no preferred location
and no preferred direction.  We formalise these observations as the
assumptions of \textbf{homogeneity} and \textbf{isotropy}.

\begin{intuition}[Why we can model the universe]
  Obviously the universe is \emph{not} homogeneous and isotropic on
  small scales---galaxies, stars, and planets manifestly break the
  symmetry.  But on scales much larger than the typical separation
  between galaxies ($\sim 1\,\text{Mpc}$), the matter distribution
  averages out and looks very nearly uniform.  The most compelling
  evidence for isotropy is the cosmic microwave background radiation,
  which is uniform to one part in $10^5$ across the sky.  We will
  exploit this large-scale uniformity to reduce Einstein's ten
  coupled PDEs to a small number of ordinary differential equations.
\end{intuition}

% ──────────────────────────────────────────────────────────────
\subsection{Homogeneity}%
\label{sec:homogeneity}

\begin{definition}[Homogeneous spacetime]\label{def:homogeneous}
  A spacetime $(\M, g_{ab})$ is \textbf{homogeneous} if there exists
  a one-parameter foliation of~$\M$ by spacelike
  hypersurfaces~$\Sigma_t$ such that, for each~$t$ and any two
  points $p, q \in \Sigma_t$, there exists an isometry
  of~$g_{ab}$ that maps $p$ to~$q$.
\end{definition}

\begin{center}
\begin{tikzpicture}[>=Stealth, thick, scale=0.85]
  % Leaves
  \foreach \y/\lab in {0/$\Sigma_{t_1}$, 1.6/$\Sigma_{t_2}$,
    3.2/$\Sigma_{t_3}$} {
    \draw[spacecadet!60, fill=cgblue!4]
      (0,\y) -- (6,\y) -- (6.8,\y+0.6) -- (0.8,\y+0.6) -- cycle;
    \node[font=\scriptsize, text=spacecadet] at (-0.4,\y+0.3) {\lab};
  }
  % Points p, q on bottom leaf
  \node[circle, fill=banana, inner sep=1.5pt,
    label={below:\scriptsize $p$}] (p) at (2.0,0.25) {};
  \node[circle, fill=banana, inner sep=1.5pt,
    label={below:\scriptsize $q$}] (q) at (4.8,0.35) {};
  \draw[->, cgblue, thick] (p) to[out=20,in=160]
    node[above, font=\scriptsize]{$\phi$} (q);
  % Timeline curves
  \foreach \x in {1.5, 3.0, 4.5} {
    \draw[munsell!50, thick, ->]
      (\x,0.1) -- (\x+0.6,3.7);
  }
  \node[font=\scriptsize, text=munsell] at (5.8,3.5) {$u^a$};
\end{tikzpicture}
\end{center}

Physically, homogeneity says: \emph{we do not occupy a privileged
position in the universe}.  At any given cosmological time~$t$, the
geometry of the spatial slice $\Sigma_t$ looks the same at every
point.

% ──────────────────────────────────────────────────────────────
\subsection{Isotropy}%
\label{sec:isotropy}

Before defining isotropy, we note an important subtlety: if one
observer at a point~$p$ sees an isotropic universe, a
\emph{boosted} observer at the same point generally does not.  The
boost introduces a preferred direction (the velocity of the boosted
observer relative to the matter), and length contraction distorts
the matter distribution along the direction of motion.  Therefore,
isotropy is a condition tied to a \emph{specific family of
observers}.

\begin{definition}[Spatially isotropic spacetime]%
  \label{def:isotropic}
  A spacetime $(\M, g_{ab})$ is \textbf{spatially isotropic at each
  point} if there exists a congruence of timelike curves filling~$\M$
  with tangent vector field~$u^a$ such that, for all $p \in \M$ and
  any two unit spacelike vectors $s_1^a, s_2^a \in T_p\M$ orthogonal
  to~$u^a$ (i.e.\ $u^a s_{1\,a} = u^a s_{2\,a} = 0$), there exists
  an isometry of~$g_{ab}$ that leaves $p$ and $u^a$ fixed but
  maps $s_1^a \mapsto s_2^a$.
\end{definition}

Isotropy says: \emph{there is no preferred spatial direction}.  No
matter which way the preferred observer looks, the universe appears
the same.

% ──────────────────────────────────────────────────────────────
\subsection{Combining homogeneity and isotropy}%
\label{sec:hom-iso-combined}

\begin{keyresult}[Orthogonality of foliation and observers]
  For a spacetime that is both homogeneous and isotropic, the
  hypersurfaces $\Sigma_t$ of the homogeneity foliation must be
  everywhere orthogonal to the tangent vectors~$u^a$ of the
  isotropic observers.

  If this were not the case, the component of~$u^a$ tangent to
  $\Sigma_t$ would define a geometrically preferred spatial direction
  at each point, violating isotropy.
\end{keyresult}

\begin{center}
\begin{tikzpicture}[>=Stealth, thick, scale=0.85]
  % Leaves
  \foreach \y/\lab in {0/$\Sigma_t$, 2.2/} {
    \draw[spacecadet!60, fill=cgblue!4]
      (0,\y) -- (6,\y) -- (6.8,\y+0.6) -- (0.8,\y+0.6) -- cycle;
  }
  \node[font=\scriptsize, text=spacecadet] at (-0.4,0.3) {$\Sigma_t$};
  % Observer at p
  \node[circle, fill=banana, inner sep=1.5pt,
    label={below:\scriptsize $p$}] (p) at (3.0,0.25) {};
  % u^a (vertical)
  \draw[->, munsell, very thick] (3.0,0.25) -- (3.4,2.2)
    node[right, font=\scriptsize]{$u^a$};
  % Spatial vectors
  \draw[->, cgblue] (3.0,0.25) -- (4.5,0.45)
    node[right, font=\scriptsize]{$s_1^a$};
  \draw[->, cgblue] (3.0,0.25) -- (2.8,0.9)
    node[left, font=\scriptsize]{$s_2^a$};
  % Isometry arc
  \draw[->, spacecadet!60, dashed] (4.2,0.6)
    to[out=90,in=0] (3.0,1.1)
    to[out=180,in=90] (2.6,0.75);
  \node[font=\scriptsize, text=spacecadet!60] at (4.6,1.0) {isometry};
  % Points p, q on bottom leaf
  \node[circle, fill=banana, inner sep=1.5pt,
    label={below:\scriptsize $q$}] (q) at (5.2,0.35) {};
  \draw[->, cgblue!50, thick] (p) to[out=10,in=170]
    node[above, font=\scriptsize]{$\phi$} (q);
\end{tikzpicture}
\end{center}

\medskip
The combination of homogeneity and isotropy is extraordinarily
powerful.  It implies that the spatial geometry of each
slice~$\Sigma_t$ is a \emph{maximally symmetric} three-dimensional
Riemannian manifold---and there are only three possibilities:
flat space ($\mathbb{R}^3$), the three-sphere ($S^3$), or hyperbolic
space ($H^3$).  In the next lecture, we will exploit this to derive
the Friedmann--Lema\^itre--Robertson--Walker metric and the
Friedmann equations that govern the expansion of the universe.
