%!TEX root = ../GeneralRelativity.tex
% ──────────────────────────────────────────────────────────────
%  Lecture 13 — Properties of Einstein's equations and maps
%              between manifolds
% ──────────────────────────────────────────────────────────────

\section{Properties of Einstein's equations and maps between
  manifolds}%
\label{sec:einstein-properties}

With Einstein's field equations~\eqref{eq:einstein-field} in hand,
we now investigate their structure and consequences in greater
detail.  We begin by deriving useful algebraic identities---the
trace of the field equations and the specialisation to pressureless
matter---and then discuss three structural properties that
distinguish general relativity from other field theories: the
extreme nonlinearity of the equations, the fact that the
stress-energy tensor cannot be regarded as an independent source,
and the remarkable self-consistency by which the field equations
contain the geodesic hypothesis.  We then provide the detailed
derivation of the linearised quantities used in the previous
lecture, and develop the mathematical formalism of \textbf{maps
between manifolds}---pushforward and pullback---which provides the
rigorous foundation for the diffeomorphism invariance that underlies
the gauge freedom of general relativity.

% ──────────────────────────────────────────────────────────────
\subsection{Algebraic consequences of the field equations}%
\label{sec:algebraic-consequences}

A natural first step when presented with a new equation of motion is
to explore what further equations can be derived from it by
elementary operations.  For a rank-two tensor equation, the most
immediate operation is contraction with the metric.

% ──────────────────────────────────────────────────────────────
\subsubsection{The trace and the alternate form}%
\label{sec:einstein-trace}

Contracting Einstein's equation~\eqref{eq:einstein-field} with
$g^{ab}$:
\[
  g^{ab}\Ein_{ab}
    = g^{ab}\Bigl(\Ric_{ab} - \tfrac{1}{2}\,\Ric\, g_{ab}\Bigr)
    = \Ric - 2\,\Ric = -\Ric
    = 8\pi\, g^{ab} T_{ab}
    = 8\pi\, T\,,
\]
where $T = T^a{}_a = g^{ab}\,T_{ab}$ is the trace of the
stress-energy tensor.  We obtain
\begin{equation}\label{eq:ricci-scalar-trace}
  \eqbox{\Ric = -8\pi\, T}
\end{equation}

\noindent
Substituting~\eqref{eq:ricci-scalar-trace} back
into~\eqref{eq:einstein-field} eliminates the Ricci scalar in
favour of the stress-energy trace:
\begin{equation}\label{eq:einstein-alternate}
  \eqbox{\Ric_{ab} = 8\pi\Bigl(T_{ab}
    - \tfrac{1}{2}\, g_{ab}\, T\Bigr)}
\end{equation}

\begin{remark}
  Equation~\eqref{eq:einstein-alternate} is sometimes more
  convenient than the original form~\eqref{eq:einstein-field},
  since the left-hand side involves only the Ricci tensor.  It is
  particularly useful in settings of high symmetry where the trace
  $T$ takes a simple form.
\end{remark}

\begin{intuition}[Geometry versus matter]
  Equation~\eqref{eq:einstein-alternate} appears to state that the
  geometry (encoded in $\Ric_{ab}$) is determined by the matter
  content (encoded in $T_{ab}$).  While suggestive, this
  interpretation must be treated with great care: the
  metric~$g_{ab}$ enters both sides of the equation, since it is
  needed to define the stress-energy tensor itself, to raise and
  lower indices, and to compute the Ricci tensor.  The information
  flows in both directions simultaneously---geometry constrains
  matter and matter constrains geometry---and one must solve the
  entire system self-consistently.  We return to this point in
  \S\ref{sec:tab-not-source} below.
\end{intuition}

% ──────────────────────────────────────────────────────────────
\subsubsection{The dust limit}%
\label{sec:dust-limit}

Consider the physically important special case where an observer at
rest with respect to the matter measures an energy density~$\rho$
that dominates all stresses.  For a pressureless perfect fluid
(\textbf{dust}) with four-velocity~$v^a$:
\begin{equation}\label{eq:dust-stress-energy}
  T_{ab} = \rho\, v_a v_b\,,
\end{equation}
so that $T = g^{ab}\, T_{ab} = \rho\, v^a v_a = -\rho$ (using
the normalisation $v^a v_a = -1$).
Substituting into~\eqref{eq:einstein-alternate} and contracting
both sides with $v^a v^b$:
\begin{equation}\label{eq:ricci-dust}
  \Ric_{ab}\, v^a v^b
    = 8\pi\bigl(T_{ab}\, v^a v^b
      - \tfrac{1}{2}\,T\, g_{ab}\, v^a v^b\bigr)
    = 8\pi\bigl(\rho - \tfrac{1}{2}\,\rho\bigr)
    = 4\pi\rho\,,
\end{equation}
where we used $T_{ab}\, v^a v^b = \rho\,(v_a v^a)^2 = \rho$,
$g_{ab}\, v^a v^b = -1$, and $T = -\rho$.

\medskip\noindent
The result $\Ric_{ab}\, v^a v^b = 4\pi\rho$ is the
curved-spacetime analogue of the Newtonian relation
$\nabla^2 \phi = 4\pi\rho$: it expresses how the convergence of
geodesics (quantified by the Ricci tensor contracted along the
time direction) is driven by the local energy density.

% ──────────────────────────────────────────────────────────────
\subsection{Three structural properties of Einstein's equations}%
\label{sec:structural-properties}

Before we continue solving Einstein's equations, three features of
the field equations deserve careful discussion.  These features
distinguish general relativity from other classical field theories
and are essential for developing the correct physical intuition.

% ──────────────────────────────────────────────────────────────
\subsubsection{Extreme nonlinearity}%
\label{sec:nonlinearity}

Einstein's field equation~\eqref{eq:einstein-field} is a system
of ten coupled, second-order, \emph{highly nonlinear} partial
differential equations for the ten independent components of the
metric~$g_{\mu\nu}$ (in any coordinate chart).  The nonlinearity
arises because the Christoffel symbols involve first derivatives
of~$g_{\mu\nu}$, the Riemann tensor involves products of
Christoffel symbols (hence products of first derivatives), and the
Ricci tensor involves derivatives of these products (hence second
derivatives multiplied by the metric inverse, which is itself a
nonlinear function of~$g_{\mu\nu}$).

\begin{intuition}[Comparison with other theories]
  The Schr\"odinger equation is linear; its difficulty lies in the
  enormous dimension of the Hilbert space.  Maxwell's equations are
  also linear: given the sources, the electromagnetic field is
  uniquely determined by superposition.  Einstein's equations enjoy
  neither of these simplifications.  They are nonlinear in a way
  that makes even existence and uniqueness of solutions a delicate
  mathematical question.  It must be said that any nonlinear
  equation can in principle be recast as an infinite system of
  linear equations---so one might argue that the difficulties of
  general relativity and quantum mechanics are not entirely
  incomparable---but in practice the nonlinearity of Einstein's
  equations makes them among the most challenging in all of physics.
\end{intuition}

% ──────────────────────────────────────────────────────────────
\subsubsection{The stress-energy tensor is not a simple source}%
\label{sec:tab-not-source}

In electrodynamics, the four-current~$J^a$ acts as a source for
Maxwell's equations: one specifies~$J^a$ independently and then
solves for the electromagnetic field.  It is tempting to draw an
analogy with general relativity and treat $T_{ab}$ as a
``source'' that independently determines the metric.  This analogy
is \emph{misleading}.

The stress-energy tensor of most physical systems---perfect fluids,
electromagnetic fields, scalar fields---is itself defined in terms
of the metric~$g_{ab}$.  For the perfect
fluid~\eqref{eq:perfect-fluid}, for instance, $T_{ab}$ depends on
$g_{ab}$ through the lowered four-velocity $u_a = g_{ab}\, u^b$
and through the metric appearing explicitly in $p\, g_{ab}$.

\begin{keyresult}[Self-consistent system]
  One cannot interpret $T_{ab}$ before solving Einstein's equation.
  The metric and the matter fields must be determined simultaneously
  as a self-consistent solution of the coupled system.  This
  stands in sharp contrast to electrodynamics, where the source
  can be specified independently of the field.
\end{keyresult}

% ──────────────────────────────────────────────────────────────
\subsubsection{Self-consistency: the geodesic hypothesis}%
\label{sec:geodesic-self-consistency}

Recall that the fundamental hypotheses of general relativity
(\S\ref{sec:fundamental-hypotheses}) include the statement that
freely falling test bodies move on geodesics.  Einstein's field
equations imply the conservation
law~\eqref{eq:stress-energy-conservation},
$\covd^a T_{ab} = 0$, as an automatic consequence of the Bianchi
identity.  Remarkably, this conservation law in turn implies the
geodesic hypothesis for pressureless matter.

Consider a dust of non-interacting test particles with
$T_{ab} = \rho\, u_a u_b$.  Expanding the divergence:
\begin{align}
  0 &= \covd^a T_{ab}
     = \covd^a(\rho\, u_a u_b) \notag\\
    &= u_b\, \covd^a(\rho\, u_a)
      + \rho\, u_a\, \covd^a u_b\,.
      \label{eq:dust-conservation}
\end{align}
Contract~\eqref{eq:dust-conservation} with $u^b$.  Since
$u^b u_b = -1$, differentiating gives
$u^b\,\covd_c u_b = 0$ for any direction~$c$, and
hence $u^b\, \rho\, u_a\, \covd^a u_b = 0$.  The contraction
therefore yields
\[
  -\covd^a(\rho\, u_a) = 0\,.
\]
This is the \textbf{continuity equation}: the mass
current~$\rho\, u^a$ is conserved.  Substituting back
into~\eqref{eq:dust-conservation}:
\begin{equation}\label{eq:geodesic-from-efe}
  \rho\, u^a\, \covd_a u_b = 0\,.
\end{equation}
Wherever $\rho \neq 0$, this gives
$u^a \covd_a u^b = 0$: \emph{the dust particles follow geodesics}.

\begin{keyresult}[Einstein's equations contain the geodesic
  hypothesis]
  For a pressureless dust, the conservation law
  $\covd^a T_{ab} = 0$---which is an automatic consequence of
  Einstein's field equations via the Bianchi identity---implies
  that the dust grains move on geodesics of the spacetime metric.
  The geodesic hypothesis is not an independent assumption: it is
  contained within the field equations themselves.
\end{keyresult}

\begin{intuition}[Self-consistency as evidence for correctness]
  This kind of self-consistency is extraordinarily rare in physics.
  We \emph{began} by postulating that freely falling test bodies
  move on geodesics, and used this together with physical reasoning
  to arrive at Einstein's field equations.  The field equations
  then turn around and \emph{imply} the geodesic hypothesis,
  without it having been explicitly built in.  This
  ``bootstrapping'' property is one of the strongest pieces of
  internal evidence that general relativity is on the right track:
  usually when one writes down equations of motion for a complex
  system, checking them for self-consistency with auxiliary
  assumptions reveals contradictions.  That Einstein's equations
  pass this test is, as one might say, breathtakingly impressive.
\end{intuition}

% ──────────────────────────────────────────────────────────────
\subsection{Detailed derivation of the linearised field equations}%
\label{sec:linearised-derivation}

Solving Einstein's field equations is extraordinarily difficult, and
the task will occupy us for the remainder of this course.  We will
approach solutions through three limiting regimes: \emph{weak
gravitational fields} (linearised gravity and gravitational
radiation), \emph{high symmetry} (the Schwarzschild solution), and
\emph{cosmological uniformity} (Friedmann--Lema\^itre--Robertson--Walker
models).

In the previous lecture (\S\ref{sec:linearised}) we stated the
linearised Einstein equation and used it to recover the Newtonian
limit.  We now provide the step-by-step derivation, filling in the
computational details.

% ──────────────────────────────────────────────────────────────
\subsubsection{Setup and index conventions}%
\label{sec:linearised-setup}

Recall the weak-field
ansatz~\eqref{eq:metric-perturbation}:
$g_{ab} = \eta_{ab} + \gamma_{ab}$ with
$|\gamma_{ab}| \ll 1$.

\begin{remark}[The meaning of ``small'']
  In a coordinate-free formulation, the statement
  ``$\gamma_{ab}$ is small'' has no intrinsic meaning: one must
  specify a standard of comparison.  For linearised gravity, we
  define smallness relative to a \emph{global inertial coordinate
  system}---the coordinates in which the background metric takes
  the form $\eta_{\mu\nu} = \diag(-1,+1,+1,+1)$---and require
  that all components $\gamma_{\mu\nu}$ be much less than~$1$ in
  this system.  The very existence of such a global coordinate
  system is a non-trivial assumption: it restricts both the
  topology of~$\M$ and the strength of the gravitational field.
\end{remark}

\noindent
Throughout this section, all indices are raised and lowered with the
flat metric~$\eta_{ab}$, with one important exception:
$g^{ab}$ always denotes the \emph{true inverse} of the full
metric.  To linear order, the inverse metric is
\begin{equation}\label{eq:inverse-metric-linear}
  g^{ab} = \eta^{ab} - \gamma^{ab} + O(\gamma^2)\,.
\end{equation}

\begin{exercise}\label{ex:inverse-metric}
  Verify~\eqref{eq:inverse-metric-linear} by showing that
  $g^{ab}\, g_{bc} = \delta^a{}_c$ to linear order:
  \[
    (\eta^{ab} - \gamma^{ab})(\eta_{bc} + \gamma_{bc})
      = \delta^a{}_c - \gamma^a{}_c + \gamma^a{}_c
        - \gamma^{ab}\gamma_{bc}
      = \delta^a{}_c + O(\gamma^2)\,.
  \]
\end{exercise}

% ──────────────────────────────────────────────────────────────
\subsubsection{Christoffel symbols to linear order}%
\label{sec:christoffel-linear}

The Christoffel symbols of the Levi-Civita connection are
\[
  \chris{c}{ab}
    = \tfrac{1}{2}\, g^{cd}
      \bigl(\partial_a g_{bd} + \partial_b g_{ad}
            - \partial_d g_{ab}\bigr)\,.
\]
Substituting $g_{ab} = \eta_{ab} + \gamma_{ab}$ and
$g^{cd} = \eta^{cd} - \gamma^{cd}$, and noting that
$\partial_a \eta_{bc} = 0$:
\begin{equation}\label{eq:christoffel-linear}
  \chris{c}{ab}
    = \tfrac{1}{2}\, \eta^{cd}
      \bigl(\partial_a \gamma_{bd}
            + \partial_b \gamma_{ad}
            - \partial_d \gamma_{ab}\bigr)
    + O(\gamma^2)\,.
\end{equation}
The $\gamma^{cd}$ contribution to $g^{cd}$ would multiply terms
already of order~$\gamma$, producing $O(\gamma^2)$ corrections
that we discard.

% ──────────────────────────────────────────────────────────────
\subsubsection{The Ricci and Einstein tensors to linear order}%
\label{sec:ricci-einstein-linear}

The Riemann tensor in
coordinates~\eqref{eq:riemann-coord} involves terms linear in
derivatives of~$\chris{}{ab}$ and terms quadratic
in~$\chris{}{ab}$.  Since the Christoffel symbols are already
$O(\gamma)$, the quadratic terms are $O(\gamma^2)$ and may be
dropped.  The Ricci tensor to linear order is therefore
\begin{equation}\label{eq:ricci-linear}
  \Ric_{ab}^{(1)}
    = \partial_c\, \chris{c}{ab}
    - \partial_a\, \chris{c}{cb}\,.
\end{equation}
Substituting~\eqref{eq:christoffel-linear} and simplifying (where
$\gamma = \gamma^c{}_c = \eta^{cd}\,\gamma_{cd}$ denotes the
trace):
\begin{equation}\label{eq:ricci-linear-explicit}
  \Ric_{ab}^{(1)}
    = \partial^c \partial_{(a}\, \gamma_{b)c}
    - \tfrac{1}{2}\, \partial^c \partial_c\, \gamma_{ab}
    - \tfrac{1}{2}\, \partial_a \partial_b\, \gamma\,.
\end{equation}

\begin{exercise}\label{ex:ricci-linear}
  Derive~\eqref{eq:ricci-linear-explicit}
  from~\eqref{eq:christoffel-linear}
  and~\eqref{eq:ricci-linear}.
\end{exercise}

\noindent
To linear order, $\Ein_{ab}^{(1)} = \Ric_{ab}^{(1)} -
\frac{1}{2}\, \eta_{ab}\, \Ric^{(1)}$ (the metric multiplying the
Ricci scalar may be replaced by~$\eta_{ab}$, since
$\gamma_{ab}\,\Ric^{(1)}$ is $O(\gamma^2)$).  Computing
$\Ric^{(1)} = \eta^{ab}\,\Ric^{(1)}_{ab}$ and combining terms:
\begin{equation}\label{eq:einstein-tensor-linear}
  \Ein_{ab}^{(1)}
    = \partial^c \partial_{(a}\, \gamma_{b)c}
    - \tfrac{1}{2}\, \partial^c \partial_c\, \gamma_{ab}
    - \tfrac{1}{2}\, \partial_a \partial_b\, \gamma
    - \tfrac{1}{2}\, \eta_{ab}
      \bigl(\partial^c \partial^d\, \gamma_{cd}
            - \partial^c \partial_c\, \gamma\bigr)\,.
\end{equation}
This expression simplifies considerably when written in terms of
the trace-reversed perturbation~\eqref{eq:trace-reversed},
$\bar\gamma_{ab} = \gamma_{ab} - \frac{1}{2}\,\eta_{ab}\,\gamma$:
\begin{equation}\label{eq:linearised-einstein-detail}
  \Ein_{ab}^{(1)}
    = -\tfrac{1}{2}\, \partial^c \partial_c\, \bar\gamma_{ab}
    + \partial^c \partial_{(a}\, \bar\gamma_{b)c}
    - \tfrac{1}{2}\, \eta_{ab}\,
      \partial^c \partial^d\, \bar\gamma_{cd}\,.
\end{equation}
Setting $\Ein_{ab}^{(1)} = 8\pi\, T_{ab}$ recovers the linearised
Einstein equation~\eqref{eq:linearised-einstein} stated in the
previous lecture.

\begin{exercise}\label{ex:trace-reversed-simplification}
  Verify that substituting $\gamma_{ab} = \bar\gamma_{ab} +
  \frac{1}{2}\, \eta_{ab}\, \bar\gamma$ (where
  $\bar\gamma = \eta^{ab}\, \bar\gamma_{ab} = -\gamma$)
  into~\eqref{eq:einstein-tensor-linear}
  yields~\eqref{eq:linearised-einstein-detail}.
\end{exercise}

% ──────────────────────────────────────────────────────────────
\subsection{Maps between manifolds}%
\label{sec:maps-manifolds}

The gauge freedom of linearised
gravity~(\S\ref{sec:gauge-lorenz}) arose from infinitesimal
diffeomorphisms.  To place this on firm mathematical footing, we
now develop the theory of maps between manifolds and their action
on geometric objects.  This formalism also underpins the Lie
derivative introduced in \S\ref{sec:lie-derivative}, where we
specialised to the case of a diffeomorphism from~$\M$ to itself
generated by a vector field.

% ──────────────────────────────────────────────────────────────
\subsubsection{Pullback of functions}%
\label{sec:pullback-functions}

Let $\M$ and $\mathcal{N}$ be smooth manifolds and
$\phi\colon \M \to \mathcal{N}$ a smooth map.  Given any smooth
function $f\colon \mathcal{N} \to \R$, the composition
\begin{equation}\label{eq:pullback-function}
  \phi^* f = f \circ \phi \colon \M \to \R
\end{equation}
defines a smooth function on~$\M$, called the \textbf{pullback}
of~$f$ by~$\phi$.

\begin{center}
\begin{tikzpicture}[>=Stealth, thick, font=\small]
  % Manifold blobs
  \draw[rounded corners=12pt, spacecadet, thick]
    (0,0) ellipse (1.2 and 0.8);
  \node[font=\sf, text=cgblue] at (0,1.1) {$\M$};
  \node[circle, fill=banana, inner sep=1.5pt,
    label=below:{\scriptsize $p$}]
    (p) at (-0.2,0) {};

  \draw[rounded corners=12pt, spacecadet, thick]
    (4,0) ellipse (1.2 and 0.8);
  \node[font=\sf, text=cgblue] at (4,1.1) {$\mathcal{N}$};
  \node[circle, fill=banana, inner sep=1.5pt,
    label=below:{\scriptsize $\phi(p)$}]
    (phip) at (3.8,0) {};

  \node (R) at (7.5,0) {$\R$};

  % Arrows
  \draw[->, cgblue] (1.4,0.3) -- node[above]{\small $\phi$}
    (2.6,0.3);
  \draw[->, cgblue] (5.4,0.3) -- node[above]{\small $f$}
    (7.0,0.3);
  \draw[->, munsell, dashed] (1.0,-0.7)
    to[out=-20,in=200] node[below]{\small $f\circ\phi$} (7.0,-0.7);
\end{tikzpicture}
\end{center}

\noindent
The pullback operates in the direction \emph{opposite} to~$\phi$:
while $\phi$ maps $\M \to \mathcal{N}$, the pullback~$\phi^*$
maps functions on~$\mathcal{N}$ to functions on~$\M$.

% ──────────────────────────────────────────────────────────────
\subsubsection{Pushforward of tangent vectors}%
\label{sec:pushforward-vectors}

For any tangent vector $v \in T_p\M$, we define the
\textbf{pushforward} $\phi_* v \in T_{\phi(p)}\mathcal{N}$ by its
action on smooth functions $f$ on~$\mathcal{N}$:
\begin{equation}\label{eq:pushforward-def}
  (\phi_* v)(f) = v(f \circ \phi)\,.
\end{equation}

\begin{exercise}\label{ex:pushforward-tangent}
  Verify that $\phi_* v$ so defined is a tangent vector at
  $\phi(p)$: check that it is $\R$-linear and satisfies the
  Leibniz rule with respect to pointwise multiplication of
  functions on~$\mathcal{N}$.
\end{exercise}

\noindent
The pushforward defines a linear map
\begin{equation}\label{eq:pushforward-map}
  \phi_*\colon T_p\M \;\longrightarrow\;
    T_{\phi(p)}\mathcal{N}\,.
\end{equation}

\medskip\noindent
\textbf{Relation to the Jacobian.}\;
Let $\{x^\mu\}$ be coordinates on~$\M$ and $\{y^\nu\}$ coordinates
on~$\mathcal{N}$.  The matrix of $\phi_*$ in the corresponding
coordinate bases is
\begin{equation}\label{eq:pushforward-jacobian}
  [\phi_*]^\nu{}_\mu = \pd{y^\nu}{x^\mu}\,,
\end{equation}
which is the \textbf{Jacobian matrix} of the map~$\phi$.

\begin{exercise}\label{ex:pushforward-jacobian}
  Derive~\eqref{eq:pushforward-jacobian}
  from~\eqref{eq:pushforward-def} by evaluating
  $\bigl(\phi_*\, \partial/\partial x^\mu\bigr)(y^\nu)$.
\end{exercise}

% ──────────────────────────────────────────────────────────────
\subsubsection{Pullback of covectors}%
\label{sec:pullback-covectors}

Given a covector $\mu \in T^*_{\phi(p)}\mathcal{N}$, the
\textbf{pullback covector} $\phi^* \mu \in T^*_p\M$ is defined by
demanding consistency with the natural pairing:
\begin{equation}\label{eq:pullback-covector}
  (\phi^* \mu)(v) = \mu(\phi_* v)
  \qquad \text{for all } v \in T_p\M\,.
\end{equation}
This defines a linear map
$\phi^*\colon T^*_{\phi(p)}\mathcal{N} \to T^*_p\M$,
going in the direction \emph{opposite} to~$\phi_*$.

% ──────────────────────────────────────────────────────────────
\subsubsection{Extension to pure tensors}%
\label{sec:pullback-push-tensors}

The pullback extends naturally to covariant tensors of
type~$(0,l)$.  For $T_{a_1 \cdots a_l}$ a tensor on~$\mathcal{N}$:
\begin{equation}\label{eq:pullback-tensor}
  (\phi^* T)_{a_1 \cdots a_l}\, v_1^{a_1} \cdots v_l^{a_l}
    = T_{a_1 \cdots a_l}\,
      (\phi_* v_1)^{a_1} \cdots (\phi_* v_l)^{a_l}
\end{equation}
for all $v_1, \ldots, v_l \in T_p\M$.
Similarly, the pushforward extends to contravariant tensors of
type~$(k,0)$.  For $T^{a_1 \cdots a_k}$ a tensor on~$\M$:
\begin{equation}\label{eq:pushforward-tensor}
  (\phi_* T)^{a_1 \cdots a_k}\,
    (\mu_1)_{a_1} \cdots (\mu_k)_{a_k}
    = T^{a_1 \cdots a_k}\,
      (\phi^* \mu_1)_{a_1} \cdots (\phi^* \mu_k)_{a_k}
\end{equation}
for all
$\mu_1, \ldots, \mu_k \in T^*_{\phi(p)}\mathcal{N}$.

\begin{remark}
  Note the directions of these maps: $\phi_*$ pushes tangent
  vectors and contravariant tensors
  \emph{forward} ($\M \to \mathcal{N}$), while $\phi^*$ pulls
  covectors and covariant tensors \emph{back}
  ($\mathcal{N} \to \M$).  For a general smooth map, there is
  no natural action on \emph{mixed} tensors of type~$(k,l)$
  with $k, l \geq 1$, because the pushforward and pullback point
  in opposite directions.
\end{remark}

% ──────────────────────────────────────────────────────────────
\subsubsection{Mixed tensors and diffeomorphisms}%
\label{sec:mixed-tensors-diffeo}

If $\phi\colon \M \to \M$ is a \textbf{diffeomorphism} (a smooth
bijection with smooth inverse), we can overcome the
arrow-direction obstruction.  Since $\phi^{-1}$ exists and is
smooth, we have an additional pushforward
\[
  (\phi^{-1})_*\colon T_{\phi(p)}\M
    \;\longrightarrow\; T_p\M
\]
that takes tangent vectors at~$\phi(p)$ \emph{back} to~$p$.  This
allows us to define the pullback of a mixed tensor
$T^{a_1 \cdots a_k}{}_{b_1 \cdots b_l}$ by:
\begin{equation}\label{eq:pullback-mixed}
  \begin{split}
  &(\phi^* T)^{a_1 \cdots a_k}{}_{b_1 \cdots b_l}\,
    (\mu_1)_{a_1} \cdots (\mu_k)_{a_k}\,
    v_1^{b_1} \cdots v_l^{b_l}\\
  &\qquad= T^{a_1 \cdots a_k}{}_{b_1 \cdots b_l}\,
    \bigl((\phi^{-1})^* \mu_1\bigr)_{a_1} \cdots
    \bigl((\phi^{-1})^* \mu_k\bigr)_{a_k}\,
    (\phi_* v_1)^{b_1} \cdots (\phi_* v_l)^{b_l}\,,
  \end{split}
\end{equation}
for all $v_1, \ldots, v_l \in T_p\M$ and
$\mu_1, \ldots, \mu_k \in T^*_p\M$.

\begin{exercise}\label{ex:pullback-consistency}
  Show that $\phi_* = (\phi^{-1})^*$ when both are defined on
  tangent vectors.
  \emph{Hint}: for $v \in T_p\M$ and any smooth function $f$,
  evaluate $\bigl((\phi^{-1})^* v\bigr)(f)$ using the pullback
  definition.
\end{exercise}

% ──────────────────────────────────────────────────────────────
\subsubsection{Isometries}%
\label{sec:isometries-def}

Among all diffeomorphisms, those that preserve the metric occupy
a special role.

\begin{definition}[Isometry]\label{def:isometry}
  A diffeomorphism $\phi\colon \M \to \M$ is an \textbf{isometry}
  if the pullback of the metric equals the metric:
  \begin{equation}\label{eq:isometry-def}
    \phi^* g_{ab} = g_{ab}\,.
  \end{equation}
\end{definition}

\noindent
An isometry preserves all distances and angles: the spacetimes
$(\M, g_{ab})$ and $(\M, \phi^* g_{ab})$ are geometrically
identical.  In Minkowski space, the isometries form the
Poincar\'e group (translations and Lorentz transformations).

\begin{keyresult}[Symmetries versus gauge freedom]
  The isometries of a spacetime are its \emph{symmetries}: they
  generate conserved quantities via the Killing vector fields
  of~\S\ref{sec:killing}.  However, the \emph{gauge freedom} of
  general relativity is much larger: \emph{all} diffeomorphisms
  (not just isometries) map solutions of Einstein's equations to
  physically equivalent solutions.  The spacetimes
  $(\M, g_{ab})$ and $(\M, \phi^* g_{ab})$ describe the same
  physics for \emph{any} diffeomorphism~$\phi$, regardless of
  whether~$\phi$ is an isometry.
\end{keyresult}

\begin{intuition}[Isometries versus diffeomorphisms]
  Think of an isometry as a rigid motion of the manifold---it
  moves points around but preserves the ``shape'' of spacetime.
  A general diffeomorphism may distort the metric, but because
  general relativity is formulated in terms of coordinate-free
  geometric quantities, the distorted and undistorted spacetimes
  make identical physical predictions.  The distinction is
  analogous to that between a rotation (which preserves a
  vector's length) and a general coordinate transformation
  (which changes the components but not the vector itself).
\end{intuition}

\medskip
With the formalism of maps between manifolds in hand, we have
placed the gauge freedom of general relativity---and in particular
the gauge transformations of linearised
gravity~\eqref{eq:gauge-transformation}---on a rigorous
mathematical footing.  In the next lecture, we will apply the
linearised theory developed in Lectures~12 and~13 to study
\textbf{gravitational radiation}: wave-like solutions of the vacuum
linearised Einstein equations that propagate at the speed of light.
