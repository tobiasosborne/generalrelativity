%!TEX root = ../GeneralRelativity.tex
% ──────────────────────────────────────────────────────────────
%  Lecture 12 — Lie derivatives and the Newtonian limit
% ──────────────────────────────────────────────────────────────

\section{Lie derivatives and the Newtonian limit}%
\label{sec:lie-newtonian}

Having developed the full apparatus of differential
geometry---manifolds, tangent spaces, tensor fields, covariant
derivatives, and curvature---we are finally ready to formulate
Einstein's general theory of relativity.  In this lecture we state
the fundamental hypotheses of GR and discuss the principles that
guide the construction of equations of motion in curved spacetime.
We then introduce the \textbf{Lie derivative}, which makes precise
the sense in which diffeomorphisms are the gauge symmetry of
general relativity, and leads naturally to \textbf{Killing vector
fields} and the connection between spacetime symmetries and
conservation laws.  Finally, we state \textbf{Einstein's field
equations} relating spacetime curvature to the distribution of
matter, linearise them around flat spacetime, and show that in the
appropriate limit they reproduce Newtonian gravity: both Poisson's
equation for the gravitational potential and Newton's equation of
motion for test particles.

% ──────────────────────────────────────────────────────────────
\subsection{The fundamental hypotheses of general relativity}%
\label{sec:fundamental-hypotheses}

The equivalence principle, developed in Lecture~2, tells us that no
experiment can locally distinguish a gravitational field from
uniform acceleration.  A far-reaching consequence is that it is
impossible, in general, to set up a family of inertial observers to
measure the gravitational ``force''---any observer falls along with
the test body, and there is no natural background motion to compare
with.  This motivates us to abandon the concept of gravitational
force altogether and instead encode gravity in the geometry of
spacetime.

\begin{keyresult}[Fundamental hypothesis of general relativity]
  \begin{enumerate}
    \item Spacetime is a four-dimensional manifold~$\M$ equipped
      with a Lorentz metric~$g_{ab}$ of signature~$(-,+,+,+)$.
    \item The manifold~$\M$ need not be topologically~$\Rn{4}$,
      and the metric~$g_{ab}$ need not be flat.
    \item The world lines of freely falling bodies are geodesics
      of~$g_{ab}$.
  \end{enumerate}
\end{keyresult}

\begin{remark}
  While the gravitational force cannot be measured directly in a
  general spacetime, the \emph{relative acceleration} of nearby
  geodesics---quantified by the geodesic deviation
  equation~\eqref{eq:geodesic-deviation}---is always well defined.
  We can therefore speak meaningfully of \emph{tidal forces} in
  general relativity.  This observation will prove critical when we
  derive Einstein's field equations below.
\end{remark}

\begin{intuition}[When can we measure ``gravitational force''?]
  In situations with time-translation symmetry---for instance, the
  gravitational field of a single isolated body like the
  Earth---one can set up a preferred family of static observers and
  define a gravitational force relative to them.  But the moment
  two or more bodies are in relative motion, the time-translation
  symmetry is broken and no preferred background exists.  General
  relativity is designed to handle this fully general situation.
\end{intuition}

% ──────────────────────────────────────────────────────────────
\subsection{Equations of motion in curved spacetime}%
\label{sec:eom-curved}

Given the fundamental hypothesis, how do we determine the laws of
physics in curved spacetime?  Two guiding principles, while not
mathematically rigorous, are powerful enough to essentially
determine the correct equations of motion in most situations of
physical interest.

% ──────────────────────────────────────────────────────────────
\subsubsection{General covariance and minimal substitution}%
\label{sec:general-covariance}

\begin{definition}[Principle of general covariance]%
\label{def:general-covariance}
  Only the metric~$g_{ab}$ and quantities derivable from it may
  appear as spacetime quantities in the laws of physics.
\end{definition}

\noindent
In particular, no reference to a preferred coordinate system or
background structure is permitted.

\begin{definition}[Correspondence principle]%
\label{def:correspondence}
  The equations of physics in general relativity must reduce to
  their special-relativistic form when $g_{ab} = \eta_{ab}$.
\end{definition}

\noindent
Together, these two principles suggest the following
\textbf{minimal substitution rules} for passing from special to
general relativity:
\begin{equation}\label{eq:minimal-substitution}
  \eta_{ab} \;\longrightarrow\; g_{ab}\,,
  \qquad
  \partial_a \;\longrightarrow\; \covd_a\,,
\end{equation}
where $\covd_a$ is the Levi-Civita connection of~$g_{ab}$.

\begin{remark}
  The minimal substitution rules are a starting point, not a
  theorem.  Additional curvature-coupling terms (e.g.\ a term
  $\alpha\, \Ric\, \phi$ in a scalar field equation) may be
  permitted by both principles, since such terms vanish in the flat
  limit.  One must supply independent arguments---experiment or
  internal consistency---to fix such ambiguities.
\end{remark}

% ──────────────────────────────────────────────────────────────
\subsubsection{Free particles, energy, and the stress-energy
  tensor}\label{sec:free-particles-gr}

Physical quantities continue to be represented by tensor fields,
just as in special relativity.  The key updates concern the
equations of motion.

\medskip\noindent
\textbf{Free particle motion.}\;
A free particle follows a time-like geodesic:
\begin{equation}\label{eq:geodesic-eom-gr}
  u^a \covd_a u^b = 0\,,
\end{equation}
where $u^a$ is the unit four-velocity tangent to the world line,
normalised so that $g_{ab}\, u^a u^b = -1$.  When a force acts,
the equation generalises to $f^a = m\, u^b \covd_b u^a$, where $m$
is the rest mass.  The four-momentum is $p^a = m\, u^a$.

\medskip\noindent
\textbf{Energy.}\;
An observer with four-velocity~$v^a$, present at an event on the
particle's world line, measures the particle's energy to be
\begin{equation}\label{eq:energy-local}
  E = -p_a\, v^a = -g_{ab}\, p^a v^b\,.
\end{equation}
This definition requires the observer and the particle to be at
the \emph{same} spacetime event: on a curved manifold, parallel
transport is path-dependent, so there is no canonical way to
compare vectors at distant points.  A given observer cannot, in
general, define the energy of a distant particle.

\medskip\noindent
\textbf{Continuous matter.}\;
A continuous distribution of matter is described by the
stress-energy tensor~$T_{ab}$.  For a \textbf{perfect fluid} with
energy density~$\rho$, pressure~$p$, and four-velocity
field~$u^a$:
\begin{equation}\label{eq:perfect-fluid}
  T_{ab} = (\rho + p)\, u_a u_b + p\, g_{ab}\,.
\end{equation}
Applying the minimal substitution rule to the
special-relativistic conservation law
$\partial^a T_{ab} = 0$ yields
\begin{equation}\label{eq:stress-energy-conservation}
  \eqbox{\covd^a T_{ab} = 0}
\end{equation}

\begin{intuition}[Why $\covd^a T_{ab} = 0$ is not global
  conservation]
  In special relativity, $\partial^a T_{ab} = 0$ implies global
  conservation of energy and momentum via Gauss's theorem: one
  sets up a family of inertial observers with parallel
  four-velocities, defines a current
  $J^a = -T^{ab}\, v_b$, and integrates its divergence over a
  spacetime region.  In general relativity, the analogous
  construction requires $v^a$ to satisfy Killing's equation
  $\covd_{(a} v_{b)} = 0$---but such vector fields need not
  exist on a general curved spacetime.  We will return to this
  point in \S\ref{sec:conservation-gr}.
\end{intuition}

% ──────────────────────────────────────────────────────────────
\subsection{The Lie derivative}\label{sec:lie-derivative}

% ──────────────────────────────────────────────────────────────
\subsubsection{Diffeomorphism invariance in general relativity}%
\label{sec:diffeo-invariance}

For a diffeomorphism $\phi\colon \M \to \M$, the spacetimes
$(\M, g_{ab})$ and $(\M, \phi^* g_{ab})$ describe the same
physics: all operationally defined quantities are built from
tensors in a coordinate-free manner, so the Jacobian factors
introduced by~$\phi$ cancel in every contraction and tensor
transformation law.

\begin{keyresult}[Gauge freedom of general relativity]
  Diffeomorphisms comprise the gauge freedom of general
  relativity.  Two metrics related by a diffeomorphism represent
  the same physical spacetime.
\end{keyresult}

\noindent
To study \emph{infinitesimal} diffeomorphisms, we need a notion
of derivative that measures how a tensor field changes along a
flow.  This is the \textbf{Lie derivative}.

% ──────────────────────────────────────────────────────────────
\subsubsection{Definition and elementary properties}%
\label{sec:lie-def}

Let $v^a$ be a smooth vector field on~$\M$ and $\phi_t$ the
one-parameter group of diffeomorphisms it generates (its
\emph{flow}, introduced in Lecture~5).  The
pullback~$\phi_t^*$ acts on tensor fields by dragging them back
along the flow.

\begin{definition}[Lie derivative]\label{def:lie-derivative}
  The \textbf{Lie derivative} of a tensor field
  $T^{a_1 \cdots a_k}{}_{b_1 \cdots b_l}$ along~$v^a$ is
  \begin{equation}\label{eq:lie-def}
    \Lie_v T^{a_1 \cdots a_k}{}_{b_1 \cdots b_l}
      = \lim_{t \to 0} \frac{1}{t}
        \Bigl(
          \phi_t^*\, T^{a_1 \cdots a_k}{}_{b_1 \cdots b_l}
            \big|_{\phi_t(p)}
          - T^{a_1 \cdots a_k}{}_{b_1 \cdots b_l}\big|_p
        \Bigr)\,.
  \end{equation}
\end{definition}

\noindent
The Lie derivative has three fundamental properties:
\begin{enumerate}
  \item \textbf{Tensorial}: $\Lie_v T$ is a tensor field of the
    same type as~$T$.
  \item \textbf{Leibniz rule}: for any tensor fields~$S$ and~$T$,
    $\Lie_v(S \tp T) = (\Lie_v S) \tp T + S \tp (\Lie_v T)$.
  \item \textbf{Linearity}: $\Lie_v(\alpha S + \beta T)
    = \alpha\, \Lie_v S + \beta\, \Lie_v T$ for
    $\alpha, \beta \in \R$.
\end{enumerate}

\medskip\noindent
\textbf{On scalar fields.}\;
For $f \in C^\infty(\M)$:
\begin{equation}\label{eq:lie-scalar}
  \Lie_v f = v(f) = v^a \partial_a f\,.
\end{equation}

\medskip\noindent
\textbf{On vector fields.}\;
For a vector field~$w^a$, the Lie derivative is the Lie bracket:
\begin{equation}\label{eq:lie-vector}
  \Lie_v w^a = [v, w]^a
    = v^b \covd_b w^a - w^b \covd_b v^a\,.
\end{equation}

\begin{exercise}\label{ex:lie-vector-proof}
  Prove~\eqref{eq:lie-vector} from the
  definition~\eqref{eq:lie-def}.
\end{exercise}

\medskip\noindent
\textbf{On covector fields.}\;
For a covector field~$\mu_a$, apply the Leibniz rule to the
scalar $\mu_a w^a$ for an arbitrary vector field~$w^a$:
\[
  \Lie_v(\mu_a w^a)
    = w^a \Lie_v \mu_a + \mu_a\, [v, w]^a\,.
\]
Since $\Lie_v(\mu_a w^a) = v^b \covd_b(\mu_a w^a)$ by
\eqref{eq:lie-scalar}, expanding the right-hand side and comparing
terms yields (since $w^a$ is arbitrary):
\begin{equation}\label{eq:lie-covector}
  \Lie_v \mu_a = v^b \covd_b \mu_a + \mu_b\, \covd_a v^b\,.
\end{equation}

% ──────────────────────────────────────────────────────────────
\subsubsection{The Lie derivative of a general tensor field}%
\label{sec:lie-general}

Extending the pattern of~\eqref{eq:lie-vector}
and~\eqref{eq:lie-covector} to a tensor of type~$(k,l)$:

\begin{theorem}[Lie derivative of a general tensor]%
\label{thm:lie-general}
  For any torsion-free derivative operator~$\covd_a$:
  \begin{equation}\label{eq:lie-general}
    \eqbox{
    \Lie_v T^{a_1 \cdots a_k}{}_{b_1 \cdots b_l}
      = v^c \covd_c T^{a_1 \cdots a_k}{}_{b_1 \cdots b_l}
        - \sum_{j=1}^{k}
          T^{a_1 \cdots c \cdots a_k}{}_{b_1 \cdots b_l}\,
          \covd_c v^{a_j}
        + \sum_{j=1}^{l}
          T^{a_1 \cdots a_k}{}_{b_1 \cdots c \cdots b_l}\,
          \covd_{b_j} v^c
    }
  \end{equation}
  In the first sum, the index~$c$ replaces~$a_j$ in the $j$-th
  contravariant slot; in the second sum, $c$ replaces~$b_j$ in the
  $j$-th covariant slot.
\end{theorem}

\begin{remark}
  Although individual terms in~\eqref{eq:lie-general} depend on the
  choice of~$\covd_a$, the Christoffel-symbol contributions cancel
  in the sum.  The Lie derivative is defined purely in terms of the
  flow of~$v^a$ and requires no connection.
\end{remark}

\begin{exercise}\label{ex:lie-general-verify}
  Verify~\eqref{eq:lie-general} by applying the Leibniz rule
  repeatedly, using the results for
  scalars~\eqref{eq:lie-scalar},
  vectors~\eqref{eq:lie-vector}, and
  covectors~\eqref{eq:lie-covector}.
\end{exercise}

% ──────────────────────────────────────────────────────────────
\subsection{Killing vector fields and conservation laws}%
\label{sec:killing}

% ──────────────────────────────────────────────────────────────
\subsubsection{The Lie derivative of the metric}%
\label{sec:lie-metric}

Since $\covd_a g_{bc} = 0$ for the Levi-Civita connection,
applying~\eqref{eq:lie-general} to the metric gives
\begin{equation}\label{eq:lie-metric}
  \eqbox{\Lie_v g_{ab} = \covd_a v_b + \covd_b v_a}
\end{equation}
where $v_a = g_{ab}\, v^b$.

\begin{intuition}[Lie derivative as metric strain]
  The quantity $\Lie_v g_{ab}$ measures how the metric changes as
  one flows along~$v^a$.  If $\Lie_v g_{ab} = 0$, then distances
  and angles are preserved by the flow: it is an isometry.
  Formula~\eqref{eq:lie-metric} shows that this condition depends
  only on the symmetrised covariant derivative of~$v^a$---the
  curved-space analogue of the strain tensor in continuum
  mechanics.
\end{intuition}

% ──────────────────────────────────────────────────────────────
\subsubsection{Killing's equation and isometries}%
\label{sec:killing-equation}

\begin{definition}[Killing vector field]\label{def:killing}
  A vector field~$v^a$ is a \textbf{Killing vector field} if it
  generates an isometry: $\Lie_v g_{ab} = 0$.
  By~\eqref{eq:lie-metric}, this is equivalent to
  \textbf{Killing's equation}:
  \begin{equation}\label{eq:killing-equation}
    \eqbox{\covd_{(a} v_{b)} = 0}
  \end{equation}
\end{definition}

\noindent
Each solution of Killing's equation corresponds to a continuous
symmetry of the spacetime geometry.  In Minkowski space, there are
ten independent Killing vector fields: four translations
(generating conservation of energy-momentum) and six Lorentz
transformations (generating conservation of angular momentum).

% ──────────────────────────────────────────────────────────────
\subsubsection{Conservation laws in general relativity}%
\label{sec:conservation-gr}

In special relativity, the condition $\partial^a T_{ab} = 0$
leads to global conservation laws via Gauss's theorem.  The
procedure requires a family of inertial observers with parallel
four-velocities ($\partial_a v^b = 0$), which allows one to
define a conserved current
$J^a = -T^{ab}\, v_b$ satisfying $\partial_a J^a = 0$.

In general relativity, the analogous condition
$\covd^a T_{ab} = 0$ does \emph{not} by itself imply a global
conservation law.  To construct a divergence-free current, we need
a vector field~$v^a$ satisfying Killing's
equation~\eqref{eq:killing-equation}.

\begin{keyresult}[No global energy conservation in general]
  On a general curved spacetime, Killing vector fields need not
  exist.  In particular, de~Sitter spacetime admits no global
  time-like Killing vector field.  Consequently, there is no
  global notion of conserved energy, and the equation
  $\covd^a T_{ab} = 0$ does \emph{not} imply global energy
  conservation.
\end{keyresult}

\noindent
However, when a Killing vector~$\xi^a$ does exist, we obtain a
genuinely conserved current:
\begin{equation}\label{eq:killing-current}
  J^a = T^{ab}\, \xi_b\,,
  \qquad
  \covd_a J^a = 0\,.
\end{equation}

\begin{exercise}\label{ex:killing-current}
  Prove that if $\xi^a$ is a Killing vector field and
  $\covd^a T_{ab} = 0$, then $\covd_a J^a = 0$ where
  $J^a = T^{ab}\, \xi_b$.
\end{exercise}

\begin{remark}
  Physically, the absence of global energy conservation reflects
  the fact that the gravitational field does work on matter via
  tidal forces, transferring energy between the curvature of
  spacetime and the matter content.  Only when a time-translation
  symmetry exists can one define a total energy that is conserved.
  Equation~\eqref{eq:stress-energy-conservation} may still be
  regarded as expressing \emph{local} conservation of
  energy-momentum in a sufficiently small, approximately flat
  region.
\end{remark}

% ──────────────────────────────────────────────────────────────
\subsection{Einstein's field equations}%
\label{sec:einstein-equations}

We have equations of motion for matter fields
($\covd^a T_{ab} = 0$), but we do not yet have an equation
determining the metric itself.  Since general relativity treats
spacetime as dynamical---its geometry responds to the presence of
matter---we need an equation of motion for~$g_{ab}$.  This is
where Mach's principle enters: the geometry of spacetime is
determined by the distribution of matter and energy.

% ──────────────────────────────────────────────────────────────
\subsubsection{From tidal forces to spacetime curvature}%
\label{sec:tidal-to-curvature}

The key clue comes from comparing the Newtonian description of
tidal forces with the geodesic deviation equation.  In Newtonian
gravity, two test particles on nearby orbits in a gravitational
potential~$\phi$ experience a relative (tidal) acceleration
\begin{equation}\label{eq:newtonian-tidal}
  a^i = -\frac{\partial^2 \phi}{\partial x^i\, \partial x^j}\, \Delta x^j\,,
\end{equation}
where $\Delta x^j$ is the separation vector.  In general
relativity, the geodesic deviation
equation~\eqref{eq:geodesic-deviation} gives
\[
  A^a = -\Riem_{cbd}{}^a\, X^b\, T^c\, T^d\,.
\]
Comparing these suggests the correspondence
\begin{equation}\label{eq:curvature-potential}
  \Riem_{cbd}{}^a\, T^c T^d
  \;\longleftrightarrow\;
  \frac{\partial^2 \phi}{\partial x^i\, \partial x^j}\,.
\end{equation}
In Newtonian theory, $\phi$ is determined by the mass
density~$\rho$ via Poisson's equation
$\nabla^2 \phi = 4\pi\rho$.  Since the energy density observed
by an observer with four-velocity~$v^a$ is
$\rho = T_{ab}\, v^a v^b$, the
correspondence~\eqref{eq:curvature-potential} points toward an
equation relating curvature to~$T_{ab}$.  Contracting over the
velocity directions yields a correspondence between the Ricci
tensor and the stress-energy tensor.

% ──────────────────────────────────────────────────────────────
\subsubsection{The Einstein equation}%
\label{sec:einstein-equation}

A first guess, following the analogy above, would be
$\Ric_{ab} = 8\pi\, T_{ab}$.  However, this is
\emph{inconsistent} with the equation of motion
$\covd^a T_{ab} = 0$.  The contracted Bianchi
identity~\eqref{eq:contracted-bianchi} gives
\[
  \covd^a \Ric_{ab} = \tfrac{1}{2}\, \covd_b \Ric\,,
\]
which vanishes only when the Ricci scalar~$\Ric$ is constant---a
condition not satisfied by most interesting spacetimes.

The resolution is to use the Einstein tensor
$\Ein_{ab} = \Ric_{ab} - \frac{1}{2}\, \Ric\, g_{ab}$, which
\emph{is} divergence-free
($\covd^a \Ein_{ab} = 0$,
cf.~\eqref{eq:einstein-divergence-free}) as a consequence of
the Bianchi identity.

\begin{keyresult}[Einstein's field equations]
  \begin{equation}\label{eq:einstein-field}
    \eqbox{\Ein_{ab}
      = \Ric_{ab} - \tfrac{1}{2}\, \Ric\, g_{ab}
      = 8\pi\, T_{ab}}
  \end{equation}
  This equation relates the curvature of spacetime (left-hand
  side) to the distribution of matter and energy (right-hand
  side).  The factor~$8\pi$ is fixed by requiring agreement with
  Newtonian gravity in the appropriate limit.
\end{keyresult}

\begin{historical}[Einstein's path to the field equations]
  Einstein first proposed $\Ric_{ab} = 8\pi\, T_{ab}$ but quickly
  realised the inconsistency with energy-momentum conservation.
  The resolution---replacing the Ricci tensor with the Einstein
  tensor---was guided by the contracted Bianchi identity.  The
  final form~\eqref{eq:einstein-field} was published by Einstein
  in November~1915.  The mathematical identity
  $\covd^a \Ein_{ab} = 0$ guarantees that Einstein's equation is
  \emph{automatically} consistent with the conservation of
  energy-momentum---the geometry and the matter talk to each other
  in a self-consistent way.
\end{historical}

\begin{remark}
  Einstein's equation~\eqref{eq:einstein-field} is a system of
  ten coupled, nonlinear partial differential equations for the ten
  independent components of~$g_{ab}$.  Finding exact solutions is
  extraordinarily difficult; we will succeed only in a few highly
  symmetric cases.  Much of the remainder of this course is devoted
  to exploring the consequences of this equation.
\end{remark}

% ──────────────────────────────────────────────────────────────
\subsection{Linearised gravity and gauge invariance}%
\label{sec:linearised}

To connect Einstein's equation with Newtonian gravity, we consider
the regime where the gravitational field is weak: the metric is
close to the flat Minkowski metric.

% ──────────────────────────────────────────────────────────────
\subsubsection{Metric perturbations and the linearised Einstein
  tensor}\label{sec:metric-perturbations}

Write the metric as a small perturbation of flat spacetime:
\begin{equation}\label{eq:metric-perturbation}
  g_{ab} = \eta_{ab} + \gamma_{ab}\,,
  \qquad |\gamma_{ab}| \ll 1\,.
\end{equation}
All indices are raised and lowered with~$\eta_{ab}$.  Define the
\textbf{trace-reversed perturbation}:
\begin{equation}\label{eq:trace-reversed}
  \bar\gamma_{ab} = \gamma_{ab}
    - \tfrac{1}{2}\, \eta_{ab}\, \gamma\,,
  \qquad
  \gamma = \eta^{ab}\, \gamma_{ab}\,.
\end{equation}

\noindent
To first order in~$\gamma_{ab}$, Einstein's
equation~\eqref{eq:einstein-field} becomes the \textbf{linearised
Einstein equation}:
\begin{equation}\label{eq:linearised-einstein}
  \Ein_{ab}^{(1)}
    = -\tfrac{1}{2}\, \partial^c \partial_c\, \bar\gamma_{ab}
      + \partial^c \partial_{(a}\, \bar\gamma_{b)c}
      - \tfrac{1}{2}\, \eta_{ab}\, \partial^c \partial^d\,
        \bar\gamma_{cd}
    = 8\pi\, T_{ab}\,.
\end{equation}

% ──────────────────────────────────────────────────────────────
\subsubsection{Gauge freedom and the Lorenz gauge}%
\label{sec:gauge-lorenz}

An infinitesimal diffeomorphism generated by a vector
field~$\xi^a$ transforms the metric perturbation as
\begin{equation}\label{eq:gauge-transformation}
  \gamma_{ab} \;\longrightarrow\;
  \gamma_{ab} + \partial_a \xi_b + \partial_b \xi_a\,.
\end{equation}
This is the linearised version of the diffeomorphism invariance
discussed in \S\ref{sec:diffeo-invariance}: it is the
\textbf{gauge freedom} of linearised gravity, directly analogous
to the gauge freedom $A_a \to A_a + \partial_a \Lambda$ of
electromagnetism.

We exploit this freedom to simplify the field equations.
Choose~$\xi^a$ satisfying the wave equation
\begin{equation}\label{eq:lorenz-gauge-solve}
  \partial^b \partial_b\, \xi_a
    = -\partial^b \bar\gamma_{ab}\,.
\end{equation}
After this gauge transformation, the \textbf{Lorenz gauge
condition}
\begin{equation}\label{eq:lorenz-gauge}
  \partial^b \bar\gamma_{ab} = 0
\end{equation}
is satisfied, and the linearised Einstein
equation~\eqref{eq:linearised-einstein} reduces to
\begin{equation}\label{eq:linearised-wave}
  \eqbox{\partial^c \partial_c\, \bar\gamma_{ab}
    = -16\pi\, T_{ab}}
\end{equation}

\begin{remark}
  In vacuum ($T_{ab} = 0$),
  equation~\eqref{eq:linearised-wave} is the wave equation
  $\dalem\, \bar\gamma_{ab} = 0$ for a massless spin-$2$ field
  propagating on flat spacetime.  Its solutions are
  \textbf{gravitational waves}---ripples in the fabric of
  spacetime---which we will study in a later lecture.
\end{remark}

% ──────────────────────────────────────────────────────────────
\subsection{The Newtonian limit}\label{sec:newtonian-limit}

We now show that Einstein's equation reduces to Newtonian gravity
in the appropriate limit, thereby fixing the constant~$8\pi$
in~\eqref{eq:einstein-field} and providing a powerful consistency
check.

% ──────────────────────────────────────────────────────────────
\subsubsection{Assumptions}\label{sec:newtonian-assumptions}

The Newtonian limit is the regime where:
\begin{enumerate}[(a)]
  \item \textbf{Gravity is weak}: the metric is well approximated
    by~\eqref{eq:metric-perturbation} with
    $|\gamma_{ab}| \ll 1$.
  \item \textbf{Motion is slow}: all velocities are much less
    than~$c$.
  \item \textbf{Stresses are small}: material stresses (pressure)
    are much less than the mass-energy density~$\rho$.
\end{enumerate}
These assumptions allow us to choose a global inertial coordinate
system in which the stress-energy tensor takes the form
\begin{equation}\label{eq:tab-newtonian}
  T_{ab} \approx \rho\, t_a\, t_b\,,
\end{equation}
where $t^a = (\partial / \partial x^0)^a$ is the unit time-like
vector.  Since sources are slowly varying, time derivatives of the
perturbation~$\bar\gamma_{ab}$ are negligible compared to spatial
derivatives.

% ──────────────────────────────────────────────────────────────
\subsubsection{Recovery of Poisson's equation}%
\label{sec:poisson-recovery}

Under these assumptions, the wave
equation~\eqref{eq:linearised-wave} reduces to the Laplace/Poisson
equations:
\[
  \nabla^2 \bar\gamma_{\mu\nu}
    = -16\pi\, T_{\mu\nu}\,.
\]
For $(\mu,\nu) \neq (0,0)$, the right-hand side vanishes, giving
$\nabla^2 \bar\gamma_{\mu\nu} = 0$.  With the boundary condition
$\bar\gamma_{\mu\nu} \to 0$ at spatial infinity, the unique
solution is $\bar\gamma_{\mu\nu} = 0$ for
$(\mu,\nu) \neq (0,0)$.

The $(0,0)$ component gives
\begin{equation}\label{eq:gamma-00}
  \nabla^2 \bar\gamma_{00} = -16\pi\rho\,.
\end{equation}
Defining the \textbf{Newtonian gravitational potential}
\begin{equation}\label{eq:newtonian-potential}
  \phi = -\tfrac{1}{4}\, \bar\gamma_{00}\,,
\end{equation}
equation~\eqref{eq:gamma-00} becomes
\begin{equation}\label{eq:poisson}
  \eqbox{\nabla^2 \phi = 4\pi\rho}
\end{equation}
which is \textbf{Poisson's equation}---the fundamental field
equation of Newtonian gravity.

\medskip\noindent
The full metric perturbation in the Newtonian limit is
\begin{equation}\label{eq:newtonian-metric}
  \gamma_{ab}
    = \bar\gamma_{ab}
      - \tfrac{1}{2}\, \eta_{ab}\, \bar\gamma
    = -(4\, t_a t_b + 2\, \eta_{ab})\,\phi\,,
\end{equation}
giving the line element
\begin{equation}\label{eq:newtonian-line-element}
  ds^2 = -(1 + 2\phi)\, dt^2
       + (1 - 2\phi)(dx^2 + dy^2 + dz^2)\,.
\end{equation}

% ──────────────────────────────────────────────────────────────
\subsubsection{Recovery of Newton's equation of motion}%
\label{sec:newton-eom}

It remains to show that freely falling particles obey Newton's
law.  The geodesic equation in coordinates reads
\[
  \frac{d^2 x^\mu}{d\tau^2}
    + \chris{\mu}{\rho\sigma}\,
      \frac{dx^\rho}{d\tau}\, \frac{dx^\sigma}{d\tau}
    = 0\,.
\]
For a slowly moving test body,
$dx^\mu/d\tau \approx (1,0,0,0)$ and $\tau \approx t$, so
\begin{equation}\label{eq:geodesic-slow}
  \frac{d^2 x^\mu}{dt^2}
    \approx -\chris{\mu}{00}\,.
\end{equation}
From the metric~\eqref{eq:newtonian-metric}, the relevant
Christoffel symbols are (neglecting time derivatives):
\begin{equation}\label{eq:christoffel-newtonian}
  \chris{i}{00}
    = -\tfrac{1}{2}\, \eta^{ij}\,
      \pd{\gamma_{00}}{x^j}
    = \pd{\phi}{x^i}\,,
  \qquad i = 1,2,3\,.
\end{equation}
Substituting into~\eqref{eq:geodesic-slow}:
\begin{equation}\label{eq:newton-law}
  \eqbox{\frac{d^2 x^i}{dt^2} = -\pd{\phi}{x^i}}
\end{equation}
In vector notation, $\vb{a} = -\nabla\phi$: a test particle
accelerates in the direction of decreasing gravitational
potential.  This is precisely \textbf{Newton's law of
gravitation}.

\begin{keyresult}[General relativity contains Newtonian gravity]
  In the limit of weak gravitational fields, slow motion, and
  small stresses, Einstein's field
  equations~\eqref{eq:einstein-field} reduce to Poisson's
  equation~\eqref{eq:poisson} for the gravitational potential, and
  the geodesic equation reduces to Newton's law of
  motion~\eqref{eq:newton-law}.  Newtonian gravity is contained
  within general relativity as a limiting case.
\end{keyresult}

\begin{intuition}[The meaning of $8\pi$]
  The factor of~$8\pi$ in Einstein's
  equation~\eqref{eq:einstein-field} is not arbitrary: it is
  uniquely fixed by the requirement that the Newtonian limit
  reproduces Poisson's equation with the correct coefficient
  $4\pi\rho$.  This is the quantitative content of Mach's
  principle in general relativity.
\end{intuition}

\medskip
With Einstein's field equations in hand and the Newtonian limit
verified, we have established the complete framework of general
relativity.  In the lectures that follow, we will explore the
consequences of Einstein's equation in greater depth: its
linearised solutions (gravitational radiation), exact solutions
in the presence of spherical symmetry (the Schwarzschild
solution), and cosmological models.
