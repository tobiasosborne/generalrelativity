%!TEX root = ../GeneralRelativity.tex
% ──────────────────────────────────────────────────────────────
%  Lecture 15 — Consequences of homogeneity and isotropy:
%              spaces of constant curvature and the FLRW metric
% ──────────────────────────────────────────────────────────────

\subsection{Consequences of homogeneity and isotropy for the metric}%
\label{sec:consequences-hom-iso}

In the previous section we defined what it means for a spacetime to
be homogeneous and isotropic.  Our goal now is to determine the
consequences of these symmetry assumptions for the fundamental
dynamical variable of general relativity: the metric~$g_{ab}$.  The
hope---which will be amply justified---is that by assuming the
symmetries of homogeneity and isotropy the metric is simplified to
the point where Einstein's field equations become tractable.
Indeed, we will reduce the problem of finding the metric of a highly
symmetric model universe to that of finding \emph{a single scalar
function}~$a(\tau)$.

% ──────────────────────────────────────────────────────────────
\subsubsection{The induced metric on the spatial slices}%
\label{sec:induced-metric}

The full spacetime metric~$g_{ab}$ induces, via restriction, a
three-dimensional Riemannian metric~$h_{ab}(t)$ on each spatial
hypersurface~$\Sigma_t$ of the homogeneity foliation.  Rather than
attacking the full four-dimensional problem at once, we first study
the consequences of homogeneity and isotropy for this induced
metric~$h_{ab}$.  Later we will embed $h_{ab}$ back into the
full metric~$g_{ab}$ and argue that very little additional freedom
remains.

For the induced metric, the symmetry conditions take the following
concrete forms:
\begin{itemize}
  \item \textbf{Homogeneity}: there exists an isometry of $h_{ab}(t)$
    taking any point $p \in \Sigma_t$ to any other
    point $q \in \Sigma_t$.
  \item \textbf{Isotropy}: it is impossible to construct a
    geometrically preferred vector in~$\Sigma_t$.
\end{itemize}

\noindent
We will attack the second constraint (isotropy) first and use
homogeneity later to fix a remaining constant.

% ──────────────────────────────────────────────────────────────
\subsubsection{The Riemann tensor as a linear map on two-forms}%
\label{sec:riemann-linear-map}

The beauty of having developed differential geometry in arbitrary
dimension in the earlier lectures is that we can immediately apply it
to three-dimensional manifolds.  From the induced metric~$h_{ab}$ on
$\Sigma_t$ we construct the three-dimensional Riemann curvature
tensor $\Riem^{(3)}_{abc}{}^d$ using the standard
rules~(\S\ref{sec:riemann-tensor}).  Raising an index with the
induced metric gives
\begin{equation}\label{eq:riemann-3-raised}
  \Riem^{(3)}_{ab}{}^{cd}
    = \Riem^{(3)}_{ab}{}^d{}_{c'}\, h^{c'c}\,,
\end{equation}
which we may interpret as a \textbf{linear map}
\[
  L\colon \mathcal{T}(0,2)\big|_p
    \;\longrightarrow\; \mathcal{T}(0,2)\big|_p
\]
at each point $p \in \Sigma_t$.

\begin{exercise}\label{ex:annihilates-symmetric}
  Show that $\Riem^{(3)}_{ab}{}^{cd}$ annihilates symmetric tensors
  in $\mathcal{T}(0,2)$.  Conclude that $L$ restricts to a linear
  map
  \[
    L\colon W \to W\,,
  \]
  where $W$ is the space of \textbf{two-forms} (antisymmetric
  $(0,2)$~tensors) at~$p$.
\end{exercise}

The pair-interchange symmetry of the Riemann tensor,
$\Riem_{abcd} = \Riem_{cdab}$, implies that $L$ is a
\textbf{symmetric} linear map with respect to the inner product
on~$W$ furnished by~$h_{ab}$.  On an inner product space, a
symmetric linear map can be diagonalised and has real eigenvalues.

\medskip
\noindent\textbf{Dimension count.}\;
On a three-dimensional manifold, the space of two-forms has dimension
$\binom{3}{2} = 3$, so $L$ is a $3\times 3$ symmetric matrix with
three real eigenvalues.

% ──────────────────────────────────────────────────────────────
\subsubsection{Isotropy forces equal eigenvalues}%
\label{sec:eigenvalue-argument}

\begin{keyresult}[All eigenvalues of $L$ are equal]
  Isotropy requires that all three eigenvalues of $L$ are
  \emph{equal}.

  If any eigenvalue were distinct, its eigenvector---a two-form
  selected by a basis-independent prescription---would define a
  geometrically preferred two-form at~$p$.  But on a
  three-dimensional manifold, any two-form can be completed to a
  unique three-form (up to sign), and a three-form together with the
  metric determines a preferred vector via the Hodge dual.  This
  would produce a geometrically preferred vector in~$\Sigma_t$,
  contradicting isotropy.
\end{keyresult}

Since all eigenvalues are equal, $L$ is proportional to the identity:
$L \propto \mathbb{I}$.  Translating back to the Riemann tensor
with raised indices:
\begin{equation}\label{eq:riemann-isotropic}
  \Riem^{(3)}_{ab}{}^{cd}
    = K\, \delta^c{}_{[a}\, \delta^d{}_{b]}\,,
\end{equation}
for some scalar~$K$.  Lowering the indices $c$ and~$d$:
\begin{equation}\label{eq:riemann-constant-k}
  \eqbox{\Riem^{(3)}_{abcd}
    = K\bigl(h_{ca}\, h_{db} - h_{cb}\, h_{da}\bigr)}
\end{equation}

% ──────────────────────────────────────────────────────────────
\subsubsection{$K$ is constant: the Bianchi identity argument}%
\label{sec:k-constant}

Is the scalar~$K$ in~\eqref{eq:riemann-constant-k} a function of
position on~$\Sigma_t$, or is it constant?  Homogeneity immediately
tells us it must be constant (an isometry mapping $p$ to $q$
preserves the Riemann tensor, hence $K(p) = K(q)$).  But we can also
prove this from isotropy alone, using the Bianchi identity.

Apply the three-dimensional covariant derivative $D_e$ (associated
to~$h_{ab}$) to~\eqref{eq:riemann-constant-k} and antisymmetrise:
\[
  0 = D_{[e}\, \Riem^{(3)}_{ab]cd}
    = (D_{[e}\, K)\, h_{|c|a}\, h_{d|b]}
\]
where we used $D_e\, h_{ab} = 0$ (metric compatibility).  For
$\dim(\Sigma_t) \geq 3$, the antisymmetrised product of metric
factors on the right is non-degenerate, so the vanishing of the
left-hand side requires
\begin{equation}\label{eq:k-constant-proof}
  D_e\, K = 0\,.
\end{equation}
Therefore $K$ is \emph{constant} on each $\Sigma_t$.

\begin{intuition}[Spaces of constant curvature]
  A Riemannian manifold whose Riemann tensor takes the
  form~\eqref{eq:riemann-constant-k} with $K$ constant is called a
  \textbf{space of constant curvature}.  The physical content of
  this result is transparent: recall that the Riemann tensor measures
  the failure of parallel transport operations to commute.  On a
  space of constant curvature, no matter which way you perform the
  little parallelogram of parallel transport, you get the same
  failure to commute---the curvature is the same in every direction
  and at every point.  This is exactly what one would expect from
  homogeneity and isotropy.
\end{intuition}

% ──────────────────────────────────────────────────────────────
\subsection{The three cases: representative spatial geometries}%
\label{sec:three-cases}

We have established that the spatial slices $\Sigma_t$ are spaces of
constant curvature~$K$.  There are three qualitatively distinct
cases, classified by the sign of~$K$.  In each case one can write
down an explicit representative manifold and metric.

\subsubsection{$K > 0$: the three-sphere (closed universe)}

The spatial slices are three-spheres $S^3$ of radius~$R$
(with $K = 1/R^2$), defined as embedded submanifolds of
$\Rn{4}$:
\[
  x^2 + y^2 + z^2 + w^2 = R^2\,.
\]
(The ambient $\Rn{4}$ is a mathematical convenience; the
three-spheres are intrinsically defined three-dimensional
manifolds.)  In spherical coordinates $(\psi, \theta, \phi)$ the
induced metric reads:
\begin{equation}\label{eq:metric-S3}
  ds^2_{\Sigma}
    = d\psi^2
      + \sin^2\!\psi\,\bigl(d\theta^2
      + \sin^2\!\theta\, d\phi^2\bigr)\,.
\end{equation}
This describes a \textbf{closed} universe: the spatial sections are
compact (finite volume, no boundary), so everything is connected to
everything else---there is no ``going off to infinity.''

\subsubsection{$K = 0$: flat space (open universe)}

The spatial slices are copies of flat $\Rn{3}$:
\begin{equation}\label{eq:metric-flat}
  ds^2_{\Sigma}
    = dx^2 + dy^2 + dz^2\,.
\end{equation}
This is called an \textbf{open} universe (or \textbf{flat}
universe): the spatial sections are non-compact.

\begin{exercise}\label{ex:flat-spherical}
  Rewrite the flat
  metric~\eqref{eq:metric-flat} in spherical coordinates
  $(r, \theta, \phi)$ and verify that it takes the form
  $dr^2 + r^2\,(d\theta^2 + \sin^2\!\theta\, d\phi^2)$.
\end{exercise}

\subsubsection{$K < 0$: the hyperboloid (open universe)}

The spatial slices are three-dimensional hyperboloids of
radius~$R$ (with $K = -1/R^2$), which may be described as
embedded submanifolds of $\Rn{4}$ with Lorentzian signature:
\[
  t^2 - x^2 - y^2 - z^2 = R^2\,.
\]
In hyperbolic coordinates $(\psi, \theta, \phi)$ the metric reads:
\begin{equation}\label{eq:metric-H3}
  ds^2_{\Sigma}
    = d\psi^2
      + \sinh^2\!\psi\,\bigl(d\theta^2
      + \sin^2\!\theta\, d\phi^2\bigr)\,.
\end{equation}
This is also an \textbf{open} universe (non-compact, negatively
curved).

% ──────────────────────────────────────────────────────────────
\subsection{The Friedmann--Lema\^itre--Robertson--Walker metric}%
\label{sec:flrw-metric}

We now embed the spatial slices back into the full four-dimensional
spacetime.  Since the isotropic observers move on worldlines
orthogonal to~$\Sigma_t$ (\S\ref{sec:hom-iso-combined}), we can
express the full 4D metric as:
\begin{equation}\label{eq:4d-metric-hab}
  g_{ab} = -u_a\, u_b + h_{ab}(t)\,,
\end{equation}
where $u^a$ is the unit tangent vector of the isotropic observers and
$h_{ab}(t)$ is one of the three spatial metrics above.

All isotropic observers assign the same proper time~$\tau$ to the
surface~$\Sigma_t$ (by homogeneity, no observer is distinguished).
In coordinates adapted to this proper time, the full metric becomes:

\begin{keyresult}[FLRW metric]
  \begin{equation}\label{eq:flrw}
    \eqbox{ds^2 = -d\tau^2 + a^2(\tau)\, ds^2_{\Sigma}}
  \end{equation}
  where $a(\tau)$ is the \textbf{scale factor} and $ds^2_\Sigma$ is
  one of the three spatial metrics
  \eqref{eq:metric-S3}/\eqref{eq:metric-flat}/\eqref{eq:metric-H3}:
  \[
    ds^2 = -d\tau^2 + a^2(\tau)
    \begin{cases}
      d\psi^2 + \sin^2\!\psi\,
        (d\theta^2 + \sin^2\!\theta\, d\phi^2)
        & (K > 0,\;\text{closed})\\[4pt]
      dx^2 + dy^2 + dz^2
        & (K = 0,\;\text{flat})\\[4pt]
      d\psi^2 + \sinh^2\!\psi\,
        (d\theta^2 + \sin^2\!\theta\, d\phi^2)
        & (K < 0,\;\text{open})
    \end{cases}
  \]
  This is the
  \textbf{Friedmann--Lema\^itre--Robertson--Walker (FLRW) metric}.
\end{keyresult}

\begin{intuition}[An extraordinary simplification]
  By assuming homogeneity and isotropy we have reduced the task of
  finding the metric of our model universe to solving for just one
  single scalar function~$a(\tau)$---the scale factor.  This is a
  huge simplification: instead of ten coupled nonlinear PDEs for the
  ten components of~$g_{\mu\nu}$, we hope (and it is indeed the
  case) that substituting the FLRW metric into Einstein's field
  equations with an appropriate stress-energy tensor on the
  right-hand side will reduce everything to a single ordinary
  differential equation for~$a(\tau)$.  That equation---the
  \textbf{Friedmann equation}---is the subject of the next lecture.
\end{intuition}
