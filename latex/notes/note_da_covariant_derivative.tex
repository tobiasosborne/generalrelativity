\documentclass[11pt,a4paper]{article}

\usepackage{amsmath,amssymb,amsthm}
\usepackage[margin=2.5cm]{geometry}
\usepackage{mathtools}

\newcommand{\pd}[2]{\frac{\partial #1}{\partial #2}}

\theoremstyle{definition}
\newtheorem*{remark}{Remark}

\title{A Note on $\partial_a$ as a Covariant Derivative}
\author{}
\date{}

\begin{document}
\maketitle

Let $T^{a_1 \cdots a_k}{}_{b_1 \cdots b_l}$ be a tensor of type $(k,l)$.
Let $\psi$ be a chart. Then
\begin{equation}\label{eq:tensor-expansion}
  T = \sum_{\substack{\mu_1 \cdots \mu_k \\ \nu_1 \cdots \nu_l}}
    T^{\mu_1 \cdots \mu_k}{}_{\nu_1 \cdots \nu_l}\;
    \pd{}{x^{\mu_1}} \otimes \cdots \otimes \pd{}{x^{\mu_k}}
    \otimes dx^{\nu_1} \otimes \cdots \otimes dx^{\nu_l}.
\end{equation}

\textbf{Definition.}
Define $\partial_a T^{a_1 \cdots a_k}{}_{b_1 \cdots b_l}$ to be the tensor $T'$
whose representation in this basis is
\begin{equation}\label{eq:def-partial}
  T' = \sum_{\substack{\mu_1 \cdots \mu_k \\ \nu_1 \cdots \nu_l}}
    \pd{T^{\mu_1 \cdots \mu_k}{}_{\nu_1 \cdots \nu_l}}{x^\sigma}\;
    \pd{}{x^{\mu_1}} \otimes \cdots \otimes \pd{}{x^{\mu_k}}
    \otimes dx^{\nu_1} \otimes \cdots \otimes dx^{\nu_l}.
\end{equation}

When we change coordinates, the basis elements transform as
\begin{align}
  \pd{}{x^\mu} &\;\longmapsto\; \pd{x^{\mu'}}{x^\mu}\,\pd{}{x^{\mu'}},
  \label{eq:basis-transform} \\[6pt]
  dx^\nu &\;\longmapsto\; \pd{x^{\nu}}{x^{\nu'}}\,dx^{\nu'}.
  \label{eq:cobasis-transform}
\end{align}

Thus, in the new coordinate basis,
\begin{equation}\label{eq:Tprime-transformed}
\begin{split}
  T' = \sum_{\substack{\mu'_1 \cdots \mu'_k \\ \nu'_1 \cdots \nu'_l}}\;
       \sum_{\substack{\mu_1 \cdots \mu_k \\ \nu_1 \cdots \nu_l}}\;
       \sum_{\sigma'}
    &\pd{T^{\mu_1 \cdots \mu_k}{}_{\nu_1 \cdots \nu_l}}{x^{\sigma'}}\;
     \pd{x^{\sigma'}}{x^\sigma}\;
     \pd{x^{\mu'_1}}{x^{\mu_1}} \cdots \pd{x^{\mu'_k}}{x^{\mu_k}}\;
     \pd{x^{\nu_1}}{x^{\nu'_1}} \cdots \pd{x^{\nu_l}}{x^{\nu'_l}} \\[4pt]
    &\quad\times\pd{}{x^{\mu'_1}} \otimes \cdots \otimes \pd{}{x^{\mu'_k}}
     \otimes dx^{\nu'_1} \otimes \cdots \otimes dx^{\nu'_l}.
\end{split}
\end{equation}

Thus, when you change coordinates, the Jacobian-type factors come from the
$\pd{}{x^\mu}$ and $dx^\nu$ terms.
The components of the tensor itself do not transform; rather, the basis
elements of $V_p$ and $V_p^*$ transform.

Thus the tensor $T'$, as defined above, does transform like a tensor.
The reason why this isn't our favourite covariant derivative is that, had we
defined $\partial_a T$ with respect to a \emph{different} chart, we would
obtain different components when we transform back to our original chart.
This derivative operator is \textbf{basis dependent}.

\end{document}
